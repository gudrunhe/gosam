\documentclass[a4paper]{article}
\usepackage{axodraw}
\usepackage{longtable}
\usepackage{listings}
\usepackage{amsmath}
\usepackage{makeidx}
\usepackage{hyperref}

\newcommand{\bra}[1]{\langle #1 \vert}
\newcommand{\brb}[1]{[ #1 \vert}
\newcommand{\kea}[1]{\vert #1 \rangle}
\newcommand{\keb}[1]{\vert #1 ]}
\newcommand{\Spaa}[1]{\langle #1 \rangle}
\newcommand{\Spab}[1]{\langle #1]}
\newcommand{\Spba}[1]{[ #1 \rangle}
\newcommand{\Spbb}[1]{[ #1 ]}

\allowdisplaybreaks[1]

\title{\texttt{GoSam 2.1.0}: ${g}{g}\rightarrow{h}{h}$}
\author{jannisl}
\date{2023-04-26 (13:44:18)}

\renewcommand{\indexname}{Index of all Loop Diagrams}

\makeindex
\begin{document}
\maketitle
\begin{abstract}
\noindent This process consists of no LO diagrams and 10 NLO diagrams. GoSam has identified 3 groups  of NLO diagrams by analyzing their one-loop integrals.
\end{abstract}
\newpage
\tableofcontents
\newpage

\section{Helicities}

\begin{longtable}[c]{r|cccc|c|c}
\bf{Index} &1&2&3&4 & Permutation & Set\\
\hline
\endfirsthead
\bf{Index} &1&2&3&4\\
\hline
\endhead 
$0$& $-$& $-$& $0$& $0$ &  & $0$ \\
$1\rightarrow 0$& $+$& $-$& $0$& $0$ & $(34)$ & $1$ \\
$2\rightarrow 0$& $-$& $+$& $0$& $0$ & $(34)$ & $2$ \\
$3\rightarrow 0$& $+$& $+$& $0$& $0$ & $(34)$ & $3$ \\
\end{longtable}
\section{Wave Functions}
In this section, we use $l_i=k_i$ for massless particles;
in spinors $\kea{i}$ (resp. $\keb{i}$) denote $\kea{l_i}$ (resp. $\keb{l_i}$).
For the massive particles we have:
\begin{align}
l_{3} &= k_{3} - \frac{mdlMh^2}{%
      2 k_{3}\cdot k_{2}}k_{2}\\
l_{4} &= k_{4} - \frac{mdlMh^2}{%
      2 k_{4}\cdot k_{2}}k_{2}
\end{align}

All helicity amplitudes are defined in terms of the following wave functions:
\begin{itemize}
\item $g(k_{1})$ 
% incoming vector particle
\begin{align}
\varepsilon^\mu_+(k_{1}) &=
   \frac{\Spab{2\vert\gamma^\mu\vert 1}}{%
   \sqrt{2}\Spaa{2\vert 1}}\\
\varepsilon^\mu_-(k_{1}) &=
   \frac{\Spba{2\vert\gamma^\mu\vert 1}}{%
   \sqrt{2}\Spbb{1\vert 2}}
\end{align}
\item $g(k_{2})$ 
% incoming vector particle
\begin{align}
\varepsilon^\mu_+(k_{2}) &=
   \frac{\Spab{1\vert\gamma^\mu\vert 2}}{%
   \sqrt{2}\Spaa{1\vert 2}}\\
\varepsilon^\mu_-(k_{2}) &=
   \frac{\Spba{1\vert\gamma^\mu\vert 2}}{%
   \sqrt{2}\Spbb{2\vert 1}}
\end{align}
\item $h(k_3)$ 
\begin{align}
% outgoing scalar particle
\epsilon(k_{3}) &= 1
\end{align}
\item $h(k_4)$ 
\begin{align}
% outgoing scalar particle
\epsilon(k_{4}) &= 1
\end{align}
\end{itemize}

%------------------------------------------------------------------------
\section{Colour Basis}
\begin{align}
\vert c_{1}\rangle &=g^{A_{1}}_{(1)}g^{A_{2}}_{(2)}\textrm{tr}\left\{T^{A_{2}}T^{A_{1}}\right\}
\end{align}




\section{One-Loop Diagrams}
\subsection*{General Information}
\lstinputlisting[title={QGraf Setup},frame=tlrb]{../diagrams-1.log}

Loop diagrams are grouped into sets of diagrams which share
loop-propagators. A loop integral can be written as
\begin{equation}
\int\!\frac{\mathrm{d}^nk}{i\pi^{\frac{n}{2}}}%
\frac{\mathcal{N}(q)}{\prod_{j=1}{N}\left[(k+r_j)^2-(m_j^2
   -i m_j\Gamma_j) + i\delta\right]}.
\end{equation}
For each group we list $r_j$, $m_j$ and $\Gamma_j$.
For $m_j$ and $\Gamma_j$ only non-vanishing symbols are listed.
Furthermore, we give the matrix $S$ which is defined as
\begin{equation}
S_{\alpha\beta} = (r_\alpha-r_\beta)^2
-(m_\alpha^2-im_\alpha\Gamma_\alpha)
-(m_\beta^2-im_\beta\Gamma_\beta).
\end{equation}
For each diagram we denote how the matrix $S^\prime$ for the specific diagram
is obtained from the original~$S$. The notation
\begin{equation}
S^\prime=S_{Q\to q^\prime}^{\{l_1,l_2,\ldots\}}
\end{equation}
means, that the rows and columns labeled by $l_1,l_2,\ldots$ should be
removed from $S$ (likewise $r_{l_1}, r_{l_2}, \ldots$ are removed from the
list of propagators) and $\mathcal{N}(q)$ has to be replaced by
$\mathcal{N}(q^\prime)$.
The maximum effective rank of a group is the rank that has to be passed
to \textsc{Samurai} if the whole group is reduced at once; this number
is calculated as
\begin{equation}
\max_{\text{diagrams}}\left\{(\text{rank of diagram})+
(\text{number of pinches})\right\}.
\end{equation}
Diagrams with massless closed quark lines are multiplied by a factor
$\mathtt{Nfrat}=\mathtt{Nf}/\mathtt{Nfgen}$. This multiplication is indicated
by the symbol $N_f$ following the rank. By default $\mathtt{Nfrat}$ evaluates
to one but can be changed by modifying $\mathtt{Nf}$ or $\mathtt{Nfgen}$ in the
model~file.


\subsection{Group~0 (4-Point)}
\subsubsection*{General Information}
The maximum effective rank in this group is~4.

\begin{subequations}
\begin{align}
r_{1} &= -k_{2}+k_{4},\quad m_{1} = \text{MT}\\
r_{2} &= -k_{2},\quad m_{2} = \text{MT}\\
r_{3} &= 0,\quad m_{3} = \text{MT}\\
r_{4} &= -k_{3},\quad m_{4} = \text{MT}
\end{align}
\end{subequations}

\begin{equation}
S=\left(\begin{array}{cccc}
   S_{1,1}&
   S_{1,2}&
   S_{1,3}&
   S_{1,4}\\
   S_{2,1}&
   S_{2,2}&
   S_{2,3}&
   S_{2,4}\\
   S_{3,1}&
   S_{3,2}&
   S_{3,3}&
   S_{3,4}\\
   S_{4,1}&
   S_{4,2}&
   S_{4,3}&
   S_{4,4}\end{array}\right)
\end{equation}
\begin{subequations}
\begin{align}
   S_{1,1}&=-2\text{MT}^2\\
   S_{1,2}&=\text{Mh}^2-2\text{MT}^2\\
   S_{1,3}&=2\text{Mh}^2-s_{23}-s_{12}-2\text{MT}^2\\
   S_{1,4}&=-2\text{MT}^2\\
   S_{2,2}&=-2\text{MT}^2\\
   S_{2,3}&=-2\text{MT}^2\\
   S_{2,4}&=s_{23}-2\text{MT}^2\\
   S_{3,3}&=-2\text{MT}^2\\
   S_{3,4}&=\text{Mh}^2-2\text{MT}^2\\
   S_{4,4}&=-2\text{MT}^2
\end{align}
\end{subequations}

\subsubsection{Diagrams (2)}\begin{longtable}{cc}
\endfirsthead
\endhead
%---#[ loop diagram13:
\index{Diagram0000000013=Diagram 13 (Group 0)}
\hbox{
\begin{minipage}{0.45\textwidth}
\begin{center}
% Diagram 13:
\begin{picture}(140,120)(-10,-10)
   \Gluon(102.4,14.6)(77.7,35.3){3}{6} % part21-propagator
   \Text(100.5,16.9)[lt]{$g(k_{1})$}
   \Gluon(17.6,85.4)(42.3,64.7){3}{6} % part21-propagator
   \Text(19.5,83.1)[rb]{$g(k_{2})$}
   \DashLine(77.7,64.7)(102.4,85.4){5} % part5000000-propagator
   \Text(100.5,83.1)[lb]{$h(k_{3})$}
   \DashLine(42.3,35.3)(17.6,14.6){5} % part5000000-propagator
   \Text(19.5,16.9)[rt]{$h(k_{4})$}
   \Vertex(77.7,35.3){3} % part6-anti6-part21 vertex
   \Vertex(77.7,64.7){3} % part6-anti6-part5000000 vertex
   \Vertex(42.3,35.3){3} % part6-anti6-part5000000 vertex
   \Vertex(42.3,64.7){3} % part6-anti6-part21 vertex
   \ArrowLine(77.7,64.7)(77.7,35.3) % anti6-propagator
   \Text(80.7,50.0)[lt]{$t$}
   \ArrowLine(77.7,35.3)(42.3,35.3) % anti6-propagator
   \Text(60.0,32.3)[rt]{$t$}
   \ArrowLine(42.3,64.7)(77.7,64.7) % anti6-propagator
   \Text(60.0,67.7)[rb]{$t$}
   \ArrowLine(42.3,35.3)(42.3,64.7) % anti6-propagator
   \Text(39.3,50.0)[rt]{$t$}
\end{picture}
\\
{\sl -Diagram~13}\\
$S^\prime=S_{Q\to -q-(-k3)}$, $\mathrm{rk}=4$
\end{center}
\end{minipage}}
&
%---#] loop diagram13:

%---#[ loop diagram14:
\index{Diagram0000000014=Diagram 14 (Group 0)}
\hbox{
\begin{minipage}{0.45\textwidth}
\begin{center}
% Diagram 14:
\begin{picture}(140,120)(-10,-10)
   \Gluon(102.4,85.4)(77.7,64.7){3}{6} % part21-propagator
   \Text(104.3,87.7)[lb]{$g(k_{1})$}
   \Gluon(17.6,14.6)(42.3,35.3){3}{6} % part21-propagator
   \Text(15.7,12.3)[rt]{$g(k_{2})$}
   \DashLine(77.7,35.3)(102.4,14.6){5} % part5000000-propagator
   \Text(104.3,12.3)[lt]{$h(k_{3})$}
   \DashLine(42.3,64.7)(17.6,85.4){5} % part5000000-propagator
   \Text(15.7,87.7)[rb]{$h(k_{4})$}
   \Vertex(77.7,64.7){3} % part6-anti6-part21 vertex
   \Vertex(77.7,35.3){3} % part6-anti6-part5000000 vertex
   \Vertex(42.3,64.7){3} % part6-anti6-part5000000 vertex
   \Vertex(42.3,35.3){3} % part6-anti6-part21 vertex
   \ArrowLine(77.7,64.7)(77.7,35.3) % anti6-propagator
   \Text(80.7,50.0)[lt]{$t$}
   \ArrowLine(42.3,64.7)(77.7,64.7) % anti6-propagator
   \Text(60.0,67.7)[rb]{$t$}
   \ArrowLine(77.7,35.3)(42.3,35.3) % anti6-propagator
   \Text(60.0,32.3)[rt]{$t$}
   \ArrowLine(42.3,35.3)(42.3,64.7) % anti6-propagator
   \Text(39.3,50.0)[rt]{$t$}
\end{picture}
\\
{\sl -Diagram~14}\\
$S^\prime=S_{Q\to -q-(-k3)}$, $\mathrm{rk}=4$
\end{center}
\end{minipage}}

\end{longtable}

%---#] loop diagram14:

\subsection{Group~1 (4-Point)}
\subsubsection*{General Information}
The maximum effective rank in this group is~4.

\begin{subequations}
\begin{align}
r_{1} &= -k_{3}-k_{4},\quad m_{1} = \text{MT}\\
r_{2} &= -k_{3},\quad m_{2} = \text{MT}\\
r_{3} &= 0,\quad m_{3} = \text{MT}\\
r_{4} &= -k_{2},\quad m_{4} = \text{MT}
\end{align}
\end{subequations}

\begin{equation}
S=\left(\begin{array}{cccc}
   S_{1,1}&
   S_{1,2}&
   S_{1,3}&
   S_{1,4}\\
   S_{2,1}&
   S_{2,2}&
   S_{2,3}&
   S_{2,4}\\
   S_{3,1}&
   S_{3,2}&
   S_{3,3}&
   S_{3,4}\\
   S_{4,1}&
   S_{4,2}&
   S_{4,3}&
   S_{4,4}\end{array}\right)
\end{equation}
\begin{subequations}
\begin{align}
   S_{1,1}&=-2\text{MT}^2\\
   S_{1,2}&=\text{Mh}^2-2\text{MT}^2\\
   S_{1,3}&=s_{12}-2\text{MT}^2\\
   S_{1,4}&=-2\text{MT}^2\\
   S_{2,2}&=-2\text{MT}^2\\
   S_{2,3}&=\text{Mh}^2-2\text{MT}^2\\
   S_{2,4}&=s_{23}-2\text{MT}^2\\
   S_{3,3}&=-2\text{MT}^2\\
   S_{3,4}&=-2\text{MT}^2\\
   S_{4,4}&=-2\text{MT}^2
\end{align}
\end{subequations}

\subsubsection{Diagrams (2)}\begin{longtable}{cc}
\endfirsthead
\endhead
%---#[ loop diagram11:
\index{Diagram0000000011=Diagram 11 (Group 1)}
\hbox{
\begin{minipage}{0.45\textwidth}
\begin{center}
% Diagram 11:
\begin{picture}(140,120)(-10,-10)
   \Gluon(102.4,14.6)(77.7,35.3){3}{6} % part21-propagator
   \Text(100.5,16.9)[lt]{$g(k_{1})$}
   \Gluon(102.4,85.4)(77.7,64.7){3}{6} % part21-propagator
   \Text(104.3,87.7)[lb]{$g(k_{2})$}
   \DashLine(42.3,64.7)(17.6,85.4){5} % part5000000-propagator
   \Text(15.7,87.7)[rb]{$h(k_{3})$}
   \DashLine(42.3,35.3)(17.6,14.6){5} % part5000000-propagator
   \Text(19.5,16.9)[rt]{$h(k_{4})$}
   \Vertex(77.7,35.3){3} % part6-anti6-part21 vertex
   \Vertex(77.7,64.7){3} % part6-anti6-part21 vertex
   \Vertex(42.3,35.3){3} % part6-anti6-part5000000 vertex
   \Vertex(42.3,64.7){3} % part6-anti6-part5000000 vertex
   \ArrowLine(77.7,64.7)(77.7,35.3) % anti6-propagator
   \Text(80.7,50.0)[lt]{$t$}
   \ArrowLine(77.7,35.3)(42.3,35.3) % anti6-propagator
   \Text(60.0,32.3)[rt]{$t$}
   \ArrowLine(42.3,64.7)(77.7,64.7) % anti6-propagator
   \Text(60.0,67.7)[rb]{$t$}
   \ArrowLine(42.3,35.3)(42.3,64.7) % anti6-propagator
   \Text(39.3,50.0)[rt]{$t$}
\end{picture}
\\
{\sl -Diagram~11}\\
$S^\prime=S_{Q\to q-(k2)}$, $\mathrm{rk}=4$
\end{center}
\end{minipage}}
&
%---#] loop diagram11:

%---#[ loop diagram12:
\index{Diagram0000000012=Diagram 12 (Group 1)}
\hbox{
\begin{minipage}{0.45\textwidth}
\begin{center}
% Diagram 12:
\begin{picture}(140,120)(-10,-10)
   \Gluon(102.4,85.4)(77.7,64.7){3}{6} % part21-propagator
   \Text(104.3,87.7)[lb]{$g(k_{1})$}
   \Gluon(102.4,14.6)(77.7,35.3){3}{6} % part21-propagator
   \Text(100.5,16.9)[lt]{$g(k_{2})$}
   \DashLine(42.3,35.3)(17.6,14.6){5} % part5000000-propagator
   \Text(19.5,16.9)[rt]{$h(k_{3})$}
   \DashLine(42.3,64.7)(17.6,85.4){5} % part5000000-propagator
   \Text(15.7,87.7)[rb]{$h(k_{4})$}
   \Vertex(77.7,64.7){3} % part6-anti6-part21 vertex
   \Vertex(77.7,35.3){3} % part6-anti6-part21 vertex
   \Vertex(42.3,64.7){3} % part6-anti6-part5000000 vertex
   \Vertex(42.3,35.3){3} % part6-anti6-part5000000 vertex
   \ArrowLine(77.7,64.7)(77.7,35.3) % anti6-propagator
   \Text(80.7,50.0)[lt]{$t$}
   \ArrowLine(42.3,64.7)(77.7,64.7) % anti6-propagator
   \Text(60.0,67.7)[rb]{$t$}
   \ArrowLine(77.7,35.3)(42.3,35.3) % anti6-propagator
   \Text(60.0,32.3)[rt]{$t$}
   \ArrowLine(42.3,35.3)(42.3,64.7) % anti6-propagator
   \Text(39.3,50.0)[rt]{$t$}
\end{picture}
\\
{\sl -Diagram~12}\\
$S^\prime=S_{Q\to q-(k2)}$, $\mathrm{rk}=4$
\end{center}
\end{minipage}}

\end{longtable}

%---#] loop diagram12:

\subsection{Group~2 (4-Point)}
\subsubsection*{General Information}
The maximum effective rank in this group is~4.

\begin{subequations}
\begin{align}
r_{1} &= -k_{3}-k_{4},\quad m_{1} = \text{MT}\\
r_{2} &= -k_{4},\quad m_{2} = \text{MT}\\
r_{3} &= 0,\quad m_{3} = \text{MT}\\
r_{4} &= -k_{2},\quad m_{4} = \text{MT}
\end{align}
\end{subequations}

\begin{equation}
S=\left(\begin{array}{cccc}
   S_{1,1}&
   S_{1,2}&
   S_{1,3}&
   S_{1,4}\\
   S_{2,1}&
   S_{2,2}&
   S_{2,3}&
   S_{2,4}\\
   S_{3,1}&
   S_{3,2}&
   S_{3,3}&
   S_{3,4}\\
   S_{4,1}&
   S_{4,2}&
   S_{4,3}&
   S_{4,4}\end{array}\right)
\end{equation}
\begin{subequations}
\begin{align}
   S_{1,1}&=-2\text{MT}^2\\
   S_{1,2}&=\text{Mh}^2-2\text{MT}^2\\
   S_{1,3}&=s_{12}-2\text{MT}^2\\
   S_{1,4}&=-2\text{MT}^2\\
   S_{2,2}&=-2\text{MT}^2\\
   S_{2,3}&=\text{Mh}^2-2\text{MT}^2\\
   S_{2,4}&=2\text{Mh}^2-s_{23}-s_{12}-2\text{MT}^2\\
   S_{3,3}&=-2\text{MT}^2\\
   S_{3,4}&=-2\text{MT}^2\\
   S_{4,4}&=-2\text{MT}^2
\end{align}
\end{subequations}

\subsubsection{Diagrams (6)}\begin{longtable}{cc}
\endfirsthead
\endhead
%---#[ loop diagram2:
\index{Diagram0000000002=Diagram 2 (Group 2)}
\hbox{
\begin{minipage}{0.45\textwidth}
\begin{center}
% Diagram 2:
\begin{picture}(140,120)(-10,-10)
   \Gluon(102.4,14.6)(81.2,35.9){3}{6} % part21-propagator
   \Text(100.3,16.8)[lt]{$g(k_{1})$}
   \Gluon(102.4,85.4)(81.2,64.1){3}{6} % part21-propagator
   \Text(104.5,87.5)[lb]{$g(k_{2})$}
   \DashLine(38.8,50.0)(17.6,14.6){5} % part5000000-propagator
   \Text(20.1,16.2)[rt]{$h(k_{3})$}
   \DashLine(38.8,50.0)(17.6,85.4){5} % part5000000-propagator
   \Text(15.0,86.9)[rb]{$h(k_{4})$}
   \Vertex(81.2,35.9){3} % part6-anti6-part21 vertex
   \Vertex(81.2,64.1){3} % part6-anti6-part21 vertex
   \Vertex(38.8,50.0){3} % part6-anti6-part5000000-part5000000 vertex
   \ArrowLine(81.2,64.1)(81.2,35.9) % anti6-propagator
   \Text(84.2,50.0)[lt]{$t$}
   \ArrowLine(81.2,35.9)(38.8,50.0) % anti6-propagator
   \Text(59.1,40.1)[rt]{$t$}
   \ArrowLine(38.8,50.0)(81.2,64.1) % anti6-propagator
   \Text(59.1,59.9)[rb]{$t$}
\end{picture}
\\
{\sl -Diagram~2}\\
$S^\prime=S^{\{2\}}_{Q\to q-(k2)}$, $\mathrm{rk}=3$
\end{center}
\end{minipage}}
&
%---#] loop diagram2:

%---#[ loop diagram3:
\index{Diagram0000000003=Diagram 3 (Group 2)}
\hbox{
\begin{minipage}{0.45\textwidth}
\begin{center}
% Diagram 3:
\begin{picture}(140,120)(-10,-10)
   \Gluon(102.4,85.4)(81.2,64.1){3}{6} % part21-propagator
   \Text(104.5,87.5)[lb]{$g(k_{1})$}
   \Gluon(102.4,14.6)(81.2,35.9){3}{6} % part21-propagator
   \Text(100.3,16.8)[lt]{$g(k_{2})$}
   \DashLine(38.8,50.0)(17.6,14.6){5} % part5000000-propagator
   \Text(20.1,16.2)[rt]{$h(k_{3})$}
   \DashLine(38.8,50.0)(17.6,85.4){5} % part5000000-propagator
   \Text(15.0,86.9)[rb]{$h(k_{4})$}
   \Vertex(81.2,64.1){3} % part6-anti6-part21 vertex
   \Vertex(81.2,35.9){3} % part6-anti6-part21 vertex
   \Vertex(38.8,50.0){3} % part6-anti6-part5000000-part5000000 vertex
   \ArrowLine(81.2,64.1)(81.2,35.9) % anti6-propagator
   \Text(84.2,50.0)[lt]{$t$}
   \ArrowLine(38.8,50.0)(81.2,64.1) % anti6-propagator
   \Text(59.1,59.9)[rb]{$t$}
   \ArrowLine(81.2,35.9)(38.8,50.0) % anti6-propagator
   \Text(59.1,40.1)[rt]{$t$}
\end{picture}
\\
{\sl -Diagram~3}\\
$S^\prime=S^{\{2\}}_{Q\to q-(k2)}$, $\mathrm{rk}=3$
\end{center}
\end{minipage}}
\\
%---#] loop diagram3:

%---#[ loop diagram7:
\index{Diagram0000000007=Diagram 7 (Group 2)}
\hbox{
\begin{minipage}{0.45\textwidth}
\begin{center}
% Diagram 7:
\begin{picture}(140,120)(-10,-10)
   \Gluon(102.4,14.6)(88.3,35.9){3}{5} % part21-propagator
   \Text(99.9,16.3)[lt]{$g(k_{1})$}
   \Gluon(102.4,85.4)(88.3,64.1){3}{5} % part21-propagator
   \Text(104.9,87.0)[lb]{$g(k_{2})$}
   \DashLine(31.7,50.0)(17.6,14.6){5} % part5000000-propagator
   \Text(20.4,15.8)[rt]{$h(k_{3})$}
   \DashLine(31.7,50.0)(17.6,85.4){5} % part5000000-propagator
   \Text(14.8,86.5)[rb]{$h(k_{4})$}
   \Vertex(31.7,50.0){3} % part5000000-part5000000-part5000000 vertex
   \Vertex(88.3,35.9){3} % part6-anti6-part21 vertex
   \Vertex(88.3,64.1){3} % part6-anti6-part21 vertex
   \Vertex(60.0,50.0){3} % part6-anti6-part5000000 vertex
   \DashLine(60.0,50.0)(31.7,50.0){5} % part5000000-propagator
   \Text(45.9,47.0)[rt]{$h$}
   \ArrowLine(88.3,64.1)(88.3,35.9) % anti6-propagator
   \Text(91.3,50.0)[lt]{$t$}
   \ArrowLine(88.3,35.9)(60.0,50.0) % anti6-propagator
   \Text(72.8,40.2)[rt]{$t$}
   \ArrowLine(60.0,50.0)(88.3,64.1) % anti6-propagator
   \Text(72.8,59.8)[rb]{$t$}
\end{picture}
\\
{\sl -Diagram~7}\\
$S^\prime=S^{\{2\}}_{Q\to q-(k2)}$, $\mathrm{rk}=3$
\end{center}
\end{minipage}}
&
%---#] loop diagram7:

%---#[ loop diagram8:
\index{Diagram0000000008=Diagram 8 (Group 2)}
\hbox{
\begin{minipage}{0.45\textwidth}
\begin{center}
% Diagram 8:
\begin{picture}(140,120)(-10,-10)
   \Gluon(102.4,85.4)(88.3,64.1){3}{5} % part21-propagator
   \Text(104.9,87.0)[lb]{$g(k_{1})$}
   \Gluon(102.4,14.6)(88.3,35.9){3}{5} % part21-propagator
   \Text(99.9,16.3)[lt]{$g(k_{2})$}
   \DashLine(31.7,50.0)(17.6,14.6){5} % part5000000-propagator
   \Text(20.4,15.8)[rt]{$h(k_{3})$}
   \DashLine(31.7,50.0)(17.6,85.4){5} % part5000000-propagator
   \Text(14.8,86.5)[rb]{$h(k_{4})$}
   \Vertex(31.7,50.0){3} % part5000000-part5000000-part5000000 vertex
   \Vertex(88.3,64.1){3} % part6-anti6-part21 vertex
   \Vertex(88.3,35.9){3} % part6-anti6-part21 vertex
   \Vertex(60.0,50.0){3} % part6-anti6-part5000000 vertex
   \DashLine(60.0,50.0)(31.7,50.0){5} % part5000000-propagator
   \Text(45.9,47.0)[rt]{$h$}
   \ArrowLine(88.3,64.1)(88.3,35.9) % anti6-propagator
   \Text(91.3,50.0)[lt]{$t$}
   \ArrowLine(60.0,50.0)(88.3,64.1) % anti6-propagator
   \Text(72.8,59.8)[rb]{$t$}
   \ArrowLine(88.3,35.9)(60.0,50.0) % anti6-propagator
   \Text(72.8,40.2)[rt]{$t$}
\end{picture}
\\
{\sl -Diagram~8}\\
$S^\prime=S^{\{2\}}_{Q\to q-(k2)}$, $\mathrm{rk}=3$
\end{center}
\end{minipage}}
\\
%---#] loop diagram8:

%---#[ loop diagram9:
\index{Diagram0000000009=Diagram 9 (Group 2)}
\hbox{
\begin{minipage}{0.45\textwidth}
\begin{center}
% Diagram 9:
\begin{picture}(140,120)(-10,-10)
   \Gluon(102.4,14.6)(77.7,35.3){3}{6} % part21-propagator
   \Text(100.5,16.9)[lt]{$g(k_{1})$}
   \Gluon(102.4,85.4)(77.7,64.7){3}{6} % part21-propagator
   \Text(104.3,87.7)[lb]{$g(k_{2})$}
   \DashLine(42.3,35.3)(17.6,14.6){5} % part5000000-propagator
   \Text(19.5,16.9)[rt]{$h(k_{3})$}
   \DashLine(42.3,64.7)(17.6,85.4){5} % part5000000-propagator
   \Text(15.7,87.7)[rb]{$h(k_{4})$}
   \Vertex(77.7,35.3){3} % part6-anti6-part21 vertex
   \Vertex(77.7,64.7){3} % part6-anti6-part21 vertex
   \Vertex(42.3,35.3){3} % part6-anti6-part5000000 vertex
   \Vertex(42.3,64.7){3} % part6-anti6-part5000000 vertex
   \ArrowLine(77.7,64.7)(77.7,35.3) % anti6-propagator
   \Text(80.7,50.0)[lt]{$t$}
   \ArrowLine(77.7,35.3)(42.3,35.3) % anti6-propagator
   \Text(60.0,32.3)[rt]{$t$}
   \ArrowLine(42.3,64.7)(77.7,64.7) % anti6-propagator
   \Text(60.0,67.7)[rb]{$t$}
   \ArrowLine(42.3,35.3)(42.3,64.7) % anti6-propagator
   \Text(39.3,50.0)[rt]{$t$}
\end{picture}
\\
{\sl -Diagram~9}\\
$S^\prime=S_{Q\to q-(k2)}$, $\mathrm{rk}=4$
\end{center}
\end{minipage}}
&
%---#] loop diagram9:

%---#[ loop diagram10:
\index{Diagram0000000010=Diagram 10 (Group 2)}
\hbox{
\begin{minipage}{0.45\textwidth}
\begin{center}
% Diagram 10:
\begin{picture}(140,120)(-10,-10)
   \Gluon(102.4,85.4)(77.7,64.7){3}{6} % part21-propagator
   \Text(104.3,87.7)[lb]{$g(k_{1})$}
   \Gluon(102.4,14.6)(77.7,35.3){3}{6} % part21-propagator
   \Text(100.5,16.9)[lt]{$g(k_{2})$}
   \DashLine(42.3,64.7)(17.6,85.4){5} % part5000000-propagator
   \Text(15.7,87.7)[rb]{$h(k_{3})$}
   \DashLine(42.3,35.3)(17.6,14.6){5} % part5000000-propagator
   \Text(19.5,16.9)[rt]{$h(k_{4})$}
   \Vertex(77.7,64.7){3} % part6-anti6-part21 vertex
   \Vertex(77.7,35.3){3} % part6-anti6-part21 vertex
   \Vertex(42.3,64.7){3} % part6-anti6-part5000000 vertex
   \Vertex(42.3,35.3){3} % part6-anti6-part5000000 vertex
   \ArrowLine(77.7,64.7)(77.7,35.3) % anti6-propagator
   \Text(80.7,50.0)[lt]{$t$}
   \ArrowLine(42.3,64.7)(77.7,64.7) % anti6-propagator
   \Text(60.0,67.7)[rb]{$t$}
   \ArrowLine(77.7,35.3)(42.3,35.3) % anti6-propagator
   \Text(60.0,32.3)[rt]{$t$}
   \ArrowLine(42.3,35.3)(42.3,64.7) % anti6-propagator
   \Text(39.3,50.0)[rt]{$t$}
\end{picture}
\\
{\sl -Diagram~10}\\
$S^\prime=S_{Q\to q-(k2)}$, $\mathrm{rk}=4$
\end{center}
\end{minipage}}

\end{longtable}

%---#] loop diagram10:


\printindex

\section{Related Work}
If you publish results obtained by using this matrix element code
please cite the appropriate papers in the bibliography of this document.

Scientific publications prepared using the present version of
\textsc{GoSam} or any modified version of it or any code linking to
\textsc{GoSam} or parts of it should make a clear
reference to the publications~\cite{Cullen:2014yla,Cullen:2011ac}.

For graph generation we use QGraf~\cite{Nogueira:1991ex}.
The Feynman diagrams are further processed with the symbolic manipulation
program FORM~\cite{Kuipers:2012rf,Vermaseren:2000nd} using the FORM library
SPINNEY~\cite{Cullen:2010jv}.
The Fortran~90 code is generated using
FORM~\cite{Kuipers:2012rf,Vermaseren:2000nd}.
For the reduction of the tensor integrals
the code uses an implementation of the Laurent series expansion
method~\cite{Mastrolia:2012bu}
from the library Ninja~\cite{Peraro:2014cba}.


Please, make sure, you also give credit to the authors of the scalar
loop libraries, if you configured the amplitude code such that it calls
other libraries than the ones mentioned so far. Depending on your
configuration you might use one or more of the following programs for
the evaluation of the scalar integrals:
\begin{itemize}
\item OneLOop~\cite{vanHameren:2010cp},
\item QCDLoop~\cite{Ellis:2007qk}, which uses FF~\cite{vanOldenborgh:1990yc},
\item LoopTools~\cite{Hahn:1998yk}, which uses FF~\cite{vanOldenborgh:1990yc}.
\item GOLEM95~\cite{Binoth:2008uq,Guillet:2013msa} which uses OneLOop~\cite{vanHameren:2010cp}
   and may be configured such that it uses
   LoopTools~\cite{Hahn:1998yk,vanOldenborgh:1990yc}.
\end{itemize}

\begin{thebibliography}{ABC}
%\cite{Cullen:2014yla}
\bibitem{Cullen:2014yla}
  G.~Cullen, H.~van Deurzen, N.~Greiner, G.~Heinrich, G.~Luisoni, P.~Mastrolia, E.~Mirabella and G.~Ossola {\it et al.},
  ``GoSam-2.0: a tool for automated one-loop calculations within the Standard Model and beyond,''
  Eur.\ Phys.\ J.\ C {\bf 74} (2014) 8,  3001
  [\href{http://arxiv.org/abs/1404.7096}{arXiv:1404.7096 [hep-ph]}].
  %%CITATION = ARXIV:1404.7096;%%
%\cite{Cullen:2011ac}
\bibitem{Cullen:2011ac}
  G.~Cullen, N.~Greiner, G.~Heinrich, G.~Luisoni, P.~Mastrolia, G.~Ossola, T.~Reiter and F.~Tramontano,
  ``Automated One-Loop Calculations with GoSam,''
  Eur.\ Phys.\ J.\ C {\bf 72} (2012) 1889
  [\href{http://arxiv.org/abs/1111.2034}{arXiv:1111.2034 [hep-ph]}].
  %%CITATION = ARXIV:1111.2034;%%
%\cite{Nogueira:1991ex}
\bibitem{Nogueira:1991ex}
  P.~Nogueira,
  ``Automatic Feynman graph generation,''
  J.\ Comput.\ Phys.\  {\bf 105} (1993) 279.
  %%CITATION = JCTPA,105,279;%%
%\cite{Kuipers:2012rf}
\bibitem{Kuipers:2012rf}
  J.~Kuipers, T.~Ueda, J.~A.~M.~Vermaseren and J.~Vollinga,
  ``FORM version 4.0,''
  Comput.\ Phys.\ Commun.\  {\bf 184} (2013) 1453
  [\href{http://arxiv.org/abs/1203.6543}{arXiv:1203.6543 [cs.SC]}].
  %%CITATION = ARXIV:1203.6543;%%
%\cite{Vermaseren:2000nd}
\bibitem{Vermaseren:2000nd}
  J.~A.~M.~Vermaseren,
  ``New features of FORM,''
  arXiv:math-ph/0010025.
  %%CITATION = MATH-PH/0010025;%%
%\cite{Cullen:2010jv}
\bibitem{Cullen:2010jv}
  G.~Cullen, M.~Koch-Janusz and T.~Reiter,
  ``spinney: A Form Library for Helicity Spinors,''
  \href{http://arxiv.org/abs/1008.0803}{arXiv:1008.0803 [hep-ph]}.
  %%CITATION = ARXIV:1008.0803;%%
%\cite{Reiter:2009ts}
%\cite{Peraro:2014cba}
\bibitem{Peraro:2014cba}
  T.~Peraro,
  ``Ninja: Automated Integrand Reduction via Laurent Expansion for One-Loop Amplitudes,''
  Comput.\ Phys.\ Commun.\  {\bf 185} (2014) 2771
  [\href{http://arxiv.org/abs/1403.1229}{arXiv:1403.1229 [hep-ph]}].
  %%CITATION = ARXIV:1403.1229;%%
%\cite{Mastrolia:2012bu}
\bibitem{Mastrolia:2012bu}
  P.~Mastrolia, E.~Mirabella and T.~Peraro,
  ``Integrand reduction of one-loop scattering amplitudes through Laurent series expansion,''
  JHEP {\bf 1206} (2012) 095
   [Erratum-ibid.\  {\bf 1211} (2012) 128]
  [\href{http://arxiv.org/abs/1203.0291}{arXiv:1203.0291 [hep-ph]}].
  %%CITATION = ARXIV:1203.0291;%%
%\cite{Guillet:2013msa}
\bibitem{Guillet:2013msa}
  J.~P.~Guillet, G.~Heinrich and J.~F.~von Soden-Fraunhofen,
  ``Tools for NLO automation: extension of the golem95C integral library,''
  Comput.\ Phys.\ Commun.\  {\bf 185} (2014) 1828
  [\href{http://arxiv.org/abs/1312.3887}{arXiv:1312.3887 [hep-ph]}].
  %%CITATION = ARXIV:1312.3887;%%
%\cite{Binoth:2008uq}
\bibitem{Binoth:2008uq}
  T.~Binoth, J.~P.~Guillet, G.~Heinrich, E.~Pilon and T.~Reiter,
  ``Golem95: a numerical program to calculate one-loop tensor integrals with up
  to six external legs,''
  Comput.\ Phys.\ Commun.\  {\bf 180} (2009) 2317
  [\href{http://arxiv.org/abs/0810.0992}{arXiv:0810.0992 [hep-ph]}].
  %%CITATION = CPHCB,180,2317;%%
%\cite{Cullen:2011kv}
\bibitem{Cullen:2011kv}
  G.~Cullen, J.~P.~.Guillet, G.~Heinrich, T.~Kleinschmidt, E.~Pilon, T.~Reiter, M.~Rodgers,
  ``Golem95C: A library for one-loop integrals with complex masses,''
  Comput.\ Phys.\ Commun.\  {\bf 182 } (2011)  2276-2284.
  [\href{http://arxiv.org/abs/1101.5595}{arXiv:1101.5595 [hep-ph]}].
%\cite{vanHameren:2010cp}
\bibitem{vanHameren:2010cp}
  A.~van Hameren,
  ``OneLOop: For the evaluation of one-loop scalar functions,''
  [\href{http://arxiv.org/abs/1007.4716}{arXiv:1007.4716 [hep-ph]}].
%\cite{Ellis:2007qk}
\bibitem{Ellis:2007qk}
  R.~K.~Ellis, G.~Zanderighi,
  ``Scalar one-loop integrals for QCD,''
  JHEP {\bf 0802 } (2008)  002.
  [\href{http://arxiv.org/abs/0712.1851}{arXiv:0712.1851 [hep-ph]}].
%\cite{vanOldenborgh:1990yc}
\bibitem{vanOldenborgh:1990yc}
  G.~J.~van Oldenborgh,
  ``FF: A Package to evaluate one loop Feynman diagrams,''
  Comput.\ Phys.\ Commun.\  {\bf 66 } (1991)  1-15.
%\cite{Hahn:1998yk}
\bibitem{Hahn:1998yk}
  T.~Hahn, M.~Perez-Victoria,
  ``Automatized one loop calculations in four-dimensions and D-dimensions,''
  Comput.\ Phys.\ Commun.\  {\bf 118 } (1999)  153-165.
  [hep-ph/9807565].
%\cite{Heinrich:2010ax}
\bibitem{Heinrich:2010ax}
  G.~Heinrich, G.~Ossola, T.~Reiter, F.~Tramontano,
  ``Tensorial Reconstruction at the Integrand Level,''
  JHEP {\bf 1010 } (2010)  105.
  [\href{http://arxiv.org/abs/1008.2441}{arXiv:1008.2441 [hep-ph]}].
\end{thebibliography}
\end{document}
