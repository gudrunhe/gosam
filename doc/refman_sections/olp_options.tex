\lstset{
  breaklines = true,
  breakatwhitespace = true
}

%
%
%

\subsection{Required Keywords}
Only two BLHA keywords have to present in the order file so that \gosam can process it, otherwise an error will be returned immediately. Those are:
\begin{basedescript}{\desclabelstyle{\pushlabel}}
    \item[\hspace{-1em}]\colorbox{gray!30}{\lstinline[style=in]|CorrectionType|} --- Standard: BLHA1, BLHA2\vspace{0.1cm}\\
        \lstinline[style=in]|supported_values = ["QCD", "EW", "QED"]|
    \item[\hspace{-1em}]\colorbox{gray!30}{\lstinline[style=in]|IRregularisation|} --- Standard: BLHA1, BLHA2\vspace{0.1cm}\\
        \lstinline[style=in]|supported_values = "tHV", "DRED", "CDR"|
\end{basedescript}
Of course, those two keywords are not enough to fully specify a process. Only in conjunction with additional optional keywords (see Section~\ref{sec:optional_BLHA_keywords} below) a process can be defined. If the MC program fails to provide a consistent setup, \gosam will return an error. There is one additional keyword which is required, but for convenience a default value is defined so that it can be omitted:
\begin{basedescript}{\desclabelstyle{\pushlabel}}
    \item[\hspace{-1em}]\colorbox{gray!30}{\lstinline[style=in]|InterfaceVersion|} --- Standard: BLHA2\vspace{0.1cm}\\
        \lstinline[style=in]|supported_values = ["BLHA1", "BLHA2"]|\\
        This is a required keyword introduced with the BLHA2 standard. To be able to use BLHA1 compliant order files the default \texttt{"BLHA1"} is set, when the keyword is absent from the order file.
\end{basedescript}
Note that the BLHA2 standard defines two additional mandatory keywords, \texttt{Model} and \texttt{CouplingPower}. Those have been introduced with BLHA2 and were not present in BLHA1, where alternative keywords could be used. In \gosam one can use those alternative keywords instead, independent of the version standard. Therefore they are treated as optional. Note, however, that the corresponding information has to be provided in one or the other way otherwise the setup of the process will be incomplete and \gosam will exit with an error.

%
%
%

\subsection{Optional Keywords}\label{sec:optional_BLHA_keywords}
There are several ways how to choose a model. If non of the following keywords is present \gosam will assume the built-in Standard Model with diagonal CKM-Matrix (\texttt{smdiag}) as the default.
\begin{basedescript}{\desclabelstyle{\pushlabel}}
    \item[\hspace{-1em}]\colorbox{gray!30}{\lstinline[style=in]|Model|} --- Standard: BLHA2\vspace{0.1cm}\\
        \lstinline[style=in]|supported_values = ["SMdiag", "SMnondiag", "ufo:/<path/to/ufo-model>"]|\\
        The BLHA2 keys will be converted internally: \texttt{SMdiag} $\to$ \texttt{smdiag} and \texttt{SMnondiag} $\to$ \texttt{sm}.
    \item[\hspace{-1em}]\colorbox{gray!30}{\lstinline[style=in]|ModelFile|} --- Standard: BLHA1\vspace{0.1cm}\\
        In the BLHA1 standard it is assumed that the underlying physics model is fixed by the OLP. The possibility to switch between distinct and independent model implemetations is not envisaged. However, model parameters can be passed from the MC to to the OLP using a parameter file in the SLHA format~\cite{Skands:2003cj,Allanach:2008qq}, using the \texttt{ModelFile} keyword. Note: \gosam does not check if the SLHA file is valid.
    \item[\hspace{-1em}]\colorbox{gray!30}{\lstinline[style=in]|UFOModel|} --- Standard: Not part of the standard\vspace{0.1cm}\\
        \lstinline[style=in]|UFOModel <path/to/ufo-model>| is an alternative for \lstinline[style=in]|Model ufo:/<path/to/ufo-model>|
\end{basedescript}
Built-in models different from \texttt{sm} and \texttt{smdiag} can be selected by supplying an additional config file (see Section~\ref{sec:BLHA_running}) or through the special \texttt{\#@} instructions (see Section~\ref{sec:hashat_instructions} below).\\

Another key ingredient to the process definition is the power in the QCD and/or EW coupling. The corresponding keywords in both the BLHA1 and the BLHA2 standard are supported, irrespective of the actual \texttt{InterfaceVersion}.
\begin{basedescript}{\desclabelstyle{\pushlabel}}
    \item[\hspace{-1em}]\colorbox{gray!30}{\lstinline[style=in]|CouplingPower|} --- Standard: BLHA2\vspace{0.1cm}\\
        \lstinline[style=in]|supported_values = ["qcd" <int>, "qed" <int>]|
     \item[\hspace{-1em}]\colorbox{gray!30}{\lstinline[style=in]|AlphasPower|} --- Standard: BLHA1\vspace{0.1cm}\\
        Takes exactly one argument of type \texttt{<int>}.
    \item[\hspace{-1em}]\colorbox{gray!30}{\lstinline[style=in]|AlphaPower|} --- Standard: BLHA1\vspace{0.1cm}\\
        Takes exactly one argument of type \texttt{<int>}.
\end{basedescript}

The BLHA1 standard was focused on one-loop amplitudes only. The keyword \texttt{MatrixElementSquareType} could be used to select different settings with respect to spin and colour averaging. With the BLHA2 standard this feature has been dismissed. At the same time the keyword \texttt{AmplitudeType} has been introduced, which now also allows to call tree-level and spin and colour correlated amplitudes (see Section~\ref{sec:BLHA2_calling}). In the BLHA2 standard the behaviour wrt. spin and colour averaging/summing is that of \texttt{MatrixElementSquareType CHsummed}. \gosam still supports the \texttt{MatrixElementSquareType} keyword, even in a BLHA2 setting.
\begin{basedescript}{\desclabelstyle{\pushlabel}}
    \item[\hspace{-1em}]\colorbox{gray!30}{\lstinline[style=in]|MatrixElementSquareType|} --- Standard: BLHA1\vspace{0.1cm}\\
        \lstinline[style=in]|supported_values = "CHsummed", "CHaveraged", "Csummed", "Caveraged", "Hsummed", "Haveraged", "CHaveragedSymm", "CHsummedSymm", "NoTreeLevel"|
    \item[\hspace{-1em}]\colorbox{gray!30}{\lstinline[style=in]|AmplitudeType|} --- Standard: BLHA2\vspace{0.1cm}\\
        \lstinline[style=in]|supported_values = ["Loop", "Tree", "ccTree", "scTree", "scTree2", "LoopInduced", "ccLoop", "scLoop"]|
\end{basedescript}
If neither \texttt{MatrixElementSquareType} nor \texttt{AmplitudeType} is provided, the behaviour is that of \lstinline[style=in]|AmplitudeType Loop|, based on the provided coupling powers and the correction type.\\

Further optional keywords available with \gosam are:
\begin{basedescript}{\desclabelstyle{\pushlabel}}
    \item[\hspace{-1em}]\colorbox{gray!30}{\lstinline[style=in]|AccuracyTarget|} --- Standard: BLHA2\vspace{0.1cm}\\
        Takes exactly one value of type \texttt{<float>}, must be positive.\\
        Sets the properties \lstinline[style=in]|PSP_chk_th1 = int(-math.log10(<float>))| and  \lstinline[style=in]|PSP_check = True|.
    \item[\hspace{-1em}]\colorbox{gray!30}{\lstinline[style=in]|DebugUnstable|} --- Standard: BLHA2\vspace{0.1cm}\\
        \lstinline[style=in]|supported_values = ["yes", "no", "true", "false"]|\\
        Sets the property \texttt{PSP\_verbosity}.
    \item[\hspace{-1em}]\colorbox{gray!30}{\lstinline[style=in]|EWScheme|} --- Standard: BLHA2\vspace{0.1cm}\\
        \lstinline[style=in]|supported_values = ["alphaGF", "alpha0", "alphaMZ", "OLPDefined"]|\\
        The options \texttt{"alphaRUN"} and \texttt{"alphaMSbar"} are not supported.
    \item[\hspace{-1em}]\colorbox{gray!30}{\lstinline[style=in]|ExcludedParticles|} --- Standard: BLHA2\vspace{0.1cm}\\
        Input must be PDG numbers.
    \item[\hspace{-1em}]\colorbox{gray!30}{\lstinline[style=in]|Extra|} --- Standard: BLHA2\vspace{0.1cm}\\
        Currently the \texttt{Extra} keyword is not used to its full potential. Only settings can be passed which have their own BLHA specific keyword anyway. See Section~\ref{sec:BLHA_precisionchecks} for a usage example.
    \item[\hspace{-1em}]\colorbox{gray!30}{\lstinline[style=in]|OperationMode|} --- Standard: BLHA1, BLHA2\vspace{0.1cm}\\
        \lstinline[style=in]|supported_values = "CouplingsStrippedOff"|
     \item[\hspace{-1em}]\colorbox{gray!30}{\lstinline[style=in]|ParameterCard|} --- Standard: Not part of the standard\vspace{0.1cm}\\
        Can be used to pass parameters defined in an SLHA file to \gosam during process generation. See Section~\ref{sec:BLHA_calling}. This keyword is essentially an alias for BLHA1's \lstinline[style=in]|ModelFile|.
    \item[\hspace{-1em}]\colorbox{gray!30}{\lstinline[style=in]|Parameters|} --- Standard: Not part of the standard\vspace{0.1cm}\\
        Can be used to pass additional parameters besides $\alpha_s$ from the MC to \gosam during process setup. Takes in a list of parameters where the first is expected to be $\alpha_s$. Mainly used to compensate the lack of such an option in the BLHA1 standard. In the BLHA2 standard parameters can be passed at runtime using the \texttt{OLP\_SetParameter} routine, see Section~\ref{sec:BLHA2_calling}.
    \item[\hspace{-1em}]\colorbox{gray!30}{\lstinline[style=in]|Precision|} --- Standard: Not part of the standard\vspace{0.1cm}\\
        Takes exactly one value of type \texttt{<float>}, must be positive. Is and alias for \texttt{AccuracyTarget}.
    \item[\hspace{-1em}]\colorbox{gray!30}{\lstinline[style=in]|PrecisionCheck|} --- Standard: Not part of the standard\vspace{0.1cm}\\
        \lstinline[style=in]|supported_values = ["disabled", "off", "automatic", "polerotation", "rotation", "loopinduced"]|\\
        \lstinline[style=in]|"disabled", "off"| sets the properties \lstinline[style=in]|PSP_chk_method = Automatic| and  \lstinline[style=in]|PSP_check = False|. See also Section~\ref{sec:BLHA_precisionchecks}.
    \item[\hspace{-1em}]\colorbox{gray!30}{\lstinline[style=in]|SubdivideSubprocess|} --- Standard: BLHA1, BLHA2\vspace{0.1cm}\\
        \lstinline[style=in]|supported_values = ["yes", "no", "true", "false"]|\\
        For \texttt{true}/\texttt{yes} \gosam will generate labels for every helicity configuration contributing to a subprocess. For \texttt{false}/\texttt{no} \gosam will generate only one label per subprocess.
\end{basedescript}

%
%
%

\subsection{Common Keywords which are not supported}
\begin{basedescript}{\desclabelstyle{\pushlabel}}
    \item[\hspace{-1em}]\colorbox{gray!30}{\lstinline[style=in]|IRsubtractionMethod|} --- Standard: BLHA1\vspace{0.1cm}\\
        \lstinline[style=in]|suppored_values = "None"|\\
        \gosam does not provide IR-subtracted results, yet.
    \item[\hspace{-1em}]\colorbox{gray!30}{\lstinline[style=in]|LightMassiveParticles|} --- Standard: BLHA2\vspace{0.1cm}\\
        Is related to mass regularisation in EW corrections. Not supported by \gosam.
    \item[\hspace{-1em}]\colorbox{gray!30}{\lstinline[style=in]|MassiveParticleScheme|} --- Standard: BLHA1, BLHA2 \vspace{0.1cm}\\
        This keyword will be ignored as \gosam does not perform EW renormalisation, for which the mass scheme becomes relevant. See also Section~\ref{sec:complexmasses}
    \item[\hspace{-1em}]\colorbox{gray!30}{\lstinline[style=in]|MassiveParticles|} --- Standard: BLHA2\vspace{0.1cm}\\
        This keyword will be ignored as it can interfere in an unexpected way with the settings provided by the model file. The approach taken by \gosam is to initialize parameters including masses as they are provided in the model. During process generation particle masses can be set to zero using the \texttt{zero} process card keyword. See also the corresponding paragraph ``Setting parameters'' and the ``Attention''-blocks in Section~\ref{sec:BLHA2_calling}.
    \item[\hspace{-1em}]\colorbox{gray!30}{\lstinline[style=in]|WidthScheme|} --- Standard: BLHA2\vspace{0.1cm}\\
        \lstinline[style=in]|supported_values = ["ComplexMass", "FixedWidth"]|\\
        Replaces the BLHA1 keyword \texttt{ResonanceTreatment}. This keyword is implemented in \gosam, but currently has no effect, as \gosam does not perform EW renormalisation. See also Section~\ref{sec:complexmasses}
\end{basedescript}
