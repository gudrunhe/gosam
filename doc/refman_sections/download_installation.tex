\section{Prerequisites}

The distribution of \gosamv provides all external and auxiliary programs which are necessary
to successfully run \gosam.
Therefore, the user does not have to install any external programs manually.

The program \gosam is designed to run in any modern Unix-like environment (Linux, Mac).\\
Using \gosam requires a Python installation, a Fortran compiler, a C++ compiler, \textsc{Make} and the \textsc{Meson} build system. It is tested with \texttt{Python} $\geq 3.9$ and \textsc{GNU~Fortran}.
To use a different Fortran compiler and linker, the \texttt{FC} and \texttt{FC\_LD} environment variables can
be set during the installation.


\section{Download}

The \gosamv package can either be cloned
via \texttt{git}
or downloaded as a release tar-ball from the GitHub webpage.

\subsection*{HTTP Download}
At the URL \url{https://github.com/gudrunhe/gosam/releases} the package
\gosamv is available as a tar-ball.
It can be unpacked using the command
\begin{lstlisting}[style=sh]
tar -xzvf GoSam-3.0-<hash>.tar.gz
\end{lstlisting}

\subsection*{Git}
A working copy of the repository can be cloned with the command
\begin{lstlisting}[style=sh]
git clone https://github.com/gudrunhe/gosam
\end{lstlisting}
This will create a folder \texttt{gosam} in your current directory.

\section{Installation}

The installation of \gosam and its dependencies is very simple and fully automated by means of the \texttt{meson}
build system. In the \gosam source directory, running
\begin{lstlisting}[style=sh]
meson setup build --prefix <prefix> [-Doption=value] \\
meson install -C build
\end{lstlisting}
will download and install \gosam as well as all its dependencies. The options \texttt{-Doption=value} can
be used to set the build option \texttt{option} to \texttt{value}. A complete list of build options is available by
running \texttt{meson configure} in the build directory after
setup. Notably, \golemVC is not installed by default, it can be
enabled via \texttt{-Dgolem95=true}. The command \texttt{meson configure -Doption=value} can also
be used to set a build option in an already configured build directory. To avoid collisions with existing
installations of the dependencies, everything will be installed to the subfolder \texttt{GoSam},
e.g. \texttt{<prefix>/lib/GoSam}.

To run \gosam after the installation, the file \texttt{gosam.py}, located at \texttt{<prefix>/bin}, has to be found by the shell, e.g. by appending it to \texttt{\$PATH}.

For the default installation, internet access is required to download the dependencies during the build process.

\attention{
\gosam communicates the location of the dependencies to the generated code via \texttt{meson}'s \texttt{pkg\_config\_path} option. This overrides the \emph{default} search location of \texttt{pkgconf}, and is therefore overriden again by the environment variable \texttt{PKG\_CONFIG\_PATH} if it is set. This results in a compilation error for the process library. To circumvent this, the \texttt{PKG\_CONFIG\_PATH} variable can be removed by using the \texttt{unset} command, or the generated \texttt{<prefix>/gosam\_setup\_env.sh} can be loaded with the \texttt{source} command to add the required paths to \texttt{PKG\_CONFIG\_PATH}.
}

\subsection{Updating an existing installation}

If \gosam was cloned from git, it can be updated by running
\begin{lstlisting}[style=sh]
git pull
\end{lstlisting}
If a previous \texttt{build} directory exists,
\begin{lstlisting}[style=sh]
meson setup --reconfigure
\end{lstlisting}
must be executed in the build directory in order to regenerate the build files. The new version can then be installed by running
\begin{lstlisting}[style=sh]
meson install
\end{lstlisting}
in the build directory.

\section{Description of the components}


The generation of matrix element code using \gosamv can be understood
as a three-step process.
\begin{enumerate}
\item  \textbf{diagram generation}: \python
and \qgraf are used. This phase is initiated by
running \texttt{gosam.py process.in}, where \texttt{process.in} contains the
user input for the process to be calculated.
\item \textbf{code generation}: only \form is run.
This phase is automatically initiated when the process is being built.
\item \textbf{compilation and running}:
a Fortran compiler and the chosen reduction libraries are used.
The compilation is initiated by
\begin{lstlisting}[style=sh]
meson install
\end{lstlisting}
\end{enumerate}
The individual steps are explained in more detail in section~\ref{sec:usage}.

If you use the \gosamv package, you should be aware that
the following programs are used.
(The numbers indicate during which phase of the code generation
the tools will be required).

\begin{longtable}{r p{0.7\textwidth}}
\qgraf (1) & \qgraf~\cite{Nogueira:1991ex}  is required in version 3{.}1 or higher.
\qgraf-3.4.2 will automatically be downloaded and installed during the installation of \gosam.
It can be also downloaded manually from
\url{http://cefema-gt.tecnico.ulisboa.pt/~paulo/d.html}. \\

\python (1) & \gosam requires \texttt{Python} version 3{.}9 or newer. \\

\form (2) & \form~\cite{Vermaseren:2000nd,Kuipers:2012rf,Ueda:2020wqk}
version 4{.}0 or higher is required to profit from all optimisation features.
\form-4.2.1 will automatically be downloaded and installed during the installation of \gosam.
For manual download, \form is available from
\url{http://www.nikhef.nl/~form/}. \\

\ninja/\golemVC (3) &
For one-loop calculations, at least one of these two libraries is required. The libraries are downloaded and installed during the installation of \gosam. Alternatively, \golemVC~\cite{Binoth:2008uq, Cullen:2011kv, Guillet:2013msa} is available from \url{https://github.com/gudrunhe/golem95} and \ninja~\cite{Mastrolia:2012bu,vanDeurzen:2013saa,Peraro:2014cba} is available from \url{https://github.com/peraro/ninja}.\\

\oneloop (3) &
Both \ninja and \golemVC rely on \oneloop~\cite{vanHameren:2010cp} to calculate scalar integrals. \oneloop will automatically be downloaded and installed during the installation of \gosam. It can alternatively be obtained from \url{https://bitbucket.org/hameren/oneloop/}.\\

\texttt{refrep.cls} (3) & The generation of documentation (optional)
is based on the \LaTeX-class \texttt{refrep}, which may not be
present in all \LaTeX{}
distributions. It can be downloaded from \url{http://www.ctan.org/}
as part of the \texttt{refman} package.
This file is only needed if one intends to run \texttt{make doc},
which generates some documentation like drawing the diagrams,
listing the colour structures, etc.
\end{longtable}


\attention{
Please note that some of these programs may have
license policies which are different from the license
applying to \gosamv. The authors of \gosamv do \emph{not}
take any responsibility for any problems related to the
above-mentioned software packages.
}

\section{Running \gosam}

An example of how to use \gosam to set up a process and generate the corresponding libraries is given in the next chapter in Section~\ref{sec:usage}.
