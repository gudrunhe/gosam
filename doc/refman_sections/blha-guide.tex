The interface of \gosam with a Monte Carlo event generator program 
is based on the Binoth-Les Houches Accord (BLHA)
standard interface.
\gosamv supports both BLHA1~\cite{Binoth:2010xt}
and BLHA2~\cite{Alioli:2013nda}.
Certainly, a dedicated interface without using the BLHA is also possible.


\section{Preparation of the order file}
This step should be done by the Monte Carlo (MC) program. 
We give a generic example of an order file for the process \\
$pp\to (Z\to e^+e^-)+$\,jet in both BLHA1 and BLHA2 standards 
in Figs.~\ref{fig:BLHA1} and \ref{fig:BLHA2}.
\begin{figure}[htb!]
\begin{subfigure}[]{0.49\textwidth}
\centering
\fbox{
    \parbox[t][][c]{145\unitlength}{\ttfamily\scriptsize
\# OLP\_order.lh   \\
\# created by MC Sherpowig-1.0\\
\# Process: p p $->$ e+ e- jet\\
Model                    SMdiag\\
CorrectionType           QCD\\
IRregularisation         DRED\\
AlphasPower              2\\
AlphaPower               1\\
MatrixElementSquareType  CHsummed\\
OperationMode            CouplingsStrippedOff\\
SubdivideSubprocess      no\\
\# Subprocesses \\
1 -1 $->$ 11 -11 21\\ 
1 21 $->$ 11 -11 1\\
2 -2 $->$ 11 -11 21\\
...\\
21 -2 $->$ 11 -11 -2\\
\\
\# Process specific GoSam settings\\
\#@ symmetries family,generation}
}
\end{subfigure}
\begin{subfigure}[]{0.49\textwidth}
\centering
\fbox{
    \parbox[t]{165\unitlength}{\ttfamily\scriptsize
\# vim: syntax=olp\\
\#@OLP GOSAM 2.0\\
\#@IgnoreUnknown False\\
\#@IgnoreCase False\\
\#@SyntaxExtensions \\
CorrectionType QCD $|$ OK\\
IRregularisation DRED $|$ OK\\
AlphasPower 2 $|$ OK\\
AlphaPower  1   $|$ OK\\           1\\
MatrixElementSquareType CHsummed $|$ OK\\
OperationMode CouplingsStrippedOff $|$ OK\\
SubdivideSubprocess  no $|$ OK\\
1 -1 $->$ 11 -11 21 $|$ 1 1\\
1 21 $->$ 11 -11 1  $|$ 1 2\\
2 -2 $->$ 11 -11 21 $|$ 1 3\\
...\\
21 -2 $->$ 11 -11 -2 $|$ 1 13\\}
}
\end{subfigure}
\caption{Examples of order and contract files for Z+jet, with BLHA1 standards.}
\label{fig:BLHA1}
\end{figure}  



\begin{figure}[h]
\begin{subfigure}[]{0.49\textwidth}
\centering
\fbox{
    \parbox[t][][b]{148\unitlength}{\ttfamily\scriptsize
\#  OLP\_order.lh \\
\# created by MC Sherpowig-2.0\\
InterfaceVersion         BLHA2\\
CorrectionType           QCD\\
IRregularisation         DRED\\
WidthScheme              ComplexMass\\
EWScheme                 alphaGF\\
AccuracyTarget           0.0001\\
DebugUnstable            True\\

AlphasPower              1\\
AmplitudeType ccTree\\
1 -1 $->$ 11 -11 21 \\
...\\
21 -2 $->$ 11 -11 -2 \\
AmplitudeType scTree\\
1 -1 $->$ 11 -11 21 \\
...\\
21 -2 $->$ 11 -11 -2 \\
AmplitudeType Loop\\
1 -1 $->$ 11 -11 21 \\
...\\
21 -2 $->$ 11 -11 -2 \\
\\
AlphasPower              2\\
AmplitudeType Tree\\
1 1 $->$ 11 -11 1 1 \\ 
...\\
21 21 $->$ 11 -11 2 -2\\}}
\end{subfigure}
\begin{subfigure}[]{0.49\textwidth}
\centering
\fbox{
    \parbox[t][][b]{160\unitlength}{\ttfamily\scriptsize
\# vim: syntax=olp\\
\#@OLP GOSAM 2.0\\
\#@IgnoreUnknown False\\
\#@IgnoreCase False\\
\#@SyntaxExtensions \\
InterfaceVersion BLHA2 $|$ OK\\
CorrectionType QCD $|$ OK\\
IRregularisation DRED $|$ OK\\
WidthScheme              ComplexMass $|$ OK\\
EWScheme                 alphaGF $|$ OK\\
AccuracyTarget           0.0001 $|$ OK\\
DebugUnstable            True $|$ OK\\

AlphasPower 1 $|$ OK\\
AmplitudeType ccTree $|$ OK\\
1 -1 $->$ 11 -11 21 $|$ 1 131\\
...\\
21 2 $->$ 11 -11 2 $|$ 1 70\\
AmplitudeType scTree | OK\\
1 -1 $->$ 11 -11 21 $|$ 1 145\\
...\\
21 2 $->$ 11 -11 2 $|$ 1 71\\
AmplitudeType Loop $|$ OK\\
1 -1 $->$ 11 -11 21 $|$ 1 137\\
...\\
21 2 $->$ 11 -11 2 $|$ 1 63\\
\\
AlphasPower 2 $|$ OK\\
AmplitudeType Tree $|$ OK\\
1 1 $->$ 11 -11 1 1 $|$ 1 42\\
...\\
21 21 $->$ 11 -11 -2 2 $|$ 1 106\\}
}
\end{subfigure}
\caption{Order and contract files for Z+jet with BLHA2 standards.}
\label{fig:BLHA2}
\end{figure}  

\paragraph{Remarks}
\begin{itemize}
\item The order file can have any name and any extension.
      We use  the extension \texttt{.lh}
      for order files (\texttt{.olp} is also common) and \texttt{.olc} for contract files.
\item The options \texttt{WidthScheme, EWScheme} in the BLHA2  example are optional.
\item The option \texttt{SubdivdeSubprocess}  has the following effect
      on the code generation with \gosam{}: if set to 
      \texttt{no}, \gosam{} generates one label per subprocess, if set to
      \texttt{yes} it generates one label per helicity subamplitude
      and therefore \emph{many} labels per subprocess.
      
\item \gosam{} specific settings can be put into commentary lines starting
      with the letter combination `\texttt{\#@}'. This is not part of the
      BLHA standard. The line `\texttt{\#@ symmetries} \dots' restricts the
      helicity subamplitudes being generated to the ones relevant for this
      particular process, using the information that flavour changings only occur within 
      the same quark families resp. lepton generations.

      Additionally, the \texttt{Extra} keyword of BLHA2 for OLP-specific settings is also supported.
\end{itemize}

\section{Running GoSam}
To run \gosam{} within the MC/OLP setup one can use the following command:\\
\begin{lstlisting}[style=sh]
gosam.py --olp -M <MCName> -c <your-path-to>/gosam.conf order.lh
\end{lstlisting}

\paragraph{Remarks}
\begin{itemize}
\item The extension \lstinline[style=sh]|--olp| is mandatory whenever 
   a BLHA order file is processed.
\item The extension \lstinline[style=sh]|-M <MCName>| is optional. By specifying the
   name of the Monte Carlo which is the intended partner program,
   \gosam{} can choose some settings simplifying the communication and linking.
   One can either specify \lstinline[style=sh]|-M MCname| or \lstinline[style=sh]|-M name/version|.
   Alternatively (and also optional),
   one can put this information into the order file:\\
\begin{lstlisting}[style=in]
@olp.mc.name mypreferredmc
@olp.mc.version 1.0.0
\end{lstlisting}
\item The extension \lstinline[style=sh]|-C| is optional and points to a \gosam{} configuration file.
   The latter can be used to define \gosam{} specific settings, such as  
   diagram filters, treatment of the rational parts, etc.
   
   If  this option is left out \gosam{} searches for a configuration in the current working directory. In previous versions of \gosam also the user's home directory and the \gosam installation directory would be searched. Since this can and will lead to unexpected behaviour (when settings are overwritten without the user realizing it), this is not done anymore.
   Possible names for default configuration files are \texttt{gosam.in},
   \texttt{gosam.conf}, \texttt{.gosam}, \texttt{golem.in},
   \texttt{golem.conf}, \texttt{.golem}.
   If such a file is not found, \gosam{} takes the default values for 
   all unspecified settings.
\item  The option \lstinline[style=sh]|--destination=<dir>| allows 
   to place the generated files into the directory \texttt{<dir>}.
   The short form is \texttt{-D <dir>}.
\item One can specify the name of the contract file which should be written
   using the option \lstinline[style=sh]|--output-file=<contractfile>| or simply
   \lstinline[style=sh]|-o <contractfile>|.
\item The option \lstinline[style=sh]|--force| will overwrite an already existing
   contract file without any warning. 
\item A full list of available command line options is available by running 
\begin{lstlisting}[style=sh]
gosam.py --olp --help
\end{lstlisting}
\end{itemize}

\paragraph{The contract file}
From the contract file one can see whether the order file has been processed successfully.
If everything went smoothly it should look like the one in Fig.~\ref{fig:BLHA1}
resp. Fig.~\ref{fig:BLHA2}.
All settings are either acknowledged by the word \texttt{OK} or, in case
of a failure, by the word \texttt{error} followed by an error message.


The subprocesses receive an assignment to one or more
labels per subprocess. In the line
\begin{lstlisting}[style=in]
2 -2 -> 11 -11 21 | 1 3
\end{lstlisting}
the suffix \texttt{| 1 3}
states that this subprocess has been assigned to \texttt{1}
single label which has the value \texttt{3}. 
Had we set \texttt{SubdivideSubprocess} (keyword in BLHA1)
to \texttt{yes} this line might have looked like
\begin{lstlisting}[style=in]
2 -2 -> 11 11 21 | 4 0 1 2 3
\end{lstlisting}
meaning that the subamplitudes
have been assigned to
\texttt{4} labels (which is the first number after the bar) with
the values \texttt{0} to \texttt{3}, each denoting 
an individual helicity subamplitude. These labels will enter the
first argument of the routine \texttt{OLP\_EvalSubProcess}.
In order to retrieve the full amplitude the calling (MC) program should sum
over the contributions from all labels. Alternatively, it is possible to
sample the different channels by Monte Carlo techniques.

\section{Producing the libraries containing the virtual amplitudes}

Building the library is done with the same sequence of commands as in standalone mode,

\begin{lstlisting}[style=sh]
meson setup build --prefix <prefix> 
meson install -C build
\end{lstlisting}

Now one should find the following files%
\footnote{Due to backwards compatibility, they are still named \texttt{libgolem\_olp} instead of \texttt{libgosam\_olp}.}
in a subdirectory \texttt{lib/} or \texttt{lib64/}:
\begin{itemize}
\item \texttt{libgolem\_olp.a} for static linking,
\item \texttt{libgolem\_olp.so} for dynamic linking.
\end{itemize}

The Monte-Carlo program can now be linked to these files or can
use the dynamical library at runtime using the \texttt{dlopen()} and \texttt{dlsym()} system calls%
\footnote{For more details we refer to the corresponding man pages.}.

The required compiler and linking flags can be generated by calling the \texttt{config.sh} script:\\
\begin{lstlisting}[style=sh]
sh ./config.sh -cflags # prints compiler flags 
sh ./config.sh -libs   # prints linking flags
\end{lstlisting}

Inside a makefile, one can use the following lines to extend existing build flags:\\
\texttt{ FCFLAGS+=\$(shell ./config.sh -cflags)} \\
\texttt{ LDFLAGS+=\$(shell ./config.sh -libs)}

The path to \texttt{config.sh} needs of course to be adapted if the makefile is not in the
same directory.

\section{Calling the interface routines}

For the default settings the call of the interface routines 
will be automatic, so the user does not have to care about the details described below.

We should note however that there are slight differences in naming (underscoring) and calling
conventions (call by reference versus call by value) depending on the
extensions in use. For \lstinline[style=sh]|--mc=powhegbox| the extension \texttt{f77}
is automatically included and therefore the underscoring works such that
\texttt{gfortran} used as a Fortran\,77 compiler would not complain.
For all other Monte Carlo programs we follow the C/C++ conventions
(see the file \texttt{olp.h}).

In the following, we will describe BLHA1 and BLHA2 conventions separately, 
even though large parts are identical for the two BLHA versions.

\subsection{BLHA1}

\subsubsection{Initialization}
The generated \gosam{} library is initialized with the call
\begin{lstlisting}[style=fortran]
      call OLP_Start("path/to/contract.olc",ierr)
\end{lstlisting}
The variable \texttt{ierr} should be declared as an integer. If the contract
file is not found, \texttt{ierr} is set to a negative value. A non-negative
value indicates success.

Please note that calling \texttt{OLP\_Start} is mandatory even if the contract
file is not present or not read.

\subsubsection{Importing external model files}
If the contract file contains the option
\texttt{ModelFile}, which should point to a SLHA file,
the matrix element code tries to load the parameters from that file.

\subsubsection{Setting options (optional)}
Parameters can be passed by calling \texttt{OLP\_option}.
\begin{lstlisting}[style=fortran]
      call OLP_Option("name=value",ierr)
\end{lstlisting}

Note that the initialization of derived parameters only works correctly
if the corresponding input parameters are set with \texttt{OLP\_Option}
\emph{before} \texttt{OLP\_Start} is called.

Example:
\begin{lstlisting}[style=fortran]
      call OLP_Option("mZ=91.234",ierr)
      call OLP_Option("mW=80.123",ierr)
!  at this point sin(theta_w) is not up to date.
      call OLP_Start("path/to/contract.olc",ierr)
!  now sin(theta_w) is set consistently
\end{lstlisting}

Some options can be changed at any time; it is instructive to 
look at the file
\texttt{common/model.f90} which contains  the available
parameter names and  their settings.

\subsubsection{Computing the matrix element}

In BLHA1, the routine which returns a value for the matrix element is
\texttt{OLP\_EvalSubProcess}:
\begin{lstlisting}[style=fortran]
      integer ilabel
      double precision moms(5*nlegs)
      double precision mu,params(1)
      double precision res(4)
      !...
      call OLP_EvalSubProcess(
   &        ilabel,moms,mu,params,res)
\end{lstlisting}

The first argument, \texttt{ilabel} is one of the labels from the
contract file. The momenta are passed in the argument \texttt{moms},
which has the format
\begin{displaymath}
\mathtt{(/}
E_1, p^x_1, p^y_1, p^z_1, m_1,
E_2, p^x_2, p^y_2, p^z_2, m_2, \ldots
E_N, p^x_N, p^y_N, p^z_N, m_N
\mathtt{/)}
\end{displaymath}
The momenta are expected to be given in physical (in-out) 
kinematics: $p_1+p_2=p_3+\ldots+p_N$.
The components are in units of GeV.

The argument \texttt{mu} is the renormalisation scale $\mu$ (not $\mu^2$!)
in GeV. The argument \texttt{ params} is an array of which the first argument is
$\alpha_s(\mu)$. Any further array entries are ignored within BLHA1.\footnote{
Passing more than one parameter is implemented by the \texttt{Parameters}
option in the order file, which is  not part of the BLHA1 standard.}

The last argument is an array of length four which is filled by the subroutine, 
containing the result of the evaluation. The entries have as a unit some
power of GeV ($\mathrm{GeV}^{(4-N)}$).
\begin{align}
\label{eq:res}
\mathcal{M}_B^\dagger\mathcal{M}_B&=\mathtt{res(4)}\nonumber\\
2\mathrm{Re}\left(\mathcal{M}_B^\dagger\mathcal{M}_V\right)&=
\frac{(4\pi)^\varepsilon}{\Gamma(1-\varepsilon)}\left(
\frac{\mathtt{res(1)}}{\varepsilon^2}
+\frac{\mathtt{res(2)}}{\varepsilon}
+\mathtt{res(3)}
\right)
\end{align}
This means that the coefficients \texttt{amp(1:3)} contain
an explicit factor of $\alpha_s(\mu)/(4\pi)$.

\subsubsection{Finalize (optional)}
There is also a routine \texttt{OLP\_Finalize} which is only needed
if the client code needs to call \texttt{OLP\_Start} more than once, e.g.
\begin{lstlisting}[style=fortran]
      do i=1,max_i
         write(line,'(A3,F6.3)') "mZ=", mZ(i)
         call OLP_Option(line,ierr)
         ! Need olp_start to update dependent parameters
         call OLP_Start(name,ierr)
         ! ...
         call OLP_Finalize()
      enddo
\end{lstlisting}

\subsection{BLHA2}

\attention{
Please note that \gosam will ignore the BLHA2 optional keyword \texttt{MassiveParticles} as this can result in unintended behaviour depending on the used physics model. If the user wishes to set any particle masses to zero, this should be done by means of the ``\texttt{zero}'' option in the process card.
}

\subsubsection{Initialization}
The keyword \texttt{ InterfaceVersion}, which can take the values
\texttt{ BLHA1} or \texttt{ BLHA2}, should be placed on top of the order file. 
This way, if the OLP does not support one or the other, it can issue an error message and stop 
without proceeding further.
To start the run-time phase, the function
\begin{lstlisting}[style=C]
      OLP_Start(char* fname, int* ierr)
\end{lstlisting}
is the same  as in \texttt{BLHA1}.
A new function
\begin{lstlisting}[style=C]
      OLP_Info(char olp_name[15], char olp_version[15], char message[255])
\end{lstlisting}
has been introduced,
which serves to keep track of the type and version of the OLP which has been used,
and to encourage proper citation. 
The arguments are the name of the OLP, the version, and a string which  
contains information about
the relevant publications, for example the bibtex identifier.

\subsubsection{Importing external model files}

The BLHA2 offers two alternative ways of model definition, denoted by 
``keyword model" respectively ``UFO model" in the following.

Model definitions offer the possibility to define some global settings 
in the order file, which are intrinsic to the model (e.g. SM, MSSM), which 
is used.
This is done using the required keyword \texttt{ Model}.
For example, \texttt{ Model: smdiag} sets the CKM matrix to unity globally.

In the ``keyword model" setup, 
the parameters that need to be set within a certain model 
are passed via PDG codes~\cite{Beringer:1900zz} and keywords 
with naming
conventions as specified in Fig.~\ref{tab:keywords:static} for the Standard
Model. The numbers in parenthesis after \texttt{ mass} and \texttt{ width}  denote
the particle's PDG code.

\begin{figure}[htb]
\centering
\renewcommand{\arraystretch}{1.1}
\begin{tabular}{|l|l|}
\hline
keyword & parameter\\
\hline
\texttt{ mass(5)} & b quark mass \\
\texttt{ mass(6)} & top quark mass \\
\texttt{ width(6)} & top quark width\\
\texttt{ sw2}& $\sin^2\theta_w$\\
\texttt{ vev}& SM vacuum expectation value\\
\texttt{ Gf} & $G_{\mathrm{Fermi}}$\\
\texttt{ VV12}& $V_{ud}$\\
$\vdots$ & \\
\hline
\end{tabular}
\renewcommand{\arraystretch}{1.25}
\caption{List of keywords to define parameters to be passed by the function \texttt{
OLP\_SetParameter}.}
\label{tab:keywords:static}
\end{figure}

In the ``UFO model" setup, the parameters are defined in UFO (Universal Feynrules
Output)~\cite{Degrande:2011ua} format, which is particularly useful for
calculations beyond the Standard Model.
The import of the UFO model file should be specified in the \texttt{ order file} 
by 
\begin{lstlisting}[style=in]
      Model ufo:/path_to_ufo_model_directory/
\end{lstlisting}

The UFO  format also provides human readable name attributes for the 
model parameters, as well as the 
SLHA identifiers~\cite{Skands:2003cj} which are also supported by GoSam.
The UFO model setup entails the use of a SLHA parameter card to initialize the runtime phase.
This requires an additional keyword \texttt{ ParameterCard}, followed by the path to the SLHA parameter card,
to be placed into the order file when using the UFO model setup. 
The parameters which are set by reading in the SLHA parameter card do not need to be set 
again by \texttt{ OLP\_SetParameter}. However, \texttt{ OLP\_SetParameter} needs to be used at 
runtime for the dynamic parameters. 
In this case the SLHA block name should appear as a prefix prepended to the parameter name, 
in the form  
\texttt{ <BlockName>\&\&<ParamName>}.
To avoid confusion, this requires that the characters `\texttt{ \&\&}' should never appear in 
any block or parameter name.

\subsubsection{Setting parameters}
Parameters are now passed by the subroutine 
\begin{lstlisting}[style=C]
      OLP_SetParameter(char* para, double* re, double* im, int* ierr)
\end{lstlisting}

where the first argument is a (pointer to a) string serving as a keyword 
for the parameter to be set, followed by two double precision numbers
so that complex parameters can also be passed (in case of real parameters, 
the second double is zero). The integer in the fourth argument 
is set by the OLP to tell the MC whether the setting of the parameter 
was successful.\\
\texttt{ ierr=1} means the parameter has been set successfully, \\
\texttt{ ierr=0} means failure: issue an error message, \\
\texttt{ ierr=2} means that the parameter is unknown 
or the setting is ignored (for example because it is irrelevant 
for the considered case), but the MC program should proceed.


The function \texttt{ OLP\_SetParameter} can be called at runtime, 
for every phase space point, 
if used to define a dynamic parameter. 

\subsubsection{Computing the matrix element}

In BLHA2, the routine which returns a value for the matrix element is
\begin{lstlisting}[style=C]
      OLP_EvalSubProcess2(
            int* i, double* pp, double* mu, double* rval, double* acc
            )
\end{lstlisting}


The arguments are:
\begin{itemize}
\item \texttt{i}: pointer to a (one element) array with the label of the subprocess as given in the contract file
\item \texttt{pp}: pointer to an array of momenta, conventions $(E_j,k_j^x,k_j^y,k_j^z,M_j)$
\item \texttt{mu}: pointer to the renormalisation scale 
\item \texttt{rval}: pointer to an array of return values
\item \texttt{acc}: pointer to a one element array with the outcome of the 
OLP internal accuracy check 
\end{itemize}


The last argument is an array of length four which is filled by the subroutine, 
containing the result of the evaluation, as specified in eq.~(\ref{eq:res}).
The default settings for the prefactor can be changed using the option \texttt{ nlo\_prefactor}, 
see section \ref{sec:nlo_prefactors}.

For more details concerning the BLHA2 conventions we refer to \cite{Alioli:2013nda}.

\subsubsection{Loop-induced processes}

Loop-induced processes are supported by the setting \\
\texttt{AmplitudeType} \texttt{LoopInduced}.

In \gosam, they are not handled like Born processes, but like virtual corrections to non-existing born processes
and therefore returned in the virtual field $A_0$ (\texttt{PoleCoeff0}) of \texttt{rval} when calling \texttt{OLP\_EvalSubprocess} and \texttt{OLP\_EvalSubprocess2}. The returned value corresponds to
the  squared amplitude.

\attention{
Please note that in the order file, 
\texttt{CouplingPower} or \texttt{AlphasPower} and \texttt{AlphaPower}  usually refer to 
the coupling powers if the corresponding Born amplitude, 
and the type of the correction is specified as \texttt{CorrectionType}. 
As in the case of loop-induced processes the Born amplitude does not exist, 
the correct counting of the coupling powers needs to be assured by setting \texttt{CouplingPower} 
(or \texttt{AlphasPower} and \texttt{AlphaPower}) 
equal to the order of a corresponding fictitious Born process, i.e. reduce the 
coupling powers of the loop induced process correspondingly.
}
%(Note that this is ambiguous in the case of mixed QCD-QED couplings, 
%but it has no effect.)

\subsubsection{BSM-SM-interference processes}
\gosam{} can calculate interference effects between e.g. BSM-Born and
SM processes
(where the BSM Born for example comes from additional interactions in an effective field theory).
These processes are handled as corrections being next-to-leading order
in the SM coupling. \\
In the case where the SM process is loop-induced, 
the standard pole check would fail
due to the non-matching Born. In this case, the user should set \texttt{PSP\_chk\_method} to either \texttt{LoopInduced} or \texttt{Rotation} in the \gosam{} input card,
or use the \gosam-specific extension \texttt{AmplitudeType} \texttt{LoopInterference} in the BLHA2 order file,
which automatically sets \texttt{PSP\_chk\_method} to \texttt{LoopInduced}.

Please note that such a setup may require
to define some filters in the input card to select the correct diagrams.

\subsubsection{Precision checks}

The \gosam{} input card variable \texttt{PSP\_chk\_method}, which
controls the behaviour how \gosam{} checks the result for each phase-space point,
can also be set by \texttt{Extra PrecisionCheck}.
Possible values are:

\begin{itemize}
	\item \texttt{Extra PrecisionCheck Automatic} \textit{(default)} -- chooses automatically between the options \texttt{PoleRotation} and \texttt{LoopInduced}
	\item \texttt{Extra PrecisionCheck PoleRotation} -- checks the precision of the pole first and rotates if necessary
	\item \texttt{Extra PrecisionCheck Rotation} -- estimates the precision of each phase space point by rotating and re-evaluating (slow)
	\item \texttt{Extra PrecisionCheck LoopInduced} -- checks that the poles are zero (i.e. very small compared to the finite part) and rotates if necessary
	\item \texttt{Extra PrecisionCheck Disabled}  -- this sets \texttt{PSP\_check=False} which switches off all phase space point precision checks.
\end{itemize}


\subsubsection{Subprocess-specific settings in the \gosam{} input card}
Settings in the \gosam{} input card can be subprocess-specific.
This is helpful if various subprocesses, each having different settings, should be calculated at once.

For this purpose, the subprocesses are enumerated as in the BLHA order file, starting at zero (to match to the
correspondig \texttt{p*} subdirectories created by \gosam{}).

\attention{
This counting does not necessarily match  the labels
returned in the BLHA contract file.
}

The syntax is \lstinline[style=sh]|option[list-of-subprocesses] = value|.
For example, to disable the precision check 
for the second and third processes in the order file,
one can set \lstinline[style=sh]|PSP_check[1,2] = True| in the input card.

Ranges and exclusion of ranges with \texttt{!} (or \texttt{\^}) are supported.
Examples for valid lists:
\begin{align*}
    \text{\texttt{0-2}         }     &= \{0,1,2\} \\
    \text{\texttt{-6,!3-4}         }  &   = \{0,1,2,5,6\} \\
    \text{\texttt{ 1-4,!3,9}       }   &   = \{1,2,4,9\}
\end{align*}

Subprocess-specific settings need to be unambiguous, and they overwrite 
the corresponding globally set values.

\subsection{Production of colour-/spin correlated trees}

\gosam{} can also generate  tree level amplitudes in a spin- and colour-correlated form, which can be obtained via the BLHA interface by requesting \texttt{AmplitudeType scTree} and \texttt{AmplitudeType ccTree}, respectively. Colour-correlated matrix elements are defined as
\begin{equation}
 C_{ij}=\bra{{\cal M}}\textbf{T}_{i}\textbf{T}_j \ket{{\cal M}}\;,
\end{equation}
spin-correlated matrix elements can be defined as
\begin{equation}
 S_{ij}=\bra{{\cal M},-}\textbf{T}_{i}\textbf{T}_j \ket{{\cal M},+}\;.
\end{equation}
The spin-correlated matrix element above (as well as the colour correlated matrix element) contains implicitly
the sum over all other helicities, only the helicities with the indices $i$ and $j$ are fixed, i.e.\footnote{Note that the spin-correlated matrix elements depend on the conventions chosen for the implementation of the polarization vectors, which can be seen by explicitly pulling out the polarization vector from the amplitude, $|{\cal M}_{i,\lambda}\rangle=\epsilon_\lambda^\mu(p_i)|{\cal M}_{i,\mu}\rangle$. Two conventions can differ by a phase and a shift proportional to the bosons momentum, $\hat{\epsilon}_\lambda^{\mu}(p_i)=e^{i\phi_\lambda}\epsilon_\lambda^\mu(p_i)+\alpha_\lambda p_i^\mu$, from which follows $\langle{\cal M}_{i,\lambda}|{\cal M}_{i,\lambda'}\rangle=e^{i(\phi_{\lambda}-\phi_{\lambda'})}\langle\hat{{\cal M}}_{i,\lambda}|\hat{{\cal M}}_{i,\lambda'}\rangle$.}
 \begin{eqnarray}
&&\langle {\cal M}_{i,-} |{\mathbf T}_i\cdot {\mathbf T}_j |{\cal M}_{i,+}\rangle =\\
&&\sum_{\lambda_1,...,\lambda_{i-1},\lambda_{i+1},...,\lambda_n}
\langle {\cal M}_{\lambda_1,...,\lambda_{i-1},-,\lambda_{i+1},...,\lambda_n} |
{\mathbf T}_i\cdot {\mathbf T}_j | 
{\cal M}_{\lambda_1,...,\lambda_{i-1},+,\lambda_{i+1},...,\lambda_n}\rangle \;. \nonumber
\end{eqnarray}
These matrix elements are particularly useful in combination with Monte Carlo programs 
which use these trees to build the dipole subtraction terms for the infrared divergent 
real radiation part. With these modified tree level matrix elements \gosam{} is able to generate
all necessary building blocks for a complete NLO calculation.\\
Such a setup has been used successfully in combination with the framework of 
\textsc{Herwig}~\cite{LesHouches2013,Bellm:2013lba,Platzer:2011bc} and the Monte Carlo Program Whizard~\cite{Kilian:2007gr,Moretti:2001zz,Stienemeier:2021cse,Braun:2025hvr}.

\textsc{Whizard} requires the spin-correlations to be passed in the form of the (real  part of the) spin-correlation tensor $B_j^{\mu\nu}$ for particle $j$, instead. It is defined by
\begin{equation}
   B_j^{\mu\nu} = \sum_{\lambda,\lambda'}\epsilon_h^\mu(p_j)\epsilon_{\lambda'}^{*\nu}(p_j)\langle {\cal M}_{j,\lambda}|{\cal M}_{j,\lambda'}\rangle\,,
\end{equation}
where again a sum over the helicities of the other particles is implicitly understood. Invoking the BLHA interface with the non-standard \texttt{AmplitudeType scTree2} newly implemented in \gosam{}-3.0 will return the relevant entries of $B_j^{\mu\nu}$ in a way which can be directly used in \textsc{Whizard}. Additional phase-factors in $\langle {\cal M}_{j,\lambda}|{\cal M}_{j,\lambda'}\rangle$ related to the fact that \textsc{Whizard} expects the polarization vectors to be defined in the conventions of~\cite{Murayama:1992gi} are taken into account automatically. We remark that this convention for the spin-correlated tree matrix elements conincides with the one used by OpenLoops~\cite{Buccioni:2019sur} in its subroutine \texttt{evaluate\_scpowheg}.
