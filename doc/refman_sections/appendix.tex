\chapter{Conventions}
\label{sec:conventions}

\section{Conventions of \golemVC}
The integral library \golemVC{} computes integrals of the form
\begin{equation}
\mu^{2\varepsilon}\int\frac{\mathrm{d^D k}}{i\pi^{D/2}}\frac{k^{\mu_1}\cdots k^{\mu_r}}{((k+r_1)^2-m_1^2)\cdots(k+r_N)^2-m_N^2)}
=r_\Gamma\cdot\left[\frac{c_{-2}}{\varepsilon^2}+\frac{c_{-1}}{\varepsilon}+c_0
+{\mathcal{O}}(\varepsilon)\right]
\end{equation}
where $D=(4-2\varepsilon)$ and
\begin{equation}
r_\Gamma=\frac{\Gamma(1+\varepsilon)\Gamma^2(1-\varepsilon)}{%
   \Gamma(1-2\varepsilon)}.
\end{equation}
The commonly used integration measure for the internal momentum $k$ is
\begin{equation}
\frac{\mu^{2\varepsilon}\diff[D]k}{(2\pi)^D}
=\mu^{2\varepsilon}\frac{i}{2^D\pi^{D/2}}\cdot\frac{{\mathrm d}^Dk}{i\pi^{D/2}}
=\frac{(4\pi)^\varepsilon \cdot i}{(4\pi)^2}\cdot%
 \frac{\mu^{2\varepsilon}{\mathrm d}^Dk}{i\pi^{D/2}}.
\end{equation}

\section{Conventions of \gosamv}
The factor from above which does not go into the integral definition of
\golemVC{} can be written as
\begin{equation}
\frac{(4\pi)^\varepsilon \cdot i}{(4\pi)^2}=
\frac{(4\pi)^\varepsilon}{(2\pi)(4\pi)}\frac{i}{2}
\end{equation}
The factor of $i/2$ is included in the amplitude definition of \gosamv{}.
The factors $(2\pi)$ and $(4\pi)$ are later used to build up a factor of
$\alpha_x/2\pi$, where $\alpha_x$ is either $\alpha$ or $\alpha_s$.

In the following we assume that the coupling constants\footnote{
$e$ and $g_s$ in the Standard Model} have been set to one in the
setup of \gosamv{}. This ensures that the one-loop matrix
element, interfered with the Born amplitude, for QCD corrections is calculated in the $\overline{\mathrm{MS}}$ scheme as
\begin{align}
&2{\mathrm{Re}}({\cal M}_{\mathrm{loop}}{\cal
                M}_{\mathrm{Born}}^\dagger)=:\left\vert\mathcal{M}\right\vert^2_{\text{1-loop}}\nonumber\\
  &=
\frac{\alpha_s}{2\pi}\frac{(4\pi)^\varepsilon}{\Gamma(1-\varepsilon)}
\cdot\left[\frac{c_{-2}}{\varepsilon^2}+\frac{c_{-1}}{\varepsilon}+c_0
+{\mathcal{O}}(\varepsilon)\right](g_1^{n_1}\cdots g_q^{n_q}) \label{eq:Conventions:nlo_matrix_element}
\end{align}
The factor $(g_1^{n_1}\cdots g_q^{n_q})$ are the coupling constants
to the powers as they appear in the squared tree-level matrix element. \gosamv{} will
return the coefficients $c_{-2}$, $c_{-1}$ and $c_0$.

The conversion between different conventions for the $\Gamma$-functions
is straightforward:
\begin{equation}
\frac{1}{\Gamma(1-\varepsilon)}=r_\Gamma+{\mathcal O}(\varepsilon^3)=
\left(1-\frac{\pi^2}{6}\varepsilon^2\right)\Gamma(1+\varepsilon)
   +{\mathcal O}(\varepsilon^3)
\end{equation}

The relevant terms in the expansion of $r_\Gamma$ are
\begin{equation}
r_\Gamma=e^{-\gamma_E\varepsilon}
\left(1-\frac{\pi^2}{12}\varepsilon^2\right)+\mathcal{O}(\varepsilon^3)
\end{equation}

If one prefers to pull out a factor of
$e^{-\gamma_E\varepsilon}(4\pi)^{\varepsilon}$ the appropriate
definition of the matrix element up to terms of $\mathcal{O}(\epsilon)$ is
\begin{equation}
\frac{\left\vert\mathcal{M}\right\vert^2_{\text{1-loop}}}%
{e^{-\gamma_E\varepsilon}(4\pi)^\epsilon}=
\frac{\alpha_s}{2\pi}
\cdot\left[\frac{c_{-2}}{\varepsilon^2}+\frac{c_{-1}}{\varepsilon}
+\left(c_0-\frac{\pi^2}{12}\,c_{-2}\right)
\right](g_1^{n_1}\cdots g_q^{n_q})
%+{\mathcal{O}}(\varepsilon)
\end{equation}

\section{The \texttt{nlo\_prefactor} option}
\label{sec:nlo_prefactors}
In the one-loop amplitude defined in equation \eqref{eq:Conventions:nlo_matrix_element}, a factor of $\alpha_x/2\pi$ is \emph{not} included in the coefficients $c_{-2}$, $c_{-1}$ and $c_0$. The choice of what prefactor is pulled out of the one-loop amplitude is controlled by the \texttt{nlo\_prefactor} option, which has the three choices
\begin{eqnarray}
   A_0 = \frac{\alpha_x}{2\pi}, \qquad A_1 = \frac{1}{8 \pi^2} \quad \text{and} \quad A_2 = 1.
\end{eqnarray}
For \texttt{nlo\_prefactor = i}, the squared amplitude at one-loop level is then defined as
\begin{equation}
   \left|\mathcal{M}\right|^2_{\text{1-loop}}=
A_i\ \frac{(4\pi)^\varepsilon}{\Gamma(1-\varepsilon)}
\cdot\left[\frac{c_{-2}}{\varepsilon^2}+\frac{c_{-1}}{\varepsilon}+c_0
+{\mathcal{O}}(\varepsilon)\right](g_1^{n_1}\cdots g_q^{n_q}).
\end{equation}
The default choice of this option depends on the mode \gosam is running in:
\begin{itemize}
   \item In standalone mode, the default choice is \texttt{nlo\_prefactor = 0}
   \item In OLP mode, the default choice is \texttt{nlo\_prefactor = 2}
\end{itemize}
In the case of OLP mode, the default choice of the prefactor is mandated by the BLHA convention. The prefactor can still be changed by explicitly setting the \texttt{nlo\_prefactor} option, but please keep in mind that doing this \emph{violates the BLHA standard}.

In the case of a loop-induced process, the prefactor is pulled out of each appearing one-loop amplitude piece, resulting in an overall prefactor of $A_i^2$.

%%%%%%%%%%%%%%%%%%%%%%%%%%%%%%%%%%%%%%%%%%%%%%%%%%%%%%%%%%%%%%%%%%%%%%%%
\chapter{Explicit reduction of the $R_2$ terms}
\label{app:r2}
The $R_2$ term \cite{Ossola:2008xq} consists of all terms of the numerator
containing an explicit $\varepsilon$ or $\mu^2$ coming from the Lorentz
algebra in $D=4-2\varepsilon$ dimensions. 
For an explicit reduction of these terms, we give a list of all relevant integrals of the form
\begin{align}
\int\frac{\diff[D] k}{i\pi^{n/2}}
\frac{N(\hat{k})\cdot\mu^{2\alpha}\cdot\varepsilon^\beta}{D_0\cdots D_N}
\end{align}
where either $\alpha$ or $\beta$ is a positive integer number, 
$\hat{k}$ denotes 4-dimensional loop momenta, $k^2=\hat{k}^2-\mu^2$, 
and the denominators are $D_i=(k+r_i)^2-m_i^2+i\delta$.
Note that integrals where both $\alpha$ and $\beta$ are
non-zero will not contribute to the final result, as they will be of order $\varepsilon$.
An integral of rank $r$ 
can be written as~\cite{Binoth:2005ff,Reiter:2009kb}:
\begin{multline}
I_N^{D,\alpha,\beta;\mu_1\ldots\mu_r}=
(-1)^{r}\frac{\Gamma(\alpha-\varepsilon)}{\Gamma(-\varepsilon)}
\varepsilon^\beta
\sum_{l=0}^{\lfloor r/2\rfloor}\left(-\frac12\right)^l
\sum_{j_1,\ldots,j_{r-2l}=1}^N
\times\\
\left[\hat{g}^{\bullet\bullet}\ldots
\hat{g}^{\bullet\bullet}r_{j_1}^\bullet
\cdots r_{j_{r-2l}}^\bullet\right]^{\mu_1\ldots\mu_r}
I_N^{D+2\alpha+2l}(j_1,\ldots,j_{r-2l}).
\end{multline}
Here, the integral $I_N^d(j_1,j_2,\ldots)$ denotes a Feynman parameter
integral with the parameters $z_{j_1}, z_{j_2}, \ldots$ in the numerator,
\begin{equation}
I_N^d(j_1,\ldots, j_p)=
(-1)^N\Gamma\left(N-\frac{d}2\right)%
\int\!\!\diff[D]_\Box\!z\,\delta_z
\frac{\prod_{\nu=1}^p z_{j_\nu}}{%
\left[-\frac12 z^{\mathsf{T}}Sz-i\delta\right]^{N-d/2}},
\end{equation}
where $\diff[D]_\Box\!z=
\prod_{j=1}^N\mathrm{d}z_j\Theta(z_j)\Theta(1-z_j)$
and $\delta_z=\delta(1-\sum_i z_i)$.
The square brackets $[\ldots]^{\mu_1\ldots\mu_p}$ expand to the sum of
all possible assignments of indices to the $\hat{g}^{\bullet\bullet}$-tensors
and momenta $r_j^\bullet$. 
The kinematic matrix $S$ is given by $S_{ij}=(r_i-r_j)^2-m_i^2-m_j^2$.

We only need to consider integrals containing an UV pole, because only the latter lead to
a rational term when multiplied with $\varepsilon$ stemming either from
$\varepsilon^\beta$ or from the integral prefactor
\begin{equation}
\frac{\Gamma(\alpha-\varepsilon)}{\Gamma(-\varepsilon)}=
(\alpha-1)!\left[-\varepsilon +{\mathcal O}(\varepsilon^2)\right],
\quad\text{for}\,\alpha>0.
\end{equation}
The UV divergence comes from the Gamma function
\begin{equation}
\Gamma\left(N-\frac{D+2\alpha+2l}2\right)=
\Gamma(\varepsilon-(2+\alpha+l-N))\equiv\Gamma(\varepsilon-\eta)
\end{equation}
in the Feynman parameter integral~$I_N^{D+2\alpha+2l}$.
Hence, we examine further the expression
\begin{equation}
\varepsilon\cdot I_N^{D+2l+2\alpha}(l_1,\ldots, l_{r-2l})=
\left\{\begin{array}{lr}
{\mathcal O}(\varepsilon),&\eta<0\\
(-1)^N\frac1{2^\eta\eta!}\int\diff[D]_\Box\!z\delta_z
\left[z^{\mathsf{T}}Sz\right]^\eta
\prod_{i=1}^{r-2l}z_{l_i},&\eta\geq0
\end{array}\right.
\end{equation}

The remaining integration can be understood as a special case of the
Feynman parameter identity
\begin{equation}
\frac{1}{\prod_{j=1}^N A_j^{\nu_j}}=\frac{\Gamma(\nu)}{
\prod_{j=1}^N \Gamma(\nu_j)}\int\!\diff[D]_\Box\!z\,\delta_z
\frac{\prod_{j=1}^N z_j^{\nu_j-1}}{\left(
\sum_{j=1}^N z_j A_j\right)^\nu}\; , \; \nu=\sum_j \nu_j
\end{equation}
for $A_j=1$, in which case one finds
\begin{equation}
\int\!\diff[D]_\Box\!z\,\delta_z
\prod_{j=1}^N z_j^{\nu_j-1}=\frac{\prod_{j=1}^N \Gamma(\nu_j)}%
{\Gamma(\alpha)}
\end{equation}

In the \gosam{} process card, the default is \texttt{r2=explicit}, which means that
the rational part $R_2$ is calculated algebraically using the formulae below,
while the integrand reduction can be done in 4 dimensions.
Choosing \texttt{r2=implicit} means that $R_2$ will be calculated together with the 4-dimensional
part during the reduction.

\section*{Integrals with \boldmath$r\leq D$}
All non-zero cases (in the limit $\epsilon\to0$) for
integrals where the rank does not exceed the number of propagators are listed
 below~\cite{Binoth:2006hk,Reiter:2009kb}. We encounter box integrals at most; pentagons and above are all zero.

\subsection*{Tadpoles}
\begin{subequations}
\begin{align}
  I_1^{D,0,1} &= -\frac{1}{2}S_{11}\,,\\[10pt]
%
  I_1^{D,0,1;\mu_1} &= \frac{1}{2}S_{11}r_1^{\mu_1}\,.
\end{align}
\end{subequations}
Note: $S_{11}=-2m_1^2$

\subsection*{Bubbles}
\begin{subequations}
\begin{align}
    I_2^{D,0,1} =& 1\,,\\[10pt]
%
    I_2^{D,0,1;\mu_1} =& -\frac{1}{2}\qty(r_1^{\mu_1}+r_2^{\mu_1})\,,\\[10pt]
%
    I_2^{D,0,1;\mu_1\mu_2} =& \frac{1}{6}\qty(2r_1^{\mu_1}r_1^{\mu_2}+r_1^{\mu_1}r_2^{\mu_2}+r_2^{\mu_1}r_1^{\mu_2}+2r_2^{\mu_1}r_2^{\mu_2})\notag\\
      &-\frac{1}{12}\hat{g}^{\mu_1\mu_2}\qty(S_{11}+S_{12}+S_{22})\,,\\[10pt]
%
    I_2^{D,1,0} =& -\frac{1}{6}\qty(S_{11}+S_{12}+S_{22})\,.
\end{align}
\end{subequations}

\subsection*{Triangles}
\begin{subequations}
\begin{align}
  I_3^{D,0,1;\mu_1\mu_2} =& \frac{1}{4}\hat{g}^{\mu_1\mu_2}\,,\\[10pt]
%
  I_3^{D,0,1;\mu_1\mu_2\mu_3} =& -\frac{1}{12}\sum_{j_1=1}^3\qty[\hat{g}^{\bullet\bullet}r_{j_1}^\bullet]^{\mu_1\mu_2\mu_3}\,,\\[10pt]
%
  I_3^{D,1,0} =& \frac{1}{2}\,,\\[10pt]
%
  I_3^{D,1,0;\mu_1} =& -\frac{1}{6}\qty(r_1^{\mu_1}+r_2^{\mu_1}+r_3^{\mu_1})\,.
\end{align}
\end{subequations}

\subsection*{Boxes}
\begin{subequations}
\begin{align}
  I_4^{D,0,1;\mu_1\mu_2\mu_3\mu_4} =& \frac{1}{24}\qty[\hat{g}^{\bullet\bullet}\hat{g}^{\bullet\bullet}]^{\mu_1\mu_2\mu_3\mu_4}\,,\\[10pt]
%
  I_4^{D,1,0;\mu_1\mu_2} =& \frac{1}{12}\hat{g}^{\mu_1\mu_2}\,,\\[10pt]
%
  I_4^{D,2,0} =& -\frac{1}{6}\,.
\end{align}
\end{subequations}

\section*{Integrals with \boldmath$r>D$}
In addition, we list integrals which contribute to the rational part in cases where the rank exceeds the number of
propagators, for example in the presence of effective gluon-Higgs couplings, or in models involving
gravitons.
More details about higher rank integrals can be found in
Refs.~\cite{Guillet:2013msa,Mastrolia:2012bu,vanDeurzen:2013pja}. All non-zero integrals (in the limit $\epsilon\to0$) with $r=D+1$ are implemented in \gosam and shown below. Integrals with $r>D+1$ are currently not available. We encounter pentagon integrals at most; hexagons and above are all zero.
% \bea
% &&I_5^{D,3}(S)=
% \int\!\!\frac{\diff[D]k}{i\pi^{D/2}}\frac{\left(\tilde{k}^2\right)^3
% }{%nl
% \prod_{j=1}^5(q_j^2-m_j^2+i\delta)}=-\frac{1}{12}\;,\\
% %
% &&I_5^{D,2;\mu_1 \mu_2}(a_1,a_2; S)=
% \int\!\!\frac{\diff[D]k}{i\pi^{D/2}}\frac{\left(\tilde{k}^2\right)^2\;
% \hat{q}_{a_1}^{\mu_1}  \hat{q}_{a_2}^{\mu_2}}{%nl
% \prod_{j=1}^5(q_j^2-m_j^2+i\delta)}=-\frac{1}{48}\,\hat{g}^{\mu_1\mu_2}\;,\\
% %
% &&I_5^{D,1;\mu_1\cdots \mu_4}(a_1,\ldots,a_4; S)=
% \int\!\!\frac{\diff[D]k}{i\pi^{D/2}}\frac{\tilde{k}^2\;
% \hat{q}_{a_1}^{\mu_1} \ldots \hat{q}_{a_4}^{\mu_4}}{%nl
% \prod_{j=1}^5(q_j^2-m_j^2+i\delta)}\nn\\
% &&\qquad =-\frac{1}{96}\,
% \left[\hat{g}^{\mu_1\mu_2}\hat{g}^{\mu_3\mu_4}+ \hat{g}^{\mu_1\mu_3}\hat{g}^{\mu_2\mu_4}
% +\hat{g}^{\mu_1\mu_4}\hat{g}^{\mu_2\mu_3}\right]\;,
% \eea
% \be
% \eps I_4^{D+6}(S)=\frac{1}{240}\left(\sum_{i,j=1}^4 ((r_i-r_j)^2-m_i^2-m_j^2)-2\sum_{i=1}^4 m_i^2\right)\;.
% %+{\cal O}(\eps)\;.
% \ee

\subsection*{Tadpoles}
\begin{subequations}
\begin{align}
  I_1^{D,0,1;\mu_1\mu_2} &= -\frac{1}{2}S_{11}r_1^{\mu_1}r_1^{\mu_2}+\frac{1}{16}\hat{g}^{\mu_1\mu_2}\qty(S_{11})^2\,,\\[10pt]
%
  I_1^{D,1,0} &= \frac{1}{8}\qty(S_{11})^2\,.
\end{align}
\end{subequations}
Note: $S_{11}=-2m_1^2$

\subsection*{Bubbles}
\begin{subequations}
\begin{align}
    I_2^{D,0,1;\mu_1\mu_2\mu_3} =& -\frac{1}{12}\sum_{j_1,j_2,j_3=1}^2\qty[r_{j_1}^\bullet r_{j_2}^\bullet r_{j_3}^\bullet]^{\mu_1\mu_2\mu_3} -\frac{1}{6}\sum_{j=1}^2\qty[r_j^\bullet r_j^\bullet r_j^\bullet]^{\mu_1\mu_2\mu_3}\notag\\
      &+\frac{1}{8}\qty[\hat{g}^{\bullet\bullet}r_1^\bullet]^{\mu_1\mu_2\mu_3}\qty(\frac{1}{2}S_{11}+\frac{1}{3}S_{12}+\frac{1}{6}S_{22})\notag\\
      &+\frac{1}{8}\qty[\hat{g}^{\bullet\bullet}r_2^\bullet]^{\mu_1\mu_2\mu_3}\qty(\frac{1}{6}S_{11}+\frac{1}{3}S_{12}+\frac{1}{2}S_{22})\notag\\
      =& -\frac{1}{12}(3r_1^{\mu_1}r_1^{\mu_2}r_1^{\mu_3}+3r_2^{\mu_1}r_2^{\mu_2}r_2^{\mu_3}\notag\\
      &\qquad+r_1^{\mu_1}r_1^{\mu_2}r_2^{\mu_3}+r_1^{\mu_1}r_2^{\mu_2}r_1^{\mu_3}+r_2^{\mu_1}r_1^{\mu_2}r_1^{\mu_3}\notag\\
      &\qquad+r_1^{\mu_1}r_2^{\mu_2}r_2^{\mu_3}+r_2^{\mu_1}r_1^{\mu_2}r_2^{\mu_3}+r_2^{\mu_1}r_2^{\mu_2}r_1^{\mu_3})\notag\\
      &+\frac{1}{24}\qty[\hat{g}^{\bullet\bullet}r_1^\bullet]^{\mu_1\mu_2\mu_3}\qty(r_1^2+r_2^2-2r_1\cdot r_2-4m_1^2-2m_2^2)\notag\\
      &+\frac{1}{24}\qty[\hat{g}^{\bullet\bullet}r_2^\bullet]^{\mu_1\mu_2\mu_3}\qty(r_1^2+r_2^2-2r_1\cdot r_2-2m_1^2-4m_2^2)\,,\\[10pt]
%
    I_2^{D,1,0;\mu_1} =& \frac{1}{4}r_1^{\mu_1}\qty(\frac{1}{2}S_{11}+\frac{1}{3}S_{12}+\frac{1}{6}S_{22})\notag\\
      &+\frac{1}{4}r_2^{\mu_1}\qty(\frac{1}{6}S_{11}+\frac{1}{3}S_{12}+\frac{1}{2}S_{22})\notag\\
      =& \frac{1}{12}r_1^{\mu_1}\qty(r_1^2+r_2^2-2r_1\cdot r_2-4m_1^2-2m_2^2)\notag\\
      &+\frac{1}{12}r_2^{\mu_1}\qty(r_1^2+r_2^2-2r_1\cdot r_2-2m_1^2-4m_2^2)\,.%\,,\\[10pt]
%
% not currently implemented:
%    I_2^{D,2,0} =& -\frac{1}{120}\qty(3S_{11}^2+2S_{12}^2+3S_{22}^2+S_{11}S_{22}+3S_{11}S_{12}+3S_{12}S_{22})\,.
\end{align}
\end{subequations}

\subsection*{Triangles}
\begin{subequations}
\begin{align}
  I_3^{D,0,1;\mu_1\mu_2\mu_3\mu_4} =& \frac{1}{48}\sum_{j_1,j_2=1}^3\qty[\hat{g}^{\bullet\bullet}r_{j_1}^\bullet r_{j_2}^\bullet]^{\mu_1\mu_2\mu_3\mu_4} + \frac{1}{48}\sum_{j=1}^3\qty[\hat{g}^{\bullet\bullet}r_j^\bullet r_j^\bullet]^{\mu_1\mu_2\mu_3\mu_4}\notag\\
    &-\frac{1}{96}\qty[\hat{g}^{\bullet\bullet}\hat{g}^{\bullet\bullet}]^{\mu_1\mu_2\mu_3\mu_4}\qty(S_{11}+S_{22}+S_{33}+S_{12}+S_{13}+S_{23})\,,\\[10pt]
%
  I_3^{D,1,0;\mu_1\mu_2} =& \frac{1}{24}\sum_{j_1,j_2=1}^3\qty[r_{j_1}^\bullet r_{j_2}^\bullet]^{\mu_1\mu_2} + \frac{1}{24}\sum_{j=1}^3\qty[r_j^\bullet r_j^\bullet]^{\mu_1\mu_2}\notag\\
    &-\frac{1}{48}\hat{g}^{\mu_1\mu_2}\qty(S_{11}+S_{22}+S_{33}+S_{12}+S_{13}+S_{23})\,,\\[10pt]
%
  I_3^{D,2,0} =& \frac{1}{24}\qty(S_{11}+S_{22}+S_{33}+S_{12}+S_{13}+S_{23})\,.
\end{align}
\end{subequations}
We have
\begin{multline}
  S_{11}+S_{22}+S_{33}+S_{12}+S_{13}+S_{23} \\= 2\qty(r_1^2+r_2^2+r_3^2-r_1\cdot r_2-r_1\cdot r_3-r_2\cdot r_3)-4\qty(m_1^2-m_2^2-m_3^2)
\end{multline}

\subsection*{Boxes}
\begin{subequations}
\begin{align}
  I_4^{D,0,1;\mu_1\mu_2\mu_3\mu_4\mu_5} =& -\frac{1}{96}\sum_{j_1=1}^4\qty[\hat{g}^{\bullet\bullet}\hat{g}^{\bullet\bullet}r_{j_1}^\bullet]^{\mu_1\mu_2\mu_3\mu_4\mu_5}\,,\\[10pt]
%
  I_4^{D,1,0;\mu_1\mu_2\mu_3} =& -\frac{1}{48}\sum_{j_1=1}^4\qty[\hat{g}^{\bullet\bullet}r_{j_1}^\bullet]^{\mu_1\mu_2\mu_3}\,,\\[10pt]
%
  I_4^{D,2,0;\mu_1} =& \frac{1}{24}\qty(r_1^{\mu_1}+r_2^{\mu_1}+r_3^{\mu_1}+r_4^{\mu_1})\,.
\end{align}
\end{subequations}

\subsection*{Pentagons}
\begin{subequations}
\begin{align}
  I_5^{D,0,1;\mu_1\mu_2\mu_3\mu_4\mu_5\mu_6} &= \frac{1}{192}\qty[\hat{g}^{\bullet\bullet}\hat{g}^{\bullet\bullet}\hat{g}^{\bullet\bullet}]^{\mu_1\mu_2\mu_3\mu_4\mu_5\mu_6}\,,\\[10pt]
%
  I_5^{D,1,0;\mu_1\mu_2\mu_3\mu_4} &= \frac{1}{96}\qty[\hat{g}^{\bullet\bullet}\hat{g}^{\bullet\bullet}]^{\mu_1\mu_2\mu_3\mu_4}\,,\\[10pt]
%
  I_5^{D,2,0;\mu_1\mu_2} &= -\frac{1}{48}\qty[\hat{g}^{\bullet\bullet}]^{\mu_1\mu_2}\,,\\[10pt]
%
  I_5^{D,3,0} &= \frac{1}{12}\,.
\end{align}
\end{subequations}

\chapter{The included model files}
\label{chp:model-files}

\section{Format of the model files}\label{sec:modelfiles}
\gosamv{} expects three files for a proper model definition:

\begin{tabular}{r l}
\texttt{<model>.hh} & \form{} file containing the Feynman rules \\
\texttt{<model>.py} & \python{} file \\
\texttt{<model>} (no extension) & \qgraf{} model file \\
\end{tabular}

\subsection{The \python{} file}
Thy \python{} file contains the following definitions

\begin{tabular}{r p{0.7\textwidth}}
\texttt{model\_name} & a variable of string type containing a human-readable
     name for this model, such as ``Standard Model (Feyn. Gauge) w/o Higgs'' etc. \\
\texttt{particles} & a \python{} \texttt{dict} that contains all particles
     \emph{and} anti-particles of the model. The keys are the \qgraf{} names of the
     fields; the values are objects of the class \texttt{Particle}.
     The constructor has the arguments
     \begin{lstlisting}[style=py]
Particle(name,two_spin,mass,color_rep,partner,width='0',charge)
     \end{lstlisting} \\
\texttt{mnemonics} & a \python{} \texttt{dict} of
     human-readable particle names. The values are objects of the class
     \texttt{Particle}. It is save to refer to the dictionary \texttt{particles}. \\
\texttt{parameters} & a \python{} \texttt{dict} of
     model parameters with their default values. Both key and value are strings. \\
\texttt{functions} & a \python{} \texttt{dict} of
     variable names and initialization expressions. Both key and value are strings. \\
\texttt{types} & the types of all parameters and functions indicated by
     \texttt{'R'} for real numbers and \texttt{'C'} for complex numbers. \\
\texttt{latex\_names} & a \python{} \texttt{dict} assigning \LaTeX{}
     code to the field names. Math mode is assumed. \\
\texttt{line\_styles} & a \python{} \texttt{dict} assigning line styles
     to field names. The line style used when drawing Feynman diagrams.
     Allowed values are \texttt{photon}, \texttt{ghost}, \texttt{scalar},
     \texttt{gluon}, \texttt{fermion}.
\end{tabular}

\subsection{The \qgraf{} file}
The propagators in the \qgraf{} file must contain the following functions:

\begin{longtable}{r p{0.7\textwidth}}
\texttt{TWOSPIN} & twice the spin of the particle. \\
\texttt{COLOR} &   the color representation of the particle $\in\{1,3,8\}$. \\
\texttt{MASS} &    the mass of the particle. \\
\texttt{WIDTH} &   the width of the particle (currently not used). \\
\texttt{AUX} &     must be zero for most fields. Tensor Ghosts, as introduced
                        by CalcHep have the value $1$ here. \\
\texttt{CONJ} &    for self-conjugate particles the value is \texttt{('+')},
                        otherwise it is \texttt{('+','-')}. \\
\end{longtable}

The vertices must provide all fields that should be accessible in \texttt{VSUM} statements
and therefore also the ones that \gosamv{} uses in the \texttt{order} option.

\subsection{The \form{} file}
There are two possible ways of specifying the Feynman rules in the \form{} file.
If a model contains only Standard Model like interactions one can make use of
the file \texttt{src/form/vertices.hh} in the \gosamv{} directory and just define
the coefficients \texttt{CL} and \texttt{CR} in front of the vertices. This
strategy is implemented by the modelfiles \texttt{models/sm}. The file
\form{} contains a procedure \texttt{VertexConstants} which
replaces the the vertex constants by their symbols. A QED example would be
\begin{lstlisting}[style=form]
#Procedure VertexConstants
   Id CL([field.em], [field.ep], [field.ph]) = e;
   Id CR([field.em], [field.ep], [field.ph]) = e;
#EndProcedure
\end{lstlisting}
In the header of the \form{} file all model specific
symbols and functions need to be defined. For this simple
model we have the fields and the coupling constant as only
new symbols.
\begin{lstlisting}[style=form]
Symbols [field.em], [field.ep], [field.ph], e;
\end{lstlisting}

Instead of using the file \texttt{vertices.hh} one can also use
his own vertex definitions. In this case the \form{} file must contain
the definition
\begin{lstlisting}[style=form]
#Define USEVERTEXPROC "1"
\end{lstlisting}
and it must define the procedure \texttt{ReplaceVertices}. An example
for QED is given below.
\begin{lstlisting}[style=form]
#Procedure ReplaceVertices
Identify Once vertex(iv?,
      [field.ep], idx1?, -1, k1?, idx1L1?, -1, idx1C1?,
      [field.em], idx2?,  1, k2?, idx2L1?,  1, idx2C1?,
      [field.ph], idx3?,  2, k3?, idx3L2?,  1, idx3C1?) =
   PREFACTOR(i_ * e) *
   NCContainer(Sm(idx3L2), idx1L1, idx2L1) *
   node(idx1, idx2, idx3);
#EndProcedure
\end{lstlisting}
It should be noted that \gosamv{} expects the procedure \texttt{VertexConstants}
to exist in both cases. If all the constants are already substituted inside
\texttt{ReplaceVertices} the file must still provide a possibly empty empty
implementation of \texttt{VertexConstants}. \gosamv{} ensures that
\texttt{VertexConstants} is always called after \texttt{ReplaceVertices}.

It is recommended to wrap any factors that are global prefactors to the diagram
into the argument of the function \texttt{PREFACTOR} as \gosamv{} scans for these
functions and brackets them out. Each vertex definition must contain a factor
\texttt{node} which contains the indices\footnote{In \qgraf's terminology
these indices are a combination of vertex and ray index of the field.}
of the fields at this vertex. \newpage

The \qgraf{} style file generates vertex functions as follows:

\begin{lstlisting}[style=form, mathescape]
vertex(vertex index,
    field$_1$, index$_1$, $\pm$$2$spin$_1$, momentum$_1$, $\mu_1$, $\pm$color rep$_1$, color index$_1$,
    field$_2$, index$_2$, $\pm$$2$spin$_2$, momentum$_2$, $\mu_2$, $\pm$color rep$_2$, color index$_2$,
    $\vdots$
    field$_n$, index$_n$, $\pm$$2$spin$_n$, momentum$_n$, $\mu_n$, $\pm$color rep$_n$, color index$_n$
  )
\end{lstlisting}

The entries are:
\begin{longtable}{r p{0.75\textwidth}}
\texttt{vertex index} & The unique index of this vertex. (\texttt{iv1}, \texttt{iv2}, \dots) \\
\texttt{field}$_i$ & The field name of the $i$-th particle. These names are constructed from the \qgraf{} field name as \texttt{[field.$\langle name\rangle$]}. \\
\texttt{index}$_i$ & A unique name for this ``ray'' (at index $1$ they are \texttt{idx1r1}, \texttt{idx1r2}, \ldots) \\
$\pm2$\texttt{spin}$_i$ & twice the spin of the $i$-th particle.
   The sign distinguishes particles~($+$) from antiparticles~($-$). \\
\texttt{momentum}$_i$ & the incoming momentum of the $i$-th particle. \\
$\mu_i$ & the Lorentz index of the $i$-th particle. Depending on the spin of the particle
   this is a spinor index (spin $1/2$), a Lorentz index (spin $1$) or a dummy index (spin $0$).
   For higher spins this index must be split into its components using the function
   \texttt{SplitLorentzIndex}. For its proper definition the reader is referred to
   the document \texttt{src/form/lorentz.pdf}. \\
$\pm$\texttt{color rep}$_i$ & the color representation of the $i$-th particle. Allowed
   values currently are $\pm1,\pm3,\pm8$, although the sign only really makes sense for the
   fundamental representation $3$ and its conjugate $\bar{3}\equiv-3$. \\
\texttt{color index}$_i$ & The color index of the $i$-th particle. Depending on the color
   representation this is an index in the fundamental, the adjoint or the trivial representation.
\end{longtable}

All symbols defined in \texttt{src/form/symbols.hh} are also accessible in this \form{} file.

\attention{
Note: until recently the definitions of \texttt{Sqrt2} and \texttt{sqrt2} were part
of the model file. Now these symbols are part of \texttt{src/form/symbols.hh} and must not be
redefined.
}

\attention{
   All Dirac matrices and metric tensors must use the notation introduced by \texttt{spinney}.
The metric tensor is $g^{\mu\nu}=\mathtt{d}(\mu, \nu)$ and $\gamma^\mu=\mathtt{Sm}(\mu)$,
$\gamma_5=\mathtt{Gamma5}$, $\Pi_+=\mathtt{ProjPlus}$, $\Pi_-=\mathtt{ProjMinus}$. All non-commuting
objects must reside inside the function \texttt{NCContainter} (see~example).
}

The color structure must use the objects $t_{ij}^A=\mathtt{T}(A, i, j)$ (where the color flow is such
that$j$ is the index of an anti-quark), $f^{ABC}=\mathtt{f}(A, B, C)$ and
$f^{ABE}f^{CDE}=\mathtt{f4}(A,B,C,D)$. At vertices coupling colored with colorless particles
it might be necessary to use the \texttt{d\_} tensor to file the color flow through the vertex.

\attention{
   Note that all propagators and wave functions are defined in a model independent
way in the files \texttt{src/form/propagators.hh} and \texttt{src/form/legs.hh}. Please,
refrain from modifying these files directly but make all changes to \texttt{src/form/lorentz.nw}.
}

In theories with Majorana fermions the model file should include the following
line:
\begin{lstlisting}[style=form]
#Define DISPOSEQGRAFSIGN "1"
\end{lstlisting}

\section{Standard Model (\texttt{sm})}
\label{sec:model-files:sm}
\subsection{Synopsis}
The model `\texttt{sm}' contains the Feynman rules for the
Standard Model in Feynman gauge as described
in~\cite[Appendix~A]{Boehm:2001}.
\renewcommand{\arraystretch}{1.1}
\subsection{Particle content}
\subsubsection{Leptons}
\begin{longtable}{|l|l|l|p{2cm}|}
\hline
Name&Alternative Names&Mass&Comment\\
\hline
\texttt{ep }& \texttt{positron e+ }& \texttt{me}& $e^+$\\
\texttt{em }& \texttt{electron e- }& \texttt{me}& $e^-$\\
\texttt{ne }& & $0$ & $\nu_e$\\
\texttt{nebar }& \texttt{ne\~}& $0$ & $\bar{\nu}_e$\\
\hline
\texttt{mup }& \texttt{mu+ }& \texttt{mmu}& $\mu^+$\\
\texttt{mum }& \texttt{mu- }& \texttt{mmu}& $\mu^-$\\
\texttt{nmu }& & $0$ & $\nu_\mu$\\
\texttt{nmubar }& \texttt{nmu\~ }& $0$ & $\bar{\nu}_\mu$\\
\hline
\texttt{taup }& \texttt{tau+ }& \texttt{mtau}& $e^+$\\
\texttt{taum }& \texttt{tau- }& \texttt{mtau}& $e^-$\\
\texttt{ntau }& & $0$ & $\nu_\tau$\\
\texttt{ntaubar }& \texttt{ntau\~ }& $0$ & $\bar{\nu}_\tau$\\
\hline
\end{longtable}

\subsubsection{Quarks}
\begin{longtable}{|l|l|l|p{2cm}|}
\hline
Name&Alternative Names&Mass&Comment\\
\hline
\texttt{U }& \texttt{u }& \texttt{mU}& $u$\\
\texttt{Ubar }& \texttt{u\~ }& \texttt{mU}& $\bar{u}$\\
\texttt{D }& d & \texttt{mD }& $d$\\
\texttt{Dbar }& \texttt{d\~}& mD & $\bar{d}$\\
\hline
\texttt{S }& \texttt{s }& \texttt{mS}& $u$\\
\texttt{Sbar }& \texttt{s\~ }& \texttt{mS}& $\bar{u}$\\
\texttt{C }& c & \texttt{mC }& $d$\\
\texttt{Cbar }& \texttt{c\~}& mC & $\bar{d}$\\
\hline
\texttt{T }& \texttt{t }& \texttt{mT}& $t$\\
\texttt{Tbar }& \texttt{t\~ }& \texttt{mT}& $\bar{t}$\\
\texttt{B }& b & \texttt{mB }& $b$\\
\texttt{Bbar }& \texttt{b\~}& mB & $\bar{b}$\\
\hline
\end{longtable}

\subsubsection{Gauge Bosons}
\begin{longtable}{|l|l|l|p{2cm}|}
\hline
Name&Alternative Names&Mass&Comment\\
\hline
\texttt{g }& \texttt{gluon }& $0$ & $g$ \\
\texttt{A }& \texttt{photon gamma }& $0$ & $\gamma$ \\
\texttt{Z }& & \texttt{mZ }& $Z$ \\
\texttt{Wp }& \texttt{W+}& \texttt{mW }& $W^+$ \\
\texttt{Wm }& \texttt{W-}& \texttt{mW }& $W^-$ \\
\hline
\end{longtable}

\subsubsection{Scalar Bosons}
\begin{longtable}{|l|l|l|p{2cm}|}
\hline
Name&Alternative Names&Mass&Comment\\
\hline
\texttt{H }& \texttt{h higgs }& \texttt{mH }& $H$ \\
\texttt{phim }& \texttt{phi- }& \texttt{mW }& $\phi^-$ \\
\texttt{phip }& \texttt{phi+ }& \texttt{mW }& $\phi^+$ \\
\texttt{chi }&  & \texttt{mZ }& $\chi$ \\
\hline
\end{longtable}

\subsubsection{Ghost Fields}
\begin{longtable}{|l|l|l|p{2cm}|}
\hline
Name&Alternative Names&Mass&Comment\\
\hline
\texttt{gh }&  & $0$ & $u^g$\\
\texttt{ghbar }&  & $0$ & $\bar{u}^g$ \\
\texttt{ghA }&  & $0$ & $u^A$ \\
\texttt{ghAbar }&  & $0$ & $\bar{u}^A$ \\
\texttt{ghZ }&  & \texttt{mZ }& $u^Z$ \\
\texttt{ghZbar }&  & \texttt{mZ }& $\bar{u}^Z$ \\
\texttt{ghWp }&  & \texttt{mW }& $u^+$ \\
\texttt{ghWpbar }&  & \texttt{mW }& $\bar{u}^+$ \\
\texttt{ghWm }&  & \texttt{mW }& $u^-$ \\
\texttt{ghWmbar }&  & \texttt{mW }& $\bar{u}^-$ \\
\hline
\end{longtable}

\subsection{Parameters}
This section lists all model parameters which are not already
listed as particle masses.

\medskip
\begin{longtable}{|l|l|l|}
\hline
Name & Symbol & Description\\
\hline
\texttt{NC }& $N_C$ & Number of colors in QCD\\
\texttt{e }& $e$ & electro-weak coupling constant: $\alpha=e^2/(4\pi)$\\
\texttt{gs }& $g_s$ & strong coupling constant: $\alpha_s=g_s^2/(4\pi)$\\
\texttt{sw }& $s_w=\sin\theta_w$ & sine of weak mixing angle\\
\texttt{cw }& $c_w=\cos\theta_w$ & cosine of weak mixing angle\\
\texttt{VUD }& $V_{ud}$ & CKM mixing matrix element\\
\texttt{CVDU }& $V_{du}^{\dagger}$ & --- '' ---\\
\texttt{VUS }& $V_{us}$ & --- '' ---\\
\texttt{CVSU }& $V_{su}^{\dagger}$ & --- '' ---\\
\texttt{VUB }& $V_{ub}$ & --- '' ---\\
\texttt{CVBU }& $V_{bu}^{\dagger}$ & --- '' ---\\
\texttt{VCD }& $V_{cd}$ & --- '' ---\\
\texttt{CVDC }& $V_{dc}^{\dagger}$ & --- '' ---\\
\texttt{VCS }& $V_{cs}$ & --- '' ---\\
\texttt{CVSC }& $V_{sc}^{\dagger}$ & --- '' ---\\
\texttt{VCB }& $V_{cb}$ & --- '' ---\\
\texttt{CVBC }& $V_{bc}^{\dagger}$ & --- '' ---\\
\texttt{VTD }& $V_{td}$ & --- '' ---\\
\texttt{CVTD }& $V_{dt}^{\dagger}$ & --- '' ---\\
\texttt{VTS }& $V_{ts}$ & --- '' ---\\
\texttt{CVST }& $V_{st}^{\dagger}$ & --- '' ---\\
\texttt{VTB }& $V_{tb}$ & --- '' ---\\
\texttt{CVTB }& $V_{bt}^{\dagger}$ & --- '' ---\\
\hline
\end{longtable}
\renewcommand{\arraystretch}{1.25}
\section{\gosam directory structure}
The \gosamv source directory has the structure as described below:

\begin{longtable}{r p{0.7\textwidth}}

\texttt{doc/} & This directory contains the documentation
and example setup files. You can run \texttt{make} in this directory
to generate the document \texttt{refman.pdf}; this is the document you
are currently reading. \\

\texttt{models/} & For each implemented model this directory
contains the \qgraf model file (no extension), a \form interface
(\texttt{*.hh}) and a \python module (\texttt{*.py}). Currently,
the Standard Model (\texttt{sm}) is distributed with \gosamv, 
where several variants are available:
\begin{itemize}
  \item \texttt{smdiag} implements a diagonal flavour structure ($V_{\text{CKM}}=\mathrm{diag}\{1,1,1\}$),
  \item \texttt{smehc} contains effective gluon-Higgs couplings,
  \item \texttt{sm\_complex} and \texttt{smdiag\_complex} support the complex mass scheme. 
\end{itemize}
The structure of the model files is discussed in more detail in
Chapter~\ref{sec:modelfiles}. Model files for the MSSM based on 
FeynRules/UFO~\cite{Degrande:2011ua}  and LanHEP~\cite{Semenov:2010qt}
can be found in the directory 
\texttt{examples/model/}, as well as \texttt{UFO}  files for ADD~\cite{ArkaniHamed:1998rs}
models with large extra dimensions (LED). \\

\texttt{templates/} & Contains templates for the creation
of the files in the process directory. The contents are transformed
by the class
\texttt{golem.util.parser.Template} and its subclasses
in \texttt{golem.templates.*}. The translation of the templates is
controled by the file \texttt{templates.xml} of the same directory. \\

\texttt{src/python/} & All model independent \python modules
can be found in this directory tree. \\

\texttt{src/form/} & Here one finds all \form files
which are not part of the template. \\

\texttt{build} & This directory is created during
building and installation of this package by running \texttt{meson setup build}.
The files in this directory are of temporary nature and can be safely
removed. \\

\texttt{examples} & This directory contains some simple examples
of validated processes. \\

\texttt{olp} & Files in this directory are used by
\texttt{gosam.py --olp}, which is \gosamv's implementation of the
Binoth Les Houches interface for one-loop programs (BLHA).
Both the original standards ~\cite{Binoth:2010xt} and the new standards 
(BLHA2)~\cite{Alioli:2013nda} are supported by \gosam-\gosamversion. \\


\end{longtable}

\chapter{Available process card options} \label{chp:process_card_options}
\begin{basedescript}{\desclabelstyle{\pushlabel}}
\setlength\itemsep{30pt}
\item[\colorbox{gray!30}{\texttt{process\_name}}] (\textit{text})
\begin{verbatim}
A symbolic name for this process. This name will be used
as a prefix for the Fortran modules.

Golem will insert an underscore after this prefix.
If the process name is left blank no prefix will be used
and no extra underscore will be generated.
\end{verbatim}
Hidden: \verb|False|
\\Experimental: \verb|False|
\\\item[\colorbox{gray!30}{\texttt{process\_path}}] (\textit{text})
\begin{verbatim}
The path to which all Form output is written.
If no absolute path is given, the path is interpreted relative
to the working directory from which gosam.py is run.

Example:
process_path=/scratch/golem_processes/process1
\end{verbatim}
Hidden: \verb|False|
\\Experimental: \verb|False|
\\\item[\colorbox{gray!30}{\texttt{in}}] (\textit{comma separated list})
\begin{verbatim}
A comma-separated list of initial state particles.
Which particle names are valid depends on the
model file in use.

Examples (Standard Model):
1) in=u,u~
2) in=e+,e-
3) in=g,g
\end{verbatim}
Hidden: \verb|False|
\\Experimental: \verb|False|
\\\item[\colorbox{gray!30}{\texttt{out}}] (\textit{comma separated list})
\begin{verbatim}
A comma-separated list of final state particles.
Which particle names are valid depends on the
model file in use.

Examples (Standard Model):
1) out=H,u,u~
2) out=e+,e-,gamma
3) out=b,b~,t,t~
\end{verbatim}
Hidden: \verb|False|
\\Experimental: \verb|False|
\\\item[\colorbox{gray!30}{\texttt{model}}] (\textit{comma separated list})
\begin{verbatim}
This option allows the selection of a model for the
Feynman rules. It has to conform with one of four possible
formats:

1) model=<name>
2) model=<path>, <name>
3) model=<path>, <number>
4) model=FeynRules, <path>

Format 1) searches for the model files <name>, <name>.hh
and <name>.py in the models/ directory under the installation
path of Golem.

Format 2) is similar to format 1) but <path> is used instead
of the models/ directory of the Golem installation

Format 3) expects the files func<number>.mdl, lgrng<number>.mdl,
prtcls<number>.mdl and vars<number>.mdl in the directory <path>.
These files need to be in CalcHEP/CompHEP format.

Format 4) expects a UFO model in the directory specified by <path>.
\end{verbatim}
Default: \verb|smdiag|
\\Hidden: \verb|False|
\\Experimental: \verb|False|
\\\item[\colorbox{gray!30}{\texttt{model.options}}] (\textit{comma separated list})
\begin{verbatim}
If the model in use supports options they can be passed via this
property.

For builtin models, the option "ewchoose"
automatically selects the EW scheme used.
\end{verbatim}
Default: \verb|ewchoose|
\\Hidden: \verb|False|
\\Experimental: \verb|False|
\\\item[\colorbox{gray!30}{\texttt{order}}] (\textit{comma separated list})
\begin{verbatim}
A 3-tuple <coupling>,<born>,<virt> where <coupling> denotes
a function of the qgraf style file which can be used as
an argument in a 'vsum' statement. For the standard model
file 'sm' there are two such functions, 'gs' which counts
powers of the strong coupling and 'gw' which counts powers
of the weak coupling. <born> is the sum of powers for the
tree level amplitude and <virt> for the virtual amplitude.
The line
   order = gs, 4, 6
would select all diagrams which have (gs)^4 at tree level
and all loop graphs with (gs)^6.

Note: The line
   order = gw, 2, 2
does not imply that no virtual corrections are calculated.
Instead, for the virtual corrections diagrams are chosen
with the same order in gw but higher order in gs.

In other models with more than two different coupling
constants additional 'vsum' statements, which can be passed
via the qgraph.verbatim option, might be needed
to select the correct set of diagrams.

The user can also use QCD instead of gs and QED instead of gw.

If the last number is omitted no virtual corrections are
calculated.

For loop induced processes, the order of the Born diagrams
should be specified as `NONE`.

See also: qgraf.options, qgraf.verbatim
\end{verbatim}
Hidden: \verb|False|
\\Experimental: \verb|False|
\\\item[\colorbox{gray!30}{\texttt{loop\_suppressed\_Born}}] (\textit{true/false})
\begin{verbatim}
In case of a a loop-induced process generate Born diagrams with tree
topology containing loop-suppressed EFT operators.
\end{verbatim}
Default: \verb|False|
\\Hidden: \verb|False|
\\Experimental: \verb|False|
\\\item[\colorbox{gray!30}{\texttt{zero}}] (\textit{comma separated list})
\begin{verbatim}
A list of symbols that should be treated as identically
zero throughout the whole calculation. All of these
symbols must be defined by the model file. For convenience
masses and widths can be set by means of PDG codes, e.g.
mass(1),width(1) for the down-quark mass and width, res-
pectively. Lists of PDG codes separated by ';' can be used
in the arguments of 'mass' and 'width'.

Examples:
1) # Light masses are set to zero here:
   zero=me,mD,mU,mS
   OR:
   zero=mass(11),mass(1),mass(2),mass(3)
   OR:
   zero=mass(11;1;2;3)
2) # Diagonal CKM matrix:
   zero=VUS, VUB, CVDC, CVDT, \
        VCD, VCB, CVSU, CVST, \
        VTD, VTS, CVBU, CVBC
   one=  VUD,  VCS,  VTB, \
        CVDU, CVSC, CVBT

See also: model, one
\end{verbatim}
Hidden: \verb|False|
\\Experimental: \verb|False|
\\\item[\colorbox{gray!30}{\texttt{one}}] (\textit{comma separated list})
\begin{verbatim}
A list of symbols that should be treated as identically
one throughout the whole calculation. All of these
symbols must be defined by the model file.

Example:
one=gs, e

See also: model, zero
\end{verbatim}
Hidden: \verb|False|
\\Experimental: \verb|False|
\\\item[\colorbox{gray!30}{\texttt{renorm}}] (\textit{true/false})
\begin{verbatim}
Indicates if the UV counterterms should be generated.

Examples:
renorm=true
renorm=false
\end{verbatim}
Default: \verb|True|
\\Hidden: \verb|False|
\\Experimental: \verb|False|
\\\item[\colorbox{gray!30}{\texttt{helsum}}] (\textit{true/false})
\begin{verbatim}
Flag whether or not 1-loop diagrams should be analytically
summed over all helicities
\end{verbatim}
Default: \verb|False|
\\Hidden: \verb|False|
\\Experimental: \verb|True|
\\\item[\colorbox{gray!30}{\texttt{regularisation\_scheme}}] (\textit{text})
\begin{verbatim}
Sets the used regularisation scheme, dimensional reduction (DRED)
or 't Hooft-Veltman (tHV).
Possible values: dred (recommended), thv
\end{verbatim}
Default: \verb|dred|
\\Hidden: \verb|False|
\\Experimental: \verb|False|
\\\item[\colorbox{gray!30}{\texttt{convert\_to\_thv}}] (\textit{true/false})
\begin{verbatim}
Sets the name of the same variable in config.f90

Activates or disables the conversion of the result into the 't Hooft-Veltman
(tHV) regularisation scheme, when the calculation has been performed in DRED.

Does not have an effect when tHV is picked as regularisation scheme in
extensions or via property 'regularisation_scheme'.
\end{verbatim}
Default: \verb|False|
\\Hidden: \verb|False|
\\Experimental: \verb|False|
\\\item[\colorbox{gray!30}{\texttt{helicities}}] (\textit{comma separated list})
\begin{verbatim}
A list of helicities to be calculated. An empty list
means that all possible helicities should be generated.

The helicities are specified as a string of characters
according to the following table:

   spin massive  |  'm'  '-'  '0'   '+'   'k'
     0   YES/NO  | ---- ----    0  ----  ----
   1/2   YES/NO  | ---- -1/2 ----  +1/2  ----
     1     NO    | ----   -1 ----    +1  ----
     1    YES    | ----   -1    0    +1  ----
   3/2     NO    | -3/2 ---- ----  ----  +3/2
   3/2    YES    | -3/2 -1/2 ----  +1/2  +3/2
     2     NO    |   -2 ---- ----  ----    +2
     2    YES    |   -2   -1    0    +1    +2

Please, note that 'k' and 'm' are not in use yet but reserved
for future extensions to higher spins.

The characters correspond to particle 1, 2, ... from left to
right.

Examples:
   # e+, e- --> gamma, gamma:
   # Only three helicities required; the other ones are
   # either zero or can be obtained by symmetry
   # transformations.
   helicities=+-++,+-+-,+---;

Multiple helicities can be encoded in patterns, which are expanded
at the time of code generation. Patterns can have one of the following
forms:
   [+-], [+-0], [+0] etc. : the bracket expands to one of the symbols
         in the bracket at a time.
   EXAMPLE
         helicities=[+-]+[+-0]
         # expands to 6 different helicities:
         # helicities=+++, ++-, ++0, -++, -+-, -+0
   [a=+-], etc. : as above, but the helicity is also assigned to the
         symbol and can be reused.
   EXAMPLE
         helicities=[i=+-]+i+
         # expands to two helicities
         # helicities=++++, -+-+
   [ab=+-0], etc. : as above, the first symbol is assigned the helicity,
         the second is minus the helicity
   EXAMPLE
         helicities=[qQ=+-][pP=+-]PQ[+-0]
         # expands to 12 helicities
         # helicities=++--+,++---,++--0,+-+-+,+-+--,+-+-0,\
         #            -+-++,-+-+-,-+-+0,--+++,--++-,--++0
\end{verbatim}
Hidden: \verb|False|
\\Experimental: \verb|False|
\\\item[\colorbox{gray!30}{\texttt{qgraf.options}}] (\textit{comma separated list})
\begin{verbatim}
A list of options which is passed to qgraf via the 'options' line.
Possible values (as of qgraf.3.1.1) are zero, one or more of:
   onepi, onshell, nosigma, nosnail, notadpole, floop
   topol

Please, refer to the QGraf documentation for details.
\end{verbatim}
Default: \verb|onshell,notadpole,nosnail|
\\Hidden: \verb|False|
\\Experimental: \verb|False|
\\\item[\colorbox{gray!30}{\texttt{filter.particles}}] (\textit{text})
\begin{verbatim}
Restrict the number of internal propagators with the given field
in every diagram. Multiple fields may be specified.

Example:

filter.particles=u:0,d:0 # No internal u or d quarks
\end{verbatim}
Hidden: \verb|False|
\\Experimental: \verb|False|
\\\item[\colorbox{gray!30}{\texttt{filter.lo.particles}}] (\textit{text})
\begin{verbatim}
Restrict the number of internal propagators with the given field
in the LO diagrams. Multiple fields may be specified.

See also filter.particles.
\end{verbatim}
Hidden: \verb|False|
\\Experimental: \verb|False|
\\\item[\colorbox{gray!30}{\texttt{filter.nlo.particles}}] (\textit{text})
\begin{verbatim}
Restrict the number of internal propagators with the given field
in the NLO diagrams. Multiple fields may be specified.

See also filter.particles.
\end{verbatim}
Hidden: \verb|False|
\\Experimental: \verb|False|
\\\item[\colorbox{gray!30}{\texttt{filter.ct.particles}}] (\textit{text})
\begin{verbatim}
Restrict the number of internal propagators with the given field
in the counter term diagrams. Multiple fields may be specified.

See also filter.particles.
\end{verbatim}
Hidden: \verb|False|
\\Experimental: \verb|False|
\\\item[\colorbox{gray!30}{\texttt{qgraf.verbatim}}] (\textit{text})
\begin{verbatim}
This option allows to send verbatim lines to
the file qgraf.dat. This can be useful if the user
wishes to put additional restricitons to the selected diagrams.
This option is mainly inteded for the use of the operators
   rprop, iprop, chord, bridge, psum
Note, that the use of 'vsum' might interfer with the
option qgraf.power.

Example:
qgraf.verbatim=\
   # no top quarks: \n\
   true=iprop[T, 0, 0];\n\
   # at least one Higgs:\n\
   false=iprop[H, 0, 0];\n

Please, refer to the QGraf documentation for details.

See also: qgraf.options, order
\end{verbatim}
Hidden: \verb|False|
\\Experimental: \verb|False|
\\\item[\colorbox{gray!30}{\texttt{qgraf.verbatim.lo}}] (\textit{text})
\begin{verbatim}
Same as qgraf.verbatim but only applied to LO diagrams.

See also: qgraf.verbatim, qgraf.verbatim.nlo
\end{verbatim}
Hidden: \verb|False|
\\Experimental: \verb|False|
\\\item[\colorbox{gray!30}{\texttt{qgraf.verbatim.nlo}}] (\textit{text})
\begin{verbatim}
Same as qgraf.verbatim but only applied to NLO diagrams.

See also: qgraf.verbatim, qgraf.verbatim.nlo
\end{verbatim}
Hidden: \verb|False|
\\Experimental: \verb|False|
\\\item[\colorbox{gray!30}{\texttt{diagsum}}] (\textit{true/false})
\begin{verbatim}
Flag whether or not 1-loop diagrams with the same propagators
should be summed before the algebraic reduction.
\end{verbatim}
Default: \verb|True|
\\Hidden: \verb|False|
\\Experimental: \verb|False|
\\\item[\colorbox{gray!30}{\texttt{reduction\_programs}}] (\textit{comma separated list})
\begin{verbatim}
Specifies the reduction libraries which should be supported. To use golem95,
GoSam has to be compiled with support for golem95.

Possible values: ninja, golem95

Default: ninja

See also reduction_interoperation, reduction_interoperation_rescue.
\end{verbatim}
Default: \verb|ninja|
\\Hidden: \verb|False|
\\Experimental: \verb|False|
\\\item[\colorbox{gray!30}{\texttt{polvec}}] (\textit{text})
\begin{verbatim}
Evaluate the polarisation vector
'numerical' or 'explicit'.
\end{verbatim}
Default: \verb|numerical|
\\Hidden: \verb|False|
\\Experimental: \verb|False|
\\\item[\colorbox{gray!30}{\texttt{extensions}}] (\textit{comma separated list})
\begin{verbatim}
A list of extension names which should be activated for the
code generation.

ninja        --- Use the ninja reduction library (default).
golem95      --- Use the golem95 reduction library (only when
                 enabled during setup step of GoSam installation).
dred         --- Use DRED as IR regularisation scheme (default).
thv          --- Use tHV as IR regularisation scheme.
numpolvec    --- Evaluate polarisation vectors numerically (default).
quadruple    --- Make a quadruple precision copy of the code (this
                 works only with ninja).
customspin2prop --- replace the propagator of spin-2 particles
                    with a custom function (read the manual for this).
gaugecheck   --- modify gauge boson wave functions to allow for
                 a limited gauge check (introduces gauge*z variables)
generate-all-helicities --- Do not use symmetries to relate helicity
                            configurations and produce separate code
                            for each configuration instead.

OLP interface only:

olp_daemon   --- Generates a C-program providing network access to
                 the amplitude.
olp_badpts   --- Allows to stear the numbering of the files containing
                 bad points from the MC.
olp_blha1    --- Use BLHA version 1 instead of version 2.
f77          --- In combination with the BLHA interface it generates
                 an olp_module.f90 linkable with Fortran77
\end{verbatim}
Hidden: \verb|False|
\\Experimental: \verb|False|
\\\item[\colorbox{gray!30}{\texttt{debug}}] (\textit{comma separated list})
\begin{verbatim}
A list of debug flags.
Currently, the words 'lo', 'nlo' and 'all' are supported.
\end{verbatim}
Hidden: \verb|False|
\\Experimental: \verb|False|
\\\item[\colorbox{gray!30}{\texttt{select.lo}}] (\textit{comma separated list})
\begin{verbatim}
A list of integer numbers, indicating leading order diagrams to be
selected. If no list is given, all diagrams are selected.
Otherwise, all diagrams not in the list are discarded.

The list may contain ranges:

select.lo=1,2,5:10:3, 50:53

which is equivalent to

select.lo=1,2,5,8,50,51,52,53

See also: select.nlo, filter.lo, filter.nlo
\end{verbatim}
Hidden: \verb|False|
\\Experimental: \verb|False|
\\\item[\colorbox{gray!30}{\texttt{select.nlo}}] (\textit{comma separated list})
\begin{verbatim}
A list of integer numbers, indicating one-loop diagrams to be selected.
If no list is given, all diagrams are selected.
Otherwise, all diagrams   not in the list are discarded.

The list may contain ranges:

select.nlo=1,2,5:10:3, 50:53

which is equivalent to

select.nlo=1,2,5,8,50,51,52,53

See also: select.lo, filter.lo, filter.nlo
\end{verbatim}
Hidden: \verb|False|
\\Experimental: \verb|False|
\\\item[\colorbox{gray!30}{\texttt{select.ct}}] (\textit{comma separated list})
\begin{verbatim}
A list of integer numbers, indicating EFT counterterm diagrams to be
selected. If no list is given, all diagrams are selected.
Otherwise, all diagrams not in the list are discarded.

The list may contain ranges:

select.ct=1,2,5:10:3, 50:53

which is equivalent to

select.ct=1,2,5,8,50,51,52,53

See also: select.nlo, filter.lo, filter.nlo
\end{verbatim}
Hidden: \verb|False|
\\Experimental: \verb|False|
\\\item[\colorbox{gray!30}{\texttt{filter.lo}}] (\textit{text})
\begin{verbatim}
A python function which provides a filter for tree diagrams.

filter.lo=lambda d: d.iprop(Z) == 1 \
   and d.vertices(Z, U, Ubar) == 0

The following methods of the diagram class can be used:

* d.rank() = the maximum rank in Q possible for this diagram
* d.loopsize() = the number of propagators in the loop
* d.vertices(field1, field2, ...) = number of vertices
    with the given fields
* d.loopvertices(field1, field2, ...) = number of vertices
    with the given fields; only those vertices which have
    at least one loop propagator attached to them
* d.iprop(field, momentum="...", twospin=..., massive=True/False,
                                                         color=...) =
    the number of propagators with the given properties:
     - field: a field or list of fields
     - momentum: a string denoting the momentum through this propagator,
            such as "k1+k2"
     - twospin: two times the spin (integer number)
     - massive: select only propagators with/without a non-zero mass
     - color: one of the numbers 1, 3, -3 or 8, or a list of
              these numbers
* d.legs(...) = number of legs
   same as iprop, but for external legs
* d.iprop_momentum(field, momentum="...") = True when the diagram contains
   a propagator of field with the specified momentum, False otherwise
* d.chord(...) = number of loop propagators with the given properties;
    the arguments are the same as in iprop
* d.bridge(...) = number of non-loop propagators with the given
    properties; the arguments are the same as in iprop
* d.order(order) = total power of diagrams with respect to specified
   coupling order. Only works whit UFO models. The order must be defined
   in the UFO model's coupling_orders.py and listed in the 'order_names'
   property of the GoSam config/runcard.

Note: Using d.iprop(field, momentum="...") in olp-mode can lead to
      inconsistencies in the automatically generated crossings. This
      can be circumvented by running GoSam with the option --no-crossings
      or using the iprop_momentum function, which tracks invalid crossings.

See also: filter.nlo, select.lo, select.nlo
\end{verbatim}
Hidden: \verb|False|
\\Experimental: \verb|False|
\\\item[\colorbox{gray!30}{\texttt{filter.nlo}}] (\textit{text})
\begin{verbatim}
A python function which provides a filter for loop diagrams.

See filter.lo for more explanation.
\end{verbatim}
Hidden: \verb|False|
\\Experimental: \verb|False|
\\\item[\colorbox{gray!30}{\texttt{filter.ct}}] (\textit{text})
\begin{verbatim}
A python function which provides a filter for eft counterterm diagrams.

See filter.lo for more explanation.
\end{verbatim}
Hidden: \verb|False|
\\Experimental: \verb|False|
\\\item[\colorbox{gray!30}{\texttt{filter.module}}] (\textit{text})
\begin{verbatim}
A python file of predefined functions which should be available
in filters.

Example:

filter.module=filter.py
filter.nlo=my_nlo_filter("vertices.txt")
filter.lo=my_nlo_filter("vertices.txt")

------ filter.py -----

class my_nlo_filter_class:
   def __init__(self, fname):
      self.fields = []
      with open(fname, 'r') as f:
         for line in f.readlines():
            fields = map(lambda s: s.strip(),
                  line.split(","))
            self.fields.append(fields)

   def __call__(self, diag):
      for lst in self.fields:
         if diag.vertices(*lst) > 0:
            return False
      return True

----------------------

See filter.lo, filter.nlo
\end{verbatim}
Hidden: \verb|False|
\\Experimental: \verb|False|
\\\item[\colorbox{gray!30}{\texttt{renorm\_beta}}] (\textit{true/false})
\begin{verbatim}
Sets the name of the same variable in config.f90

Activates or disables beta function renormalisation

QCD only
\end{verbatim}
Default: \verb|True|
\\Hidden: \verb|False|
\\Experimental: \verb|False|
\\\item[\colorbox{gray!30}{\texttt{renorm\_mqwf}}] (\textit{true/false})
\begin{verbatim}
Sets the name of the same variable in config.f90

Activates or disables UV countertems coming from
external massive quarks

QCD only
\end{verbatim}
Default: \verb|True|
\\Hidden: \verb|False|
\\Experimental: \verb|False|
\\\item[\colorbox{gray!30}{\texttt{renorm\_decoupling}}] (\textit{true/false})
\begin{verbatim}
Sets the name of the same variable in config.f90

Activates or disables UV counterterms coming from
massive quark loops

QCD only
\end{verbatim}
Default: \verb|True|
\\Hidden: \verb|False|
\\Experimental: \verb|False|
\\\item[\colorbox{gray!30}{\texttt{renorm\_mqse}}] (\textit{true/false})
\begin{verbatim}
Sets the name of the same variable in config.f90

Activates or disables the UV counterterm coming from the
massive quark propagators

QCD only
\end{verbatim}
Default: \verb|True|
\\Hidden: \verb|False|
\\Experimental: \verb|False|
\\\item[\colorbox{gray!30}{\texttt{renorm\_logs}}] (\textit{true/false})
\begin{verbatim}
Sets the name of the same variable in config.f90

Activates or disables the logarithmic finite terms
of all UV counterterms

QCD only
\end{verbatim}
Default: \verb|True|
\\Hidden: \verb|False|
\\Experimental: \verb|False|
\\\item[\colorbox{gray!30}{\texttt{renorm\_gamma5}}] (\textit{true/false})
\begin{verbatim}
Sets the same variable in config.f90

Activates finite renormalisation for axial couplings in the
't Hooft-Veltman scheme

QCD only, works only with built-in model files.
\end{verbatim}
Default: \verb|True|
\\Hidden: \verb|False|
\\Experimental: \verb|False|
\\\item[\colorbox{gray!30}{\texttt{renorm\_yukawa}}] (\textit{true/false})
\begin{verbatim}
Sets the same variable in config.f90

Enables renormalization of Yukawa coupling. Two schemes are
possible: On-Shell and MSbar.

QCD only.

See also: MSbar_yukawa
\end{verbatim}
Default: \verb|True|
\\Hidden: \verb|False|
\\Experimental: \verb|False|
\\\item[\colorbox{gray!30}{\texttt{renorm\_eftwilson}}] (\textit{true/false})
\begin{verbatim}
Sets the same variable in config.f90

Enables renormalization of EFT Wilson coefficients.
Works only with special New Physics UFO models, con-
taining 'NP' as additional coupling order. 'order_names'
must be specified and explicitly contain 'NP'.
\end{verbatim}
Default: \verb|False|
\\Hidden: \verb|False|
\\Experimental: \verb|False|
\\\item[\colorbox{gray!30}{\texttt{renorm\_ehc}}] (\textit{true/false})
\begin{verbatim}
Sets the same variable in config.f90

Turns on the finite renormalisation of effective Higgs-gluon
vertices. Implemented for models in the heavy-top limit like
smehc. Should not be used when counterterms for Wilson coeffi-
cients are supplyed by means of a UFO model (see 'renorm_eft_wilson').
CAUTION:
This will only work if the Higgs-gluon vertices factorize from the
amplitude, i.e. the number of Higgs-gluon couplings is the same for
all Born diagrams!
\end{verbatim}
Default: \verb|False|
\\Hidden: \verb|False|
\\Experimental: \verb|False|
\\\item[\colorbox{gray!30}{\texttt{reduction\_interoperation}}] (\textit{text})
\begin{verbatim}

   Default reduction library.

   Possible values are: ninja, golem95

   Sets the same variable in config.f90. A value of '-1' lets GoSam decide
   depending on reduction_libraries

   See common/config.f90 for details.
\end{verbatim}
Default: \verb|-1|
\\Hidden: \verb|False|
\\Experimental: \verb|False|
\\\item[\colorbox{gray!30}{\texttt{reduction\_interoperation\_rescue}}] (\textit{text})
\begin{verbatim}

   Rescue reduction program.

   Sets the same variable in config.f90. A value of '-1' lets GoSam
   decide.

   See common/config.f90 for details.
\end{verbatim}
Default: \verb|-1|
\\Hidden: \verb|False|
\\Experimental: \verb|False|
\\\item[\colorbox{gray!30}{\texttt{nlo\_prefactors}}] (\textit{integer number})
\begin{verbatim}

   Sets the same variable in config.f90. The values have the
   following meaning:

   0: A factor of alpha_(s)/2/pi is not included in the NLO result
   1: A factor of 1/8/pi^2 is not included in the NLO result
   2: The NLO includes all prefactors

   Note, however, that the factor of 1/Gamma(1-eps) is not
   included in any of the cases.

   Please note, that nlo_prefactors=0 is enforced temporary in test.f90
   for better testing. In OLP interface mode (BLHA/BLHA2), the default is
   nlo_prefactors=2.
\end{verbatim}
Default: \verb|0|
\\Hidden: \verb|False|
\\Experimental: \verb|False|
\\\item[\colorbox{gray!30}{\texttt{PSP\_check}}] (\textit{true/false})
\begin{verbatim}
Sets the same variable in config.f90

Activates Phase-Space Point test for the full amplitude.

!!Works only for QCD and with built-in model files!!
\end{verbatim}
Default: \verb|True|
\\Hidden: \verb|False|
\\Experimental: \verb|False|
\\\item[\colorbox{gray!30}{\texttt{PSP\_rescue}}] (\textit{true/false})
\begin{verbatim}
Sets the same variable in config.f90

Activates Phase-Space Point rescue based on the estimated
accuracy on the finite part. It needs PSP_check=True.
The accuracy is estimated using information on the single
pole accuracy and on the stability of the finite part
under rotation of the phase space point.

!!Works only for QCD and with built-in model files!!
\end{verbatim}
Default: \verb|False|
\\Hidden: \verb|False|
\\Experimental: \verb|False|
\\\item[\colorbox{gray!30}{\texttt{PSP\_verbosity}}] (\textit{true/false})
\begin{verbatim}
Sets the same variable in config.f90

Sets the verbosity of the PSP_check.
verbosity = False: no output
verbosity = True : bad point are written into gs_badpts.log
!!Works only for QCD and with built-in model files!!
\end{verbatim}
Default: \verb|False|
\\Hidden: \verb|False|
\\Experimental: \verb|False|
\\\item[\colorbox{gray!30}{\texttt{PSP\_chk\_th1}}] (\textit{integer number})
\begin{verbatim}
Sets the same variable in config.f90

Threshold to accept a PSP point without further treatment,
based on the precision of the single pole. The number has to be
an integer indicating the desired minimum number of digits
accuracy on the single pole. For poles more precise than this
threshold the finite part is not checked.
!!Works only for QCD and with built-in model files!!

The number has to be an integer.
\end{verbatim}
Default: \verb|8|
\\Hidden: \verb|False|
\\Experimental: \verb|False|
\\\item[\colorbox{gray!30}{\texttt{PSP\_chk\_th2}}] (\textit{integer number})
\begin{verbatim}
Sets the same variable in config.f90

Threshold to declare a PSP as a bad point, based of the precision
of the single pole. Points with precision less than this
threshold are directly reprocessed with the rescue system (if
available), or declared as unstable. According to the verbosity
level set, such points are written to a file and not used when
the code is interfaced to an external Monte Carlo using the new
BLHA standards.
!!Works only for QCD and with built-in model files!!

The number has to be an integer.
\end{verbatim}
Default: \verb|3|
\\Hidden: \verb|False|
\\Experimental: \verb|False|
\\\item[\colorbox{gray!30}{\texttt{PSP\_chk\_th3}}] (\textit{integer number})
\begin{verbatim}
Sets the same variable in config.f90

Threshold to declare a PSP as a bad point, based on the precision
of the finite part estimated with a rotation. Points with
precision less than this threshold are directly reprocessed
with the rescue system (if available), or declared as unstable.
According to the verbosity level set, such points are written
to a file and not used when the code is interfaced to an
external Monte Carlo using the new BLHA standards.
!!Works only for QCD and with built-in model files!!

The number has to be an integer.
\end{verbatim}
Default: \verb|5|
\\Hidden: \verb|False|
\\Experimental: \verb|False|
\\\item[\colorbox{gray!30}{\texttt{PSP\_chk\_th4}}] (\textit{integer number})
\begin{verbatim}
Sets the same variable in config.f90

Threshold to accept a PSP point without further treatment,
based on the precision of the finite part estimated by
comparing the normal and rotated double precision
evaluations against a quadruple precision evaluation.
!!Works only for QCD and with built-in model files!!
!!Used only for: extensions=quadruple!!

The number has to be an integer.
\end{verbatim}
Default: \verb|10|
\\Hidden: \verb|False|
\\Experimental: \verb|False|
\\\item[\colorbox{gray!30}{\texttt{PSP\_chk\_th5}}] (\textit{integer number})
\begin{verbatim}
Sets the same variable in config.f90

Threshold to declare a quadruple precision PSP as a bad point,
based on the precision of the finite part estimated by
comparing the normal and rotated quadruple precision
evaluations. According to the verbosity level set, such points
are written to a file and not used when the code is interfaced
to an external Monte Carlo using the new BLHA standards.
!!Works only for QCD and with built-in model files!!
!!Used only for: extensions=quadruple!!

The number has to be an integer.
\end{verbatim}
Default: \verb|7|
\\Hidden: \verb|False|
\\Experimental: \verb|False|
\\\item[\colorbox{gray!30}{\texttt{PSP\_chk\_kfactor}}] (\textit{text})
\begin{verbatim}
Sets the same variable in config.f90

Threshold on the k-factor to perform a rotation check on the PSP.
!!Works only for QCD and with built-in model files!!
\end{verbatim}
Default: \verb|1000|
\\Hidden: \verb|False|
\\Experimental: \verb|False|
\\\item[\colorbox{gray!30}{\texttt{PSP\_chk\_li1}}] (\textit{integer number})
\begin{verbatim}
Sets the same variable in config.f90. For loop-induced
processes, it is used instead of PSP_chk_th1.

Threshold to accept a PSP point without further treatment,
based on the precision of the single pole (which should be
zero). The number has to be an integer indicating the desired
minimum number of digits  accuracy on the single pole. For
poles more precise than this  threshold the finite part is
not checked.
!!Works only for QCD and with built-in model files!!

The number has to be an integer.
\end{verbatim}
Default: \verb|8|
\\Hidden: \verb|False|
\\Experimental: \verb|False|
\\\item[\colorbox{gray!30}{\texttt{PSP\_chk\_li2}}] (\textit{integer number})
\begin{verbatim}
Sets the same variable in config.f90. For loop-induced
processes, it is used instead of PSP_chk_th1.

Threshold to declare a PSP as a bad point, based of the precision
of the single pole. Points with precision less than this
threshold are directly reprocessed with the rescue system (if
available), or declared as unstable. According to the verbosity
level set, such points are written to a file and not used when
the code is interfaced to an external Monte Carlo using the new
BLHA standards.
!!Works only for QCD and with built-in model files!!

The number has to be an integer.
\end{verbatim}
Default: \verb|3|
\\Hidden: \verb|False|
\\Experimental: \verb|False|
\\\item[\colorbox{gray!30}{\texttt{PSP\_chk\_li3}}] (\textit{integer number})
\begin{verbatim}
Sets the same variable in config.f90. For loop-induced
processes, it is used instead of PSP_chk_th1.

Threshold to declare a PSP as a bad point, based on the precision
of the finite part estimated with a rotation. Points with
precision less than this threshold are directly reprocessed
with the rescue system (if available), or declared as unstable.
According to the verbosity level set, such points are written
to a file and not used when the code is interfaced to an
external Monte Carlo using the new BLHA standards.
!!Works only for QCD and with built-in model files!!

The number has to be an integer.
\end{verbatim}
Default: \verb|5|
\\Hidden: \verb|False|
\\Experimental: \verb|False|
\\\item[\colorbox{gray!30}{\texttt{PSP\_chk\_li4}}] (\textit{integer number})
\begin{verbatim}
Sets the same variable in config.f90. For loop-induced
processes, it is used instead of PSP_chk_th1.

Threshold to accept a PSP point without further treatment,
based on the precision of the finite part estimated by
comparing the normal and rotated double precision
evaluations against a quadruple precision evaluation.
!!Works only for QCD and with built-in model files!!
!!Used only for: extensions=quadruple!!

The number has to be an integer.
\end{verbatim}
Default: \verb|10|
\\Hidden: \verb|False|
\\Experimental: \verb|False|
\\\item[\colorbox{gray!30}{\texttt{PSP\_chk\_li5}}] (\textit{integer number})
\begin{verbatim}
Sets the same variable in config.f90. For loop-induced
processes, it is used instead of PSP_chk_th1.

Threshold to declare a quadruple precision PSP as a bad point,
based on the precision of the finite part estimated by
comparing the normal and rotated quadruple precision
evaluations. According to the verbosity level set, such points
are written to a file and not used when the code is interfaced
to an external Monte Carlo using the new BLHA standards.
!!Works only for QCD and with built-in model files!!
!!Used only for: extensions=quadruple!!

The number has to be an integer.
\end{verbatim}
Default: \verb|7|
\\Hidden: \verb|False|
\\Experimental: \verb|False|
\\\item[\colorbox{gray!30}{\texttt{PSP\_chk\_method}}] (\textit{text})
\begin{verbatim}
This option can be used to overwrite the automatic phase-space point
test method enabled with PSP_check=True.
Except in some BSM scenarios, the user does not need to change this.

Possible options:
Automatic    - chooses automatically a suitable phase-space point test
               method (default).
PoleRotation - check first the pole and then rotate if necessary.
Rotation     - force a rotation check on every phase space point.
LoopInduced  - check that the pole part is zero and rotate if necessary.
               Needed e.g. for interferience between BSM Born and
               SM loop-induced virtual.
\end{verbatim}
Default: \verb|Automatic|
\\Hidden: \verb|False|
\\Experimental: \verb|False|
\\\item[\colorbox{gray!30}{\texttt{reference-vectors}}] (\textit{comma separated list})
\begin{verbatim}
A list of reference vectors for massive and higher spin particles.
For vectors which are not assigned here, the program picks a
reference vector automatically.

Each entry of the list has to be of the form <index>:<index>

EXAMPLE

in=g,u
out=t,W+
reference-vectors=1:2,3:4,4:3

In this example, the gluon (particle 1) takes the momentum k2
as reference momentum for the polarisation vector. The massive
top quark (particle 3) uses the light-cone projection l4 of the
W-boson as reference direction for its own momentum splitting.
Similarly, the momentum of the W-boson is split into a direction
l4 and one along l3.

If cycles are generated in the list (l3 has to be known in order
to determine l4 and vice versa in the above example) they must be
at most of length two. For the reference momenta of lightlike
gauge bosons the length of cycles does not matter, e.g.

in=g,g
out=g,g
reference-vectors=1:2,2:3,3:4,4:1
\end{verbatim}
Hidden: \verb|False|
\\Experimental: \verb|False|
\\\item[\colorbox{gray!30}{\texttt{abbrev.color}}] (\textit{text})
\begin{verbatim}
The program in use for the generation of color related abbreviations.
The value should be one of:
   form           color algebra and optimization in form
   none           color algebra in form, no optimization
\end{verbatim}
Default: \verb|form|
\\Hidden: \verb|True|
\\Experimental: \verb|False|
\\\item[\colorbox{gray!30}{\texttt{abbrev.limit}}] (\textit{text})
\begin{verbatim}
Maximum number of instructions per subroutine when calculating
abbreviations, if this number is positive.
\end{verbatim}
Default: \verb|500|
\\Hidden: \verb|False|
\\Experimental: \verb|False|
\\\item[\colorbox{gray!30}{\texttt{templates}}] (\textit{text})
\begin{verbatim}
Path pointing to the directory containing the template
files for the process. If not set, GoSam uses the directory
<gosam_git_path>/templates, where <gosam_git_path> is the
path into which the GoSam git has been cloned.

The directory must contain a file called 'template.xml'
\end{verbatim}
Hidden: \verb|False|
\\Experimental: \verb|False|
\\\item[\colorbox{gray!30}{\texttt{qgraf.bin}}] (\textit{text})
\begin{verbatim}
Points to the QGraf executable.

Example:
qgraf.bin=/home/my_user_name/bin/qgraf
\end{verbatim}
Default: \verb|/home/marius/Programs/GoSam3/SMEFT/bin/GoSam/qgraf|
\\Hidden: \verb|False|
\\Experimental: \verb|False|
\\\item[\colorbox{gray!30}{\texttt{form.bin}}] (\textit{text})
\begin{verbatim}
Points to the Form executable.

Examples:
1) # Use TForm:
   form.bin=tform
2) # Use non-standard location:
   form.bin=/home/my_user_name/bin/form
\end{verbatim}
Default: \verb|/home/marius/Programs/GoSam3/SMEFT/bin/GoSam/tform|
\\Hidden: \verb|False|
\\Experimental: \verb|False|
\\\item[\colorbox{gray!30}{\texttt{form.threads}}] (\textit{integer number})
\begin{verbatim}
Number of Form threads.

Example:
form.threads=4
runs tform, the parallel version of FORM, on 4 cores.
\end{verbatim}
Default: \verb|2|
\\Hidden: \verb|False|
\\Experimental: \verb|False|
\\\item[\colorbox{gray!30}{\texttt{form.tempdir}}] (\textit{text})
\begin{verbatim}
Temporary directory for Form. Should point to a directory
on a local disk.

Examples:
form.tempdir=/tmp
form.tempdir=/scratch
\end{verbatim}
Default: \verb|/tmp|
\\Hidden: \verb|False|
\\Experimental: \verb|False|
\\\item[\colorbox{gray!30}{\texttt{form.workspace}}] (\textit{integer number})
\begin{verbatim}
Size of the heap (in megabytes) used by FORM.

Example (for machines with <= 2GB RAM):
form.workspace=100
set WorkSpace to 100M in FORM via form.set file.
\end{verbatim}
Default: \verb|1000|
\\Hidden: \verb|False|
\\Experimental: \verb|False|
\\\item[\colorbox{gray!30}{\texttt{r2}}] (\textit{text})
\begin{verbatim}
The algorithm how to treat the R2 term:

implicit    -- mu^2 terms are kept in the numerator and reduced
               at runtime (available only when regularisation
               scheme is DRED)
explicit    -- mu^2 terms are reduced analytically
\end{verbatim}
Default: \verb|explicit|
\\Hidden: \verb|False|
\\Experimental: \verb|False|
\\\item[\colorbox{gray!30}{\texttt{symmetries}}] (\textit{comma separated list})
\begin{verbatim}
Specifies the symmetries of the amplitude.

This information is used when the list of helicities is generated.

Possible values are:

* flavour    -- no flavour changing interactions
         When calculating the list of helicities, fermion lines
    of PDGs 1-6 are assumed not to mix.

* family     -- flavour changing only within families
         When calculating the list of helicities, fermion lines
    of PDGs 1-6 are assumed to mix only within families,
    i.e. a quark line connecting a up with a down quark would
    be considered, while up-bottom is not.
* lepton     -- means for leptons what 'flavour' means for quarks
* generation -- means for leptons what 'family' means for quarks
* parity     -- the amplitude is invariant under parity tranformation.
                === Parity is not implemented yet.
* <n>=<h>    -- restriction of particle helicities,
         e.g. 1=-, 2=+ specifies helicities of particles 1 and 2
* %<n>=<h>   -- restriction by PDG code,
         e.g. %23=+- specifies the helicity of all Z-bosons to be
         '+' and '-' only (no '0' polarisation).

         %<n> refers to both +n and -n
         %+<n> refers to +n only
         %-<n> refers to -n only
\end{verbatim}
Hidden: \verb|False|
\\Experimental: \verb|False|
\\\item[\colorbox{gray!30}{\texttt{crossings}}] (\textit{comma separated list})
\begin{verbatim}
A list of crossed processes derived from this process.

For each process in the list a module similar to matrix.f90 is
generated.

Example:

process_name=ddx_uux
in=1,-1
out=2,-2

crossings=dxd_uux: -1 1 > 2 -2, ud_ud: 2 1 > 2 1
\end{verbatim}
Hidden: \verb|False|
\\Experimental: \verb|False|
\\\item[\colorbox{gray!30}{\texttt{formopt.level}}] (\textit{text})
\begin{verbatim}
There are three levels of Horner Scheme
offered and the number here will specify
which one we use. It can only be 1,2 or 3.

Examples:
formopt.level=3
\end{verbatim}
Default: \verb|2|
\\Hidden: \verb|False|
\\Experimental: \verb|False|
\\\item[\colorbox{gray!30}{\texttt{pyxodraw}}] (\textit{true/false})
\begin{verbatim}
Specifies whether to draw any diagrams or not.
\end{verbatim}
Default: \verb|True|
\\Hidden: \verb|True|
\\Experimental: \verb|False|
\\\item[\colorbox{gray!30}{\texttt{form\_factor\_lo}}] (\textit{text})
\begin{verbatim}
This option allows to define a form factor which LO results are
multiplied with.
Example:
form_factor_lo="(1000._ki**2/
     (1000._ki**2+dotproduct(vecs(2,:)+vecs(3,:),vecs(2,:)+vecs(3,:)))
)"
\end{verbatim}
Hidden: \verb|False|
\\Experimental: \verb|True|
\\\item[\colorbox{gray!30}{\texttt{form\_factor\_nlo}}] (\textit{text})
\begin{verbatim}
This option allows to define a form factor which NLO/loop-induced results
are multiplied with.
\end{verbatim}
Hidden: \verb|False|
\\Experimental: \verb|True|
\\\item[\colorbox{gray!30}{\texttt{order\_names}}] (\textit{comma separated list})
\begin{verbatim}
Only works in combination with UFO models.
A list of additional coupling orders as defined in the model's
coupling_orders.py file that should be tracked throughout the
amplitude generation. Relevant for correct EFT treatment.

Example:
order_names=QCD,NP,QL
\end{verbatim}
Hidden: \verb|False|
\\Experimental: \verb|False|
\\\item[\colorbox{gray!30}{\texttt{enable\_truncation\_orders}}] (\textit{true/false})
\begin{verbatim}
Whether or not to generate extra code to assess different
truncation orders in EFT calculations. Only works with a
New Physics UFO model containing 'NP' as additional coupling
order. 'order_names' must be specified and explicitly contain
'NP'. When set to False (default) no truncation is performed,
i.e. the amplitude is squared in the naive way.
\end{verbatim}
Default: \verb|False|
\\Hidden: \verb|False|
\\Experimental: \verb|False|
\\\item[\colorbox{gray!30}{\texttt{use\_vertex\_labels}}] (\textit{true/false})
\begin{verbatim}
Whether or not to print the vertex label in the process.pdf. Can only
be activated when using a UFO model file.
\end{verbatim}
Default: \verb|False|
\\Hidden: \verb|False|
\\Experimental: \verb|False|
\\\item[\colorbox{gray!30}{\texttt{all\_mandelstam}}] (\textit{true/false})
\begin{verbatim}
If 'false' momentum conservation is used to reduce the set of independent
Mandelstam invariants in the construction of the amplitude. This option
can cause problems with numeric stability of the real radiation ampli-
tudes when the accuracy of the phase space point is bad, i.e. momentum
conservation is fulfilled to significantly less than double precision.
In that case it is better to use the full set of Mandelstam invariants,
without using momentum conservation relations.
\end{verbatim}
Default: \verb|False|
\\Hidden: \verb|False|
\\Experimental: \verb|False|
\\\item[\colorbox{gray!30}{\texttt{flavour\_groups}}] (\textit{comma separated list})
\begin{verbatim}
Defines which flavours should be considered equivalent, i.e. grouped
together during channel generation in olp mode. Only relevant if crossings
are generated. Uses pdg codes.

Examples:

flavour_groups=1:2:3:4:5
-> completely flavour blind process with five light quark-flavours

flavour_groups=1:3:5,2:4
-> distinguish up- and down-type quark-flavours

flavour_groups=
-> each flavour treated separately (default)
\end{verbatim}
Hidden: \verb|False|
\\Experimental: \verb|False|
\\\item[\colorbox{gray!30}{\texttt{respect\_generations}}] (\textit{true/false})
\begin{verbatim}
Boolean determining whether or not the quark generation should be taken
into account when the flavour_groups feature is used to find crossing
relations among olp channels. Is relevant if flavour changing vertices
appear (Assuming diagonal CKM!).

Examples:

respect_generations=False and flavour_groups=1:3:5,2:4
-> (c cb to d db) and (u ub to s sb) are crossings of (u ub to d db)

respect_generations=True and flavour_groups=1:3:5,2:4
-> (c cb to d db) is a crossing of (u ub to s sb) but not of (u ub to d db)

Default is respect_generations=False.
\end{verbatim}
Default: \verb|False|
\\Hidden: \verb|False|
\\Experimental: \verb|False|
\\\item[\colorbox{gray!30}{\texttt{MSbar\_yukawa}}] (\textit{comma separated list})
\begin{verbatim}
List of quarks with Yukawa couplings which shall be renormalised in
the MSbar scheme instead of the default OS scheme. Can also be used
to renormalise Yukawa couplings of particles with mass set to zero
while still keeping their coupling to the Higgs.

Examples:
MSbar_yukawa=B
-> Yukawa coupling of bottom to Higgs will be renormalised in the
   MSbar scheme, even if mB=0 (as long as Hbb coupling still exist)

See also: renorm_yukawa
\end{verbatim}
Hidden: \verb|False|
\\Experimental: \verb|False|
\\\item[\colorbox{gray!30}{\texttt{use\_MQSE}}] (\textit{true/false})
\begin{verbatim}
Whether or not to scan 1-loop amplitudes for massive quark self
energies and insert the appropriate mass counterterm during the
form step. Used mainly for debugging purposes.
\end{verbatim}
Default: \verb|False|
\\Hidden: \verb|False|
\\Experimental: \verb|True|
\\\item[\colorbox{gray!30}{\texttt{meson.buildtype}}] (\textit{text})
\begin{verbatim}
Build-type passed to meson as the '-Dbuildtype=<buildtype>' option.
The respective buildtypes represent:
                        Debug Symbols        Optimization level
   plain                false                plain
   debug                true                 0
   debugoptimized       true                 2
   release              false                3
   minsize              true                 s
\end{verbatim}
Default: \verb|release|
\\Hidden: \verb|False|
\\Experimental: \verb|False|
\\\item[\colorbox{gray!30}{\texttt{meson.arch}}] (\textit{text})
\begin{verbatim}
CPU architecture passed to the compiler as the '-march=<arch>' option.
By default, GCC generates code for a generic x86-64 CPU. When using the
'native' option, GCC uses all possible instructions available on the
currenly used CPU. This can result in faster executing code, but may make
the libraries / executables unusable on other CPUs. For all possible
options, see the GCC documentation.
\end{verbatim}
Default: \verb|x86-64|
\\Hidden: \verb|False|
\\Experimental: \verb|False|
\\\item[\colorbox{gray!30}{\texttt{unitary\_gauge}}] (\textit{true/false})
\begin{verbatim}
Use unitary gauge propagators for the massive vector bosons instead of
Feynman gauge propagators.
\end{verbatim}
Default: \verb|False|
\\Hidden: \verb|False|
\\Experimental: \verb|False|
\\\end{basedescript}
