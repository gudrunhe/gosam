\begin{description}
\item[\texttt{process\_name}] (\textit{text})
\begin{verbatim}
A symbolic name for this process. This name will be used
as a prefix for the Fortran modules.

Golem will insert an underscore after this prefix.
If the process name is left blank no prefix will be used
and no extra underscore will be generated.
\end{verbatim}
\item[\texttt{process\_path}] (\textit{text})
\begin{verbatim}
The path to which all Form output is written.
If no absolute path is given, the path is interpreted relative
to the working directory from which gosam.py is run.

Example:
process_path=/scratch/golem_processes/process1
\end{verbatim}
\item[\texttt{in}] (\textit{comma separated list})
\begin{verbatim}
A comma-separated list of initial state particles.
Which particle names are valid depends on the
model file in use.

Examples (Standard Model):
1) in=u,u~
2) in=e+,e-
3) in=g,g
\end{verbatim}
\item[\texttt{out}] (\textit{comma separated list})
\begin{verbatim}
A comma-separated list of final state particles.
Which particle names are valid depends on the
model file in use.

Examples (Standard Model):
1) out=H,u,u~
2) out=e+,e-,gamma
3) out=b,b~,t,t~
\end{verbatim}
\item[\texttt{model}] (\textit{comma separated list})
\begin{verbatim}
This option allows the selection of a model for the
Feynman rules. It has to conform with one of four possible
formats:

1) model=<name>
2) model=<path>, <name>
3) model=<path>, <number>
4) model=FeynRules, <path>

Format 1) searches for the model files <name>, <name>.hh
and <name>.py in the models/ directory under the installation
path of Golem.

Format 2) is similar to format 1) but <path> is used instead
of the models/ directory of the Golem installation

Format 3) expects the files func<number>.mdl, lgrng<number>.mdl,
prtcls<number>.mdl and vars<number>.mdl in the directory <path>.
These files need to be in CalcHEP/CompHEP format.

Format 4) expects files according to the new FeynRules Python
interface in the directory specified by <path>.
(Not fully implemented yet)
\end{verbatim}
Default: \verb|smdiag|
\item[\texttt{model.options}] (\textit{comma separated list})
\begin{verbatim}
If the model in use supports options they can be passed via this
property.

For builtin models, the option "ewchoose"
selects automatically the EW scheme based.
\end{verbatim}
Default: \verb|ewchoose|
\item[\texttt{order}] (\textit{comma separated list})
\begin{verbatim}
A 3-tuple <coupling>,<born>,<virt> where <coupling> denotes
a function of the qgraf style file which can be used as
an argument in a 'vsum' statement. For the standard model
file 'sm' there are two such functions, 'gs' which counts
powers of the strong coupling and 'gw' which counts powers
of the weak coupling. <born> is the sum of powers for the
tree level amplitude and <virt> for the virtual amplitude.
The line
   order = gs, 4, 6
would select all diagrams which have (gs)^4 at tree level
and all loop graphs with (gs)^6.

Note: The line
   order = gw, 2, 2
does not imply that no virtual corrections are calculated.
Instead, for the virtual corrections diagrams are chosen
with the same order in gw but higher order in gs.

In other models with more than two different coupling
constants additional 'vsum' statements, which can be passed
via the qgraph.verbatim option, might be needed
to select the correct set of diagrams.

If the last number is omitted no virtual corrections are
calculated.

See also: qgraf.options, qgraf.verbatim
\end{verbatim}
\item[\texttt{zero}] (\textit{comma separated list})
\begin{verbatim}
A list of symbols that should be treated as identically
zero throughout the whole calculation. All of these
symbols must be defined by the model file.

Examples:
1) # Light masses are set to zero here:
   zero=me,mU,mD,mS
2) # Diagonal CKM matrix:
   zero=VUS, VUB, CVDC, CVDT, \
        VCD, VCB, CVSU, CVST, \
        VTD, VTS, CVBU, CVBC
   one=  VUD,  VCS,  VTB, \
        CVDU, CVSC, CVBT

See also: model, one
\end{verbatim}
\item[\texttt{one}] (\textit{comma separated list})
\begin{verbatim}
A list of symbols that should be treated as identically
one throughout the whole calculation. All of these
symbols must be defined by the model file.

Example:
one=gs, e

See also: model, zero
\end{verbatim}
\item[\texttt{regularisation\_scheme}] (\textit{text})
\begin{verbatim}
Sets the used regularisation scheme.
Possible values: dred (recommended), cdr
\end{verbatim}
Default: \verb|dred|
\item[\texttt{helicities}] (\textit{comma separated list})
\begin{verbatim}
A list of helicities to be calculated. An empty list
means that all possible helicities should be generated.

The helicities are specified as a string of characters
according to the following table:

   spin massive  |  'm'  '-'  '0'   '+'   'k'
     0   YES/NO  | ---- ----    0  ----  ----
   1/2   YES/NO  | ---- -1/2 ----  +1/2  ----
     1     NO    | ----   -1 ----    +1  ----
     1    YES    | ----   -1    0    +1  ----
   3/2     NO    | -3/2 ---- ----  ----  +3/2
   3/2    YES    | -3/2 -1/2 ----  +1/2  +3/2
     2     NO    |   -2 ---- ----  ----    +2
     2    YES    |   -2   -1    0    +1    +2

Please, note that 'k' and 'm' are not in use yet but reserved
for future extensions to higher spins.

The characters correspond to particle 1, 2, ... from left to
right.

Examples:
   # e+, e- --> gamma, gamma:
   # Only three helicities required; the other ones are
   # either zero or can be obtained by symmetry
   # transformations.
   helicities=+-++,+-+-,+---;

Multiple helicities can be encoded in patterns, which are expanded
at the time of code generation. Patterns can have one of the following
forms:
   [+-], [+-0], [+0] etc. : the bracket expands to one of the symbols
         in the bracket at a time.
   EXAMPLE
         helicities=[+-]+[+-0]
         # expands to 6 different helicities:
         # helicities=+++, ++-, ++0, -++, -+-, -+0
   [a=+-], etc. : as above, but the helicity is also assigned to the
         symbol and can be reused.
   EXAMPLE
         helicities=[i=+-]+i+
         # expands to two helicities
         # helicities=++++, -+-+
   [ab=+-0], etc. : as above, the first symbol is assigned the helicity,
         the second is minus the helicity
   EXAMPLE
         helicities=[qQ=+-][pP=+-]PQ[+-0]
         # expands to 12 helicities
         # helicities=++--+,++---,++--0,+-+-+,+-+--,+-+-0,\
         #            -+-++,-+-+-,-+-+0,--+++,--++-,--++0
\end{verbatim}
\item[\texttt{qgraf.options}] (\textit{comma separated list})
\begin{verbatim}
A list of options which is passed to qgraf via the 'options' line.
Possible values (as of qgraf.3.1.1) are zero, one or more of:
   onepi, onshell, nosigma, nosnail, notadpole, floop
   topol

Please, refer to the QGraf documentation for details.
\end{verbatim}
Default: \verb|onshell,notadpole,nosnail|
\item[\texttt{qgraf.verbatim}] (\textit{text})
\begin{verbatim}
This option allows to send verbatim lines to
the file qgraf.dat. This can be useful if the user
wishes to put additional restricitons to the selected diagrams.
This option is mainly inteded for the use of the operators
   rprop, iprop, chord, bridge, psum
Note, that the use of 'vsum' might interfer with the
option qgraf.power.

Example:
qgraf.verbatim=\
   # no top quarks: \n\
   true=iprop[T, 0, 0];\n\
   # at least one Higgs:\n\
   false=iprop[H, 0, 0];\n

Please, refer to the QGraf documentation for details.

See also: qgraf.options, order
\end{verbatim}
\item[\texttt{qgraf.verbatim.lo}] (\textit{text})
\begin{verbatim}
Same as qgraf.verbatim but only applied to LO diagrams.

See also: qgraf.verbatim, qgraf.verbatim.nlo
\end{verbatim}
\item[\texttt{qgraf.verbatim.nlo}] (\textit{text})
\begin{verbatim}
Same as qgraf.verbatim but only applied to NLO diagrams.

See also: qgraf.verbatim, qgraf.verbatim.nlo
\end{verbatim}
\item[\texttt{group}] (\textit{true/false})
\begin{verbatim}
Flag whether or not the tree-level diagrams should be grouped
into a single file.
\end{verbatim}
Default: \verb|True|
\item[\texttt{diagsum}] (\textit{true/false})
\begin{verbatim}
Flag whether or not 1-loop diagrams with the same propagators
should be summed before the algebraic reduction.
\end{verbatim}
Default: \verb|True|
\item[\texttt{reduction\_programs}] (\textit{comma separated list})
\begin{verbatim}
Specifies the reduction libraries which should be supported.

Possible values: ninja, samurai, golem95, pjfry (experimental)

See also reduction_interoperation, reduction_interoperation_rescue.
\end{verbatim}
Default: \verb|ninja,golem95|
\item[\texttt{polvec}] (\textit{text})
\begin{verbatim}
Evaluate the polarisation vector
'numerical' or 'explicit'.
\end{verbatim}
Default: \verb|numerical|
\item[\texttt{extensions}] (\textit{comma separated list})
\begin{verbatim}
A list of extension names which should be activated for the
code generation.

Build system related extensions:

autotools    --- use autotools to generate Makefiles
shared       --- create shared libraries (=dynamically linkable code),
                 enabled by default with autotools extension
f77          --- in combination with the BLHA interface it generates
                 an olp_module.f90 linkable with Fortran77

Other extensions:
noformopt    --- disable diagram optimization using FORM
gaugecheck   --- modify gauge boson wave functions to allow for
                 a limited gauge check (introduces gauge*z variables)
customspin2prop --- replace the propagator of spin-2 particles
                    with a custom function (read the manual for this).
\end{verbatim}
\item[\texttt{debug}] (\textit{comma separated list})
\begin{verbatim}
A list of debug flags.
Currently, the words 'lo', 'nlo' and 'all' are supported.
\end{verbatim}
\item[\texttt{select.lo}] (\textit{comma separated list})
\begin{verbatim}
A list of integer numbers, indicating leading order diagrams to be
selected. If no list is given, all diagrams are selected.
Otherwise, all diagrams not in the list are discarded.

The list may contain ranges:

select.lo=1,2,5:10:3, 50:53

which is equivalent to

select.lo=1,2,5,8,50,51,52,53

See also: select.nlo, filter.lo, filter.nlo
\end{verbatim}
\item[\texttt{select.nlo}] (\textit{comma separated list})
\begin{verbatim}
A list of integer numbers, indicating one-loop diagrams to be selected.
If no list is given, all diagrams are selected.
Otherwise, all diagrams   not in the list are discarded.

The list may contain ranges:

select.nlo=1,2,5:10:3, 50:53

which is equivalent to

select.nlo=1,2,5,8,50,51,52,53

See also: select.lo, filter.lo, filter.nlo
\end{verbatim}
\item[\texttt{select.nnlo}] (\textit{comma separated list})
\begin{verbatim}
A list of integer numbers, indicating two-loop diagrams to be selected.
If no list is given, all diagrams are selected.
Otherwise, all diagrams   not in the list are discarded.

The list may contain ranges:

select.nnlo=1,2,5:10:3, 50:53

which is equivalent to

select.nnlo=1,2,5,8,50,51,52,53

See also: select.lo, filter.lo, filter.nlo
\end{verbatim}
\item[\texttt{filter.lo}] (\textit{text})
\begin{verbatim}
A python function which provides a filter for tree diagrams.

filter.lo=lambda d: d.iprop(Z) == 1 \
   and d.vertices(Z, U, Ubar) == 0

The following methods of the diagram class can be used:

* d.rank() = the maximum rank in Q possible for this diagram
* d.loopsize() = the number of propagators in the loop
* d.vertices(field1, field2, ...) = number of vertices
    with the given fields
* d.loopvertices(field1, field2, ...) = number of vertices
    with the given fields; only those vertices which have
    at least one loop propagator attached to them
* d.iprop(field, momentum="...", twospin=..., massive=True/False,
                                                         color=...) =
    the number of propagators with the given properties:
     - field: a field or list of fields
     - momentum: a string denoting the momentum through this propagator,
            such as "k1+k2"
     - twospin: two times the spin (integer number)
     - massive: select only propagators with/without a non-zero mass
     - color: one of the numbers 1, 3, -3 or 8, or a list of
              these numbers
* d.chord(...) = number of loop propagators with the given properties;
    the arguments are the same as in iprop
* d.bridge(...) = number of non-loop propagators with the given
    properties; the arguments are the same as in iprop

See also: filter.nlo, select.lo, select.nlo
\end{verbatim}
\item[\texttt{filter.nlo}] (\textit{text})
\begin{verbatim}
A python function which provides a filter for loop diagrams.

See filter.lo for more explanation.
\end{verbatim}
\item[\texttt{filter.nnlo}] (\textit{text})
\begin{verbatim}
A python function which provides a filter for loop diagrams.

See filter.lo for more explanation.
\end{verbatim}
\item[\texttt{filter.module}] (\textit{text})
\begin{verbatim}
A python file of predefined functions which should be available
in filters.

Example:

filter.module=filter.py
filter.nlo=my_nlo_filter("vertices.txt")
filter.lo=my_nlo_filter("vertices.txt")

------ filter.py -----

class my_nlo_filter_class:
   def __init__(self, fname):
      self.fields = []
      f = open(fname, 'r')
      for line in f.readlines():
         fields = map(lambda s: s.strip(),
               line.split(","))
         self.fields.append(fields)
      f.close()

   def __call__(self, diag):
      for lst in self.fields:
         if diag.vertices(*lst) > 0:
            return False
      return True

----------------------

See filter.lo, filter.nlo
\end{verbatim}
\item[\texttt{renorm\_beta}] (\textit{true/false})
\begin{verbatim}
Sets the name of the same variable in config.f90

Activates or disables beta function renormalisation

QCD only
\end{verbatim}
Default: \verb|True|
\item[\texttt{renorm\_mqwf}] (\textit{true/false})
\begin{verbatim}
Sets the name of the same variable in config.f90

Activates or disables UV countertems coming from
external massive quarks

QCD only
\end{verbatim}
Default: \verb|True|
\item[\texttt{renorm\_decoupling}] (\textit{true/false})
\begin{verbatim}
Sets the name of the same variable in config.f90

Activates or disables UV counterterms coming from
massive quark loops

QCD only
\end{verbatim}
Default: \verb|True|
\item[\texttt{renorm\_mqse}] (\textit{true/false})
\begin{verbatim}
Sets the name of the same variable in config.f90

Activates or disables the UV counterterm coming from the
massive quark propagators

QCD only
\end{verbatim}
Default: \verb|True|
\item[\texttt{renorm\_logs}] (\textit{true/false})
\begin{verbatim}
Sets the name of the same variable in config.f90

Activates or disables the logarithmic finite terms
of all UV counterterms

QCD only
\end{verbatim}
Default: \verb|True|
\item[\texttt{renorm\_gamma5}] (\textit{true/false})
\begin{verbatim}
Sets the same variable in config.f90

Activates finite renormalisation for axial couplings in the
't Hooft-Veltman scheme

QCD only, works only with built-in model files.
\end{verbatim}
Default: \verb|True|
\item[\texttt{reduction\_interoperation}] (\textit{text})
\begin{verbatim}

   Default reduction library.

   Possible values are: ninja, samurai, golem95, pjfry (experimental)

   Sets the same variable in config.f90. A value of '-1' lets GoSam decide
   depending on reduction_libraries

   See common/config.f90 for details.
\end{verbatim}
Default: \verb|-1|
\item[\texttt{reduction\_interoperation\_rescue}] (\textit{text})
\begin{verbatim}

   Rescue reduction program.

   Sets the same variable in config.f90. A value of '-1' lets GoSam
   decide.

   See common/config.f90 for details.
\end{verbatim}
Default: \verb|-1|
\item[\texttt{samurai\_scalar}] (\textit{integer number})
\begin{verbatim}

   Sets the same variable in config.f90.

   See common/config.f90 for details.
\end{verbatim}
Default: \verb|2|
\item[\texttt{nlo\_prefactors}] (\textit{integer number})
\begin{verbatim}

   Sets the same variable in config.f90. The values have the
   following meaning:

   0: A factor of alpha_(s)/2/pi is not included in the NLO result
   1: A factor of 1/8/pi^2 is not included in the NLO result
   2: The NLO includes all prefactors

   Note, however, that the factor of 1/Gamma(1-eps) is not
   included in any of the cases.

   Please note, that nlo_prefactors=0 is enforced temporary in test.f90
   for better testing. In OLP interface mode (BLHA/BLHA2), the default is
   nlo_prefactors=2.
\end{verbatim}
Default: \verb|0|
\item[\texttt{PSP\_check}] (\textit{true/false})
\begin{verbatim}
Sets the same variable in config.f90

Activates Phase-Space Point test for the full amplitude.

!!Works only for QCD and with built-in model files!!
\end{verbatim}
Default: \verb|True|
\item[\texttt{PSP\_rescue}] (\textit{true/false})
\begin{verbatim}
Sets the same variable in config.f90

Activates Phase-Space Point rescue based on the estimated
accuracy on the finite part. It needs PSP_check=True.
The accuracy is estimated using information on the single
pole accuracy and on the stability of the finite parte
under rotation of the phase space point.

!!Works only for QCD and with built-in model files!!
\end{verbatim}
Default: \verb|True|
\item[\texttt{PSP\_verbosity}] (\textit{true/false})
\begin{verbatim}
Sets the same variable in config.f90

Sets the verbosity of the PSP_check.
verbosity = False: no output
verbosity = True : bad point are written in a file gs_badpts.log
!!Works only for QCD and with built-in model files!!
\end{verbatim}
Default: \verb|False|
\item[\texttt{PSP\_chk\_th1}] (\textit{integer number})
\begin{verbatim}
Sets the same variable in config.f90

Threshold to activate accept the point without further treatments.
The number has to be an integer indicating the wished minimum number
of digits accuracy on the pole. For poles more precise than this
threshold the finite part is not checked.
!!Works only for QCD and with built-in model files!!
\end{verbatim}
Default: \verb|8|
\item[\texttt{PSP\_chk\_th2}] (\textit{integer number})
\begin{verbatim}
Sets the same variable in config.f90

Threshold to declare a PSP as bad point, based of the precision of the pole.
Points with precision less than this threshold are directly reprocessed with
the rescue system (if available), or declared as unstable. According to the
verbosity level set, such points are written to a file and not used when
the code is interfaced to an external Monte Carlo using the new BLHA standards.
!!Works only for QCD and with built-in model files!!
\end{verbatim}
Default: \verb|3|
\item[\texttt{PSP\_chk\_th3}] (\textit{integer number})
\begin{verbatim}
Sets the same variable in config.f90

Threshold to declare a PSP as bad point, based on the precision of
the finite part estimated with a rotation. According to the
verbosity level set, such points are written to a file and not
used when the code is interfaced to an external Monte Carlo
using the new BLHA standards.
!!Works only for QCD and with built-in model files!!
\end{verbatim}
Default: \verb|5|
\item[\texttt{PSP\_chk\_kfactor}] (\textit{text})
\begin{verbatim}
Sets the same variable in config.f90

Threshold on the k-factor to perform a rotation check on the PSP.
!!Works only for QCD and with built-in model files!!
\end{verbatim}
Default: \verb|1000|
\item[\texttt{reference-vectors}] (\textit{comma separated list})
\begin{verbatim}
A list of reference vectors for massive and higher spin particles.
For vectors which are not assigned here, the program picks a
reference vector automatically.

Each entry of the list has to be of the form <index>:<index>

EXAMPLE

in=g,u
out=t,W+
reference-vectors=1:2,3:4,4:3

In this example, the gluon (particle 1) takes the momentum k2
as reference momentum for the polarisation vector. The massive
top quark (particle 3) uses the light-cone projection l4 of the
W-boson as reference direction for its own momentum splitting.
Similarly, the momentum of the W-boson is split into a direction
l4 and one along l3.

If cycles are generated in the list (l3 has to be known in order
to determine l4 and vice versa in the above example) they must be
at most of length two. For the reference momenta of lightlike
gauge bosons the length of cycles does not matter, e.g.

in=g,g
out=g,g
reference-vectors=1:2,2:3,3:4,4:1
\end{verbatim}
\item[\texttt{abbrev.limit}] (\textit{text})
\begin{verbatim}
Maximum number of instructions per subroutine when calculating
abbreviations, if this number is positive.
\end{verbatim}
Default: \verb|0|
\item[\texttt{templates}] (\textit{text})
\begin{verbatim}
Path pointing to the directory containing the template
files for the process. If not set, golem uses the directory
<golem_path>/templates.

The directory must contain a file called 'template.xml'
\end{verbatim}
\item[\texttt{qgraf.bin}] (\textit{text})
\begin{verbatim}
Points to the QGraf executable.

Example:
qgraf.bin=/home/my_user_name/bin/qgraf
\end{verbatim}
Default: \verb|qgraf|
\item[\texttt{form.bin}] (\textit{text})
\begin{verbatim}
Points to the Form executable.

Examples:
1) # Use TForm:
   form.bin=tform
2) # Use non-standard location:
   form.bin=/home/my_user_name/bin/form
\end{verbatim}
Default: \verb|tform|
\item[\texttt{form.threads}] (\textit{integer number})
\begin{verbatim}
Number of Form threads.

Example:
form.threads=4
runs tform, the parallel version of FORM, on 4 cores.
\end{verbatim}
Default: \verb|2|
\item[\texttt{form.tempdir}] (\textit{text})
\begin{verbatim}
Temporary directory for Form. Should point to a directory
on a local disk.

Examples:
form.tempdir=/tmp
form.tempdir=/scratch
\end{verbatim}
Default: \verb|/tmp|
\item[\texttt{form.workspace}] (\textit{integer number})
\begin{verbatim}
Size of the heap (in megabytes) used by FORM.

Example (for machines with <= 2GB RAM):
form.workspace=100
set WorkSpace to 100M in FORM via form.set file.
\end{verbatim}
Default: \verb|1000|
\item[\texttt{haggies.bin}] (\textit{text})
\begin{verbatim}
Points to the Haggies executable.
Haggies is used to transform the expressions of the diagrams
into optimized Fortran90 programs. It can be obtained from
   http://www.nikhef.nl/~thomasr/download.php

Examples:
   1) haggies.bin=/home/my_user_name/bin/haggies
   2) haggies.bin=/usr/bin/java -Xmx50m -jar ./haggies.jar
\end{verbatim}
Default: \verb|java -jar /Users/sj/Documents/repo/GoSam/Release/share/golem/haggies/haggies.jar|
\item[\texttt{fc.bin}] (\textit{text})
\begin{verbatim}
Denotes the executable file of the Fortran90 compiler.
\end{verbatim}
Default: \verb|gfortran|
\item[\texttt{python.bin}] (\textit{text})
\begin{verbatim}
Denotes the executable file of Python
\end{verbatim}
Default: \verb|python|
\item[\texttt{projectors}] (\textit{text})
\begin{verbatim}
path to the form file containing projectors
\end{verbatim}
Default: \verb|projectors.hh|
\item[\texttt{integral\_families\_1loop}] (\textit{text})
\begin{verbatim}
path to the yaml file containing the integral families
\end{verbatim}
Default: \verb|integral_families_1loop.yaml|
\item[\texttt{integral\_families\_2loop}] (\textit{text})
\begin{verbatim}
path to the yaml file containing the integral families
\end{verbatim}
Default: \verb|integral_families_2loop.yaml|
\item[\texttt{reduze.bin}] (\textit{text})
\begin{verbatim}
Points to reduze executable
\end{verbatim}
Default: \verb|reduze|
\item[\texttt{ninja.fcflags}] (\textit{text})
\begin{verbatim}
FCFLAGS required to compile with ninja.

Example:
ninja.fcflags=-I/usr/local/include/ninja
\end{verbatim}
\item[\texttt{ninja.ldflags}] (\textit{text})
\begin{verbatim}
LDFLAGS required to link ninja.

Example:
ninja.ldflags=-L/usr/local/lib/ -lninja
\end{verbatim}
\item[\texttt{samurai.fcflags}] (\textit{text})
\begin{verbatim}
FCFLAGS required to compile with samurai.

Example:
samurai.fcflags=-I/usr/local/include/samurai
\end{verbatim}
\item[\texttt{samurai.ldflags}] (\textit{text})
\begin{verbatim}
LDFLAGS required to link samurai.

Example:
samurai.ldflags=-L/usr/local/lib/ -lsamurai-gfortran-double
\end{verbatim}
\item[\texttt{golem95.fcflags}] (\textit{text})
\begin{verbatim}
FCFLAGS required to compile with golem95.

Example:
golem95.fcflags=-I/usr/local/include/golem95
\end{verbatim}
\item[\texttt{golem95.ldflags}] (\textit{text})
\begin{verbatim}
LDFLAGS required to link golem95.

Example:
golem95.ldflags=-L/usr/local/lib/ -lgolem-gfortran-double
\end{verbatim}
\item[\texttt{r2}] (\textit{text})
\begin{verbatim}
The algorithm how to treat the R2 term:

implicit    -- mu^2 terms are kept in the numerator and reduced
               at runtime
explicit    -- mu^2 terms are reduced analytically
off         -- all mu^2 terms are set to zero
\end{verbatim}
Default: \verb|explicit|
\item[\texttt{symmetries}] (\textit{comma separated list})
\begin{verbatim}
Specifies the symmetries of the amplitude.

This information is used when the list of helicities is generated.

Possible values are:

* flavour    -- no flavour changing interactions
         When calculating the list of helicities, fermion lines
    of PDGs 1-6 are assumed not to mix.

* family     -- flavour changing only within families
         When calculating the list of helicities, fermion lines
    of PDGs 1-6 are assumed to mix only within families,
    i.e. a quark line connecting a up with a down quark would
    be considered, while up-bottom is not.
* lepton     -- means for leptons what 'flavour' means for quarks
* generation -- means for leptons what 'family' means for quarks
* parity     -- the amplitude is invariant under parity tranformation.
                === Parity is not implemented yet.
* <n>=<h>    -- restriction of particle helicities,
         e.g. 1=-, 2=+ specifies helicities of particles 1 and 2
* %<n>=<h>   -- restriction by PDG code,
         e.g. %23=+- specifies the helicity of all Z-bosons to be
         '+' and '-' only (no '0' polarisation).

         %<n> refers to both +n and -n
         %+<n> refers to +n only
         %-<n> refers to -n only
\end{verbatim}
\item[\texttt{crossings}] (\textit{comma separated list})
\begin{verbatim}
A list of crossed processes derived from this process.

For each process in the list a module similar to matrix.f90 is
generated.

Example:

process_name=ddx_uux
in=1,-1
out=2,-2

crossings=dxd_uux: -1 1 > 2 -2, ud_ud: 2 1 > 2 1
\end{verbatim}
\end{description}
